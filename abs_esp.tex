%\begin{center}
%\large \bf \runtitulo
%\end{center}
%\vspace{1cm}
\chapter*{\runtitulo}

\noindent 
El presente proyecto de tesis consistió en un estudio y extensión del método previamente propuesto por Daniel Ciolek en su tesis de doctorado \cite{tesisDani}. Más precisamente, se trató de analizar carencias del algoritmo de exploración on-the-fly para problemas de Supervisory Control, cuya propiedad central era de tipo Non-blocking y, posteriormente analizados los problemas, afrontarlos con una nueva especificación e implementación del algoritmo. La funcionalidad fue incorporada al software MTSA\footnote{Modal Transition System Analyser, \href{https://bitbucket.org/lnahabedian/mtsa/src/master/^}{https://bitbucket.org/lnahabedian/mtsa/src/master/}}. Finalmente, se adaptó el algoritmo para construir directores en lugar de supervisores maximales, presentando así la primera implementación de síntesis de directores.

\bigskip

\noindent\textbf{Palabras claves:} Discrete Event Systems, Supervisory Control (no menos de 5!!).
