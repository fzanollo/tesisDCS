%\begin{center}
%\large \bf \runtitulo
%\end{center}
%\vspace{1cm}
\chapter*{\runtitulo}

\noindent 

Esta tesis presenta la primera implementación de síntesis de directores para resolver problemas de Control de Eventos Discretos con la propiedad central de tipo non-blocking.

El método aprovecha la naturaleza composicional del problema y su input compacto, minimizando la explosión exponencial al componer la planta total. Con el enfoque de exploración on-the-fly de forma best-first-search guiada por distintas heurísticas, se busca reducir la parte de la planta a componer. 

La implementación fue incorporada al software MTSA\footnote{Modal Transition System Analyser, \href{https://bitbucket.org/lnahabedian/mtsa/src/master/^}{https://bitbucket.org/lnahabedian/mtsa/src/master/}}, junto con una batería de tests para conservar su correctitud ante futuros cambios.
 


%El presente proyecto de tesis consistió en un estudio y extensión del método previamente propuesto en \cite{tesisDani}. Más precisamente, se diseñó un algoritmo de exploración on-the-fly para problemas de Discrete Event Control, cuya propiedad central era de tipo Non-blocking. 
%
%La funcionalidad fue incorporada al software MTSA\footnote{Modal Transition System Analyser, \href{https://bitbucket.org/lnahabedian/mtsa/src/master/^}{https://bitbucket.org/lnahabedian/mtsa/src/master/}}. 
%
%Finalmente, se adaptó el algoritmo para construir directores en lugar de supervisores maximales, presentando así la primera implementación de síntesis de directores.

\bigskip

\noindent\textbf{Palabras claves:} Sistemas de Eventos Discretos, Síntesis de Controladores, Control Dirigido, Control Supervisado, Síntesis on-the-fly, LTS.


