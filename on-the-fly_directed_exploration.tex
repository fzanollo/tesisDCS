%Subiría de nivel el “Nuevo algoritmo”. Trataría de explicar tempranamente la idea de construcción de núcleos de posibles zonas perdedoras o ganadoras optimistas y pesimistas y la propagación. Incluso con la formalización primero.

En esta sección presentamos el nuevo algoritmo \DCS, que realiza una exploración sobre la marcha del espacio de estados. Por medio de dicha exploración el algoritmo encuentra una solución al problema composicional de "supervisory control". También discutimos la correctitud y completitud del nuevo algoritmo \DCS. \\

\section{Nuestro enfoque}

Cabe aclarar que sólo podemos concluir que un loop es error cuando este ha sido explorado en su completitud. Esto es así debido a la naturaleza optimista de los problemas non-blocking.

Fue a causa de esta necesidad de marcar errores que decidimos diseñar un nuevo algoritmo con un invariante que consideramos clave para síntesis on-the-fly: Si con la información de lo explorado hasta el momento es posible concluir que un estado es ganador o perdedor en la planta totalmente explorada, debemos marcar ese estado antes de seguir explorando.

\textcolor{red}{Expandir esta sección}

\textcolor{red}{Poner que nunca queremos que el open quede vacío, siempre queremos concluir que el inicial es W o L.}

\section{Propuesta de nuevo algoritmo}

\lstset{escapeinside={(*@}{@*)}}
\lstset{numbers=left, numberstyle=\tiny, stepnumber=1, numbersep=5pt}
\begin{lstlisting}[language={pseudocode},label={lst:dcs},caption={Algoritmo de exploración dirigida on-the-fly.},float=ht, frame=single]
 function DCS($\E {=} (E, A_E^C)$,$\:\heuristic$):
   $\initial$ = $\<\init{e}^0,\ldots,\init{e}^n\>$
   $\structure = \trimlst{(\{ \initial \}, A_E, \emptyset, \initial, M_E \cap \{\initial\})}$
   $\Goals = \Errors = \Witness = \emptyset$
   $\NONE = \{\initial\}$
   if (isDeadlock($\initial$)):(*@\label{line:initialDead}@*)
   $\Errors = \{\initial\}$
   $\NONE = \emptyset$
   while $\initial \not\in \Errors \cup \Goals$:
     $(e,\l,e')$ = expandNext($\heuristic$) (*@\label{line:expand}@*)
     $S_{\structure'} = S_\structure\cup  \{ e' \}$
     $\structure' = \trimlst{(S_{\structure'} , A_E, \D_\structure \cup \{ e \step{\l}{} e' 
     \}, \initial, M_E \cap S_{\structure'} )}$
     if $e' \in \Errors $: (*@\label{line:isLosing}@*)
       propagateError($\{e'\}$)
     else if $e' \in \Goals$: (*@\label{line:isWinning}@*)
       propagateGoal($\{e'\}$)
     else if canReach($e, e', \structure$): (*@\label{line:isLoop}@*)
       $\SCC$ = getMaxLoop($e$,$e'$) (*@\label{line:getMaxLoop}@*)
       if canBeWinningLoop($\SCC$): (*@\label{line:CanBeWinning}@*)
         C = findNewGoalsIn($\SCC$) (*@\label{line:lookingForWinner}@*)
         $\Witness = \Witness \cup (C \cap M_{\structure'})$
         $\Goals = \Goals \cup C$
         $\NONE = \NONE \setminus C$
         propagateGoal(C)   (*@\label{line:propagateGoal2}@*)
       else:    (*@\label{line:CanBeLosing}@*)
         P = findNewErrorsIn($\SCC$)
         $\Errors = \Errors \cup P$
         $\NONE = \NONE \setminus P$
         propagateError(P) (*@\label{line:propagateError2}@*)
     $\structure = \structure'$
 
   if $\initial \in \Goals$:
     $r = rankStates(\structure)$ (*@\label{line:controller-start}@*)
     $return \, \lambda w \ldot \{ \, \l \mid \trimlst{\initial \runw{w}{\structure} e 
     \step{\l}{\structure} e'} \wedge e' \in \Goals\,$
     $\wedge (l \in A^C_E \then \l = bestControllable(s,r,\structure) 
     )\}$(*@\label{line:controller-end}@*)
   else:
     return UNREALIZABLE
\end{lstlisting}  
%invariante del ciclo:  Inv($states(\structure), \structure)$

\lstset{numbers=none, numberstyle=\tiny, stepnumber=1, numbersep=5pt}
\begin{lstlisting}[language={pseudocode},label={lst:dcs.propagate},caption={Algoritmos de propagación.},float=ht, frame=single]
 function propagateGoal($newGoals$):
   $C' = \emptyset$; $C$ = ancestorsNone($newGoals$)
   while $C'\neq C$:
     $C' = C$
     $C=C \setminus \{ s \in C \mid$ 
       forcedTo($s, S_{\unexploredToBottom{\structure'}}\setminus (C \cup \Goals),\unexploredToBottom{\structure'}$) $\vee$
       cannotReachGoalIn($s, C$)$\}$
   $\Goals = \Goals \cup C$
   $\NONE = \NONE \setminus C$

 procedure propagateError($newErrors$):
   $P$ = ancestorsNone($newErrors$)
   $C$ = $P$; $C' = \emptyset$
   while $C' \neq C$:
     $C' = C$
     $C=C \setminus \{ s \in C \mid$ 
     	forcedTo($s,\Errors,\unexploredToTop{\structure'}$) $\vee$
     	cannotReachGoalIn($s, C$)$\}$
   $P = P \setminus C$
   $\Errors = \Errors\ \cup P$
   $\NONE = \NONE \setminus P$
\end{lstlisting}


\begin{lstlisting}[language={pseudocode},label={lst:dcs.gather},caption={Confirmación de clasificaciones},float=ht, frame=single]
 function findNewGoalsIn($\SCC$):
   $C = \SCC$; $C' = \emptyset$
   while $C' \neq  C:$
     $C' = C$; $C'' = \emptyset$
     while $C'' \neq C:$
       $C'' = C$
       $C=C \setminus \{ s \in C \mid$ 
               forcedTo($s, S_{\unexploredToBottom{\structure'}}\setminus (C \cup \Goals),\unexploredToBottom{\structure'}$) $\vee$
               cannotBeReached($s,C$) $\}$
     $C = C \setminus \{ s \in C \mid $cannotReachGoalOrMarkedIn($s, C$)$\}$
   return C
  
 function findNewErrorsIn($\SCC$):
   if ($\exists s \in \SCC \ldot \trimlst{s \step{\l'}{\unexploredToTop{\structure'}}  s'} 
   \wedge (s' \notin \SCC \wedge s' \notin \Errors)$):
     return $\emptyset$
   else: 
     return $\SCC$
\end{lstlisting}



%VErsion anterior
%   let $(e,\l,e')\in (\D_{E}\setminus\D_{\structure})$ such that $e\in S_\structure\ 
%   \wedge$
%     ($\forall (s,\l',s')\in (\D_{E}\setminus\D_{\structure})\, s\in S_\structure \ldot$ 
%        $\heuristic$($e, \l, e'$) $>=$ $\heuristic$($s, \l', s'$))

% AUX FUNCTIONS
\begin{lstlisting}[language={pseudocode},label={lst:dcs.aux},caption={Métodos auxiliares},float=ht, frame=single]
 procedure expandNext($\heuristic$):
   let $\trimlst{(e,\l,e') \ldot e\in S_\structure \wedge e\step{\l}{E} e' \wedge   
       \neg e\step{\l}{\structure} e' \wedge e \in \NONE \wedge}$ 
       $\trimlst{\forall (s,\l',s') \ldot  s\in S_\structure \wedge s\step{\l}{E} s' \wedge \neg s\step{\l}{\structure} s' \wedge s \in \NONE}$ 
         $\implies \trimlst{\heuristic(e, \l, e') >=\heuristic(s, \l',   s'))}$
   if isDeadlock($e'$):
     $\Errors$ = $\Errors \cup \{e'\}$
   if $e'\notin \Errors \cup \Goals$:
     $\NONE = \NONE \cup \{e'\}$
   return $(e,\l,e')$
 
 function ancestorsNone(targets):
   return $\{ \, e \in \structure' \mid \exists e' \in targets \ldot \exists w \ldot \trimlst{e \runw{w}{\structure'} e'} \wedge$
          $ \nexists s \in w(e) \ldot s \neq e' \wedge \in \Goals \cup \Errors \}$
   
 function canBeWinningLoop(loop):
   return $(\marked{loop}{\structure'}) \vee$ 
          $(\exists s \in loop \ldot $canReachInOneStep($s, \structure, \Goals$)$)$
   
 function getMaxLoop($e$,$e'$):
   return $\{s \mid \exists w, w' \ldot \trimlst{e \runw{w}{\structure'} s} \wedge \trimlst{s \runw{w'}{\structure'} e'} \wedge$ 
          $\nexists s' \ldot (s' \in w(e) \vee s' \in w'(s)) \wedge s' \neq e' \wedge s' \in \Goals \cup \Errors \}$

 function forcedTo($s,Dest,Z$):
   return $(\exists \l_u \in A_Z^U \ldot \exists e \in Dest \ldot \trimlst{s \step{\l_u}{Z} e})\ \vee$
          $(\forall \l \in A_Z \  (\trimlst{s \step{\l}{Z} e} \Rightarrow e \in Dest))$    
    
 function cannotBeReached($s, C$):
    return $\nexists s' \in C , \exists \l \ldot s' \step{\l}{\structure'} s$
   
 function cannotReachGoalOrMarkedIn($s, C$)
    return $\trimlst{\nexists w \ldot s \runw{w}{C} s' \wedge s'\in C\ \wedge}$ 
           $($canReachInOneStep$(s', \structure, \Goals)$
           $\vee\ s'\in\M_{\structure'})$
           
 function cannotReachGoalIn($s, C$)
    return $\trimlst{\nexists w \ldot s \runw{w}{C} s' \wedge s'\in C\ \wedge}$ 
           canReachInOneStep$(s', \structure, \Goals)$
           
 function canReach($s', s$)
    return $\exists w \ldot s' \runw{w}{\structure'} s$

 function canReachInOneStep($s, Targets$)
    return $\exists l \ldot s \step{\l}{\structure'} s' \wedge s' \in Targets$

 function isDeadlock($s$)
    return $\nexists l \ldot s \step{\l}{E} s'$
\end{lstlisting}

\begin{lstlisting}[language={pseudocode},label={lst:dcs.aux},caption={Métodos de ranking},float=ht, frame=single]
 function rankStates($\structure$)
    $r = 0$; $W' = \Witness$; $W = \emptyset$
    while $W' \neq  W:$
      $\forall w \in W', $rank$(w) = r$
      $W = W \cup W'$
      $W'= \{ s \in (\Goals \setminus W) \mid \exists s' \in W \ldot s \step{}{\structure} s'\}$
      $r = r + 1$
    return rank

 function bestControllable($e, r, \structure$)
    return $\l \in A^C_E \ldot e \step{\l}{\structure} e' \wedge \nexists \l' \in A^C_E \ldot $
			$e \step{\l'}{\structure} e'', r(e'') \leq r(e')$

\end{lstlisting}


%
% function isDeadlock($e$):
%   return $\neg\exists \l \! \ldot \trimlst{e \step{\l}{E} e'}$
%
% function canReach($e, e', Z$):
%   return $\exists w \ldot \trimlst{e' \runw{w}{Z} e}$
%   
% function canReachInOneStep($s, Z, \S$):
%   return $\trimlst{\exists l \ldot s \step{\l}{Z} s' \wedge s' \in \S}$

\FloatBarrier

\section{Demostración de corectitud y completitud}
\begin{notation}
Decimos que un estado $s$ es ganador["winning"] (resp. perdedor["losing"]) en el problema $\E = 
(E, A_E^C)$ siendo $s$ el estado inicial de $E$ si hay una (resp. no hay una) solución para $\E$. Nos referimos a los estados ganadores y perdedores de $E$ cuando $A_E^C$ es inferible del contexto, también usamos $W_E$ y $L_E$ para denotar el conjunto de estados ganadores y perdedores de $\E$.
\end{notation}

El algoritmo (ver Listing~\ref{lst:dcs}) explora incrementalmente el espacio de estados de $E$ utilizando una estructura de exploración parcial ($ES$), añadiéndole una transición por vez.


\begin{definition}
[Exploración Parcial] \label{def:unexploredTo}
Sea $E = (S_E,$ $A_E, \D_E, \init{e}, M_E)$. Decimos que $ES$ es una exploración parcial de $E$ ($ES \subseteq E$) si $S_\structure \subseteq 
S_E$ y $\structure = (S_\structure,A_E, \D_\structure,\init{e},M_E 
\cap S_\structure)$, donde $ \D_\structure \subseteq (\D_E \cap 
(S_\structure \times A_E \times S_\structure))$. Escribimos $ES \subset E$ cuando $S_\structure \subset S_E$.
\end{definition}

Para explicar el algoritmo y argumentar su correctitud y completitud introducimos dos nuevos problemas de control para exploraciones parciales. Uno toma una visión optimista de la región no explorada ($\top$) asumiendo que todas las transiciones no exploradas llevan a un estado ganador. El otro toma una visión pesimista ($\bot$) asumiendo que las transiciones no exploradas llevan a estados perdedores.

\begin{definition}
[Problemas de Control $\top$ y $\bot$] \label{def:unexploredTo}

Sean $\E = (E, A_E^C)$, $E = (S_E,A_E,\D_E,\init{e},M_E)$ y $\structure = 
(S_\structure,A_E, \D_\structure,\init{e},M_E \cap S_\structure)$, y $\structure 
\subseteq E$.
\\
Definimos $\E_\top$ como $(\unexploredToTop{\structure}, A_E^C)$ donde 
$\unexploredToTop{\structure} = (S_\structure \cup \, \{\top\},A_E,\D_\top, 
\init{e}, 
(M_E \cap S_\structure)\, \cup \, \{\top\})$ y $\, \D_\top \, = \, \D_\structure 
\, 
\cup\, \{(s,\l, \top) 
\;$ $ | \; \exists s' \ldot (s, \l, s') \in (\D_E \setminus \D_\structure)\} \cup \{(\top, \l, \top) \, | \, \l \in A_E\}$ \\
Definimos $\E_\bot$ como $(\unexploredToBottom{\structure}, A_E^C)$ donde 
$\unexploredToBottom{\structure} = (S_\structure \, \cup \, 
\{\bot\},A_E,\D_\bot, 
\init{e}, M_E \cap S_\structure)$ y $\D_\bot = \D_\structure \, \cup \, \{(s,\l, 
\bot) \, | \, $ $ \exists s' \ldot (s, \l, s') \in (\D_E \setminus \D_\structure)\}$ 
\end{definition}

Usamos estos problemas de control para decidir tempranamente si un estado $s$ es ganador o perdedor en $E$ basado en lo que exploramos previamente en $\structure$. Si $s$ es ganador en $\unexploredToBottom{\structure}$ esto significa que sin importar a dónde lleven las transiciones no exploradas, $s$ también va a ser ganador en $E$. Similarmente, $s$ es perdedor en $E$ si es perdedor en $\unexploredToTop{\structure}$.
Lemma~\ref{lem:WESandLesMonotonicity} refuerza este razonamiento.


\begin{lemma}\textbf{\emph{(Monotonicidad de $\WES$ y $\LES$)}}
\label{lem:WESandLesMonotonicity}
Sean $\structure$ y $\structure'$ dos exploraciones parciales de $E$ tal que $\structure 
\subset \structure'$ entonces $\WES \subseteq \WESS$ y $\LES \subseteq 
\LESS$.
\end{lemma}

El algoritmo agrega iterativamente una transición de $E$ a $\structure$ a la vez y asegura que al final de cada iteración, los estados en $\structure$ están correcta y completamente clasificados en ganadores y perdedores si hay suficiente información de $E$ en $\structure$. Los conjuntos de estados $Errors$, 
$Goals$ y $\NONE$ se usan para este propósito.

\begin{property}[Invariante]
\label{def:invariant}
El loop principal del Algorithm~\ref{lst:dcs} tiene el siguiente invariante: 
$\structure \subseteq E \ \wedge $ $\forall s \in \structure \ldot (s\in\Goals 
\Leftrightarrow 
s \in 
\WES)$ $\wedge$  $(s\in\Errors 
\Leftrightarrow s \in \LES)$ $\wedge$  
$s\in\Errors\uplus\Goals\uplus\NONE$

% \begin{tabbing}
% ajfdlkadl\= \kill
% \> \= 
% \>$s\in\Goals \Leftrightarrow s \in \WES \wedge$ \\
% \> $s\in\Errors \Leftrightarrow s \in \LES \wedge$ \\
% \> $s\in\Errors\uplus\Goals\uplus\NONE$
% \end{tabbing}
\end{property}

La explicación del Algorithm~\ref{lst:dcs} que detallamos a continuación sirve también como un esquema de demostración para Property~\ref{def:invariant}.   

Para empezar, notar que la función \texttt{expandNext} 
(line~\ref{line:expand}) retorna una nueva transición $e \step{\l}{E} e'$ 
garantizando que $e$ ya se encontraba en $\structure$ y $e \in \NONE$. 
Esto significa que en cada iteración, hay algo de información nueva disponible para un estado que actualmente no está clasificado en ganador ni perdedor.

Si el estado $e'$ ya es clasificado como ganador en
$\unexploredToBottom{\structure}$ (line~\ref{line:isWinning}) o
perdedor en $\unexploredToTop{\structure}$ (line~\ref{line:isLosing}) 
entonces esta información necesita ser propagada a los estados en $\NONE$ para ver si pueden convertirse en ganadores en 
$\unexploredToBottom{\structure'}$ o perdedores en
$\unexploredToTop{\structure'}$. Tanto \texttt{propagateGoal} como
\texttt{propagateError} realizan un punto fijo estándar~\cite{Ramadge:1989:CDES} sobre
$\unexploredToBottom{\structure}$ y
$\unexploredToTop{\structure}$ pero solo sobre predecesores de $e'$ que están en $\NONE$. 
Lemma~\ref{lem:newWinnersLosersAreNonePredecessors} asegura la completitud de esta propagación restringida.

%Hack de titulo de lemma porque si no, no corta linea.
\begin{lemma}\textbf{\emph{(Ganadores/Perdedores nuevos tienen camino de estados-\textit{None} a transición nueva)}}
\label{lem:newWinnersLosersAreNonePredecessors}
Sea la transición $e \step{\l}{\structure} e'$ la única diferencia entre dos exploraciones parciales, $\structure$ y $\structure'$, de $E$. Si $s \notin (\WES \cup \LES)$ y $s \in (\WESS \cup \LESS)$, entonces hay $s_0, \ldots, s_n \notin (\WES \cup \LES)$ tal que $s = s_0 \wedge
s_0 \step{\l_0}{\structure}\ldots s_{n} \step{\l}{\structure} e'$.
\end{lemma}

Ya en la linea~\ref{line:isLoop} sabemos que $e'$ no es ganador en $\unexploredToBottom{\structure}$ ni perdedor en $\unexploredToTop{\structure}$, chequeamos si $e 
\step{\l}{\structure} e'$ cierra un nuevo loop. Si no es el caso, entonces no hay nada que hacer ya que $e'$ alcanza las mismas transiciones en $\structure'$ que en $\structure$. Entonces, $e' \notin 
(\WESS \cup \LESS)$ ya que cualquier supervisor para $e'$ en $\unexploredToBottom{\structure'}$ (resp. $\unexploredToTop{\structure'}$) es también un supervisor en $\unexploredToBottom{\structure}$ (resp. 
$\unexploredToTop{\structure}$) y vice versa.
%Thus, there is no new information about whether it is winning or 
%losing (Lemma~\ref{lem:noNewLoopImpliesNoNewInformation}). 
%
%%Hack de titulo de lemma porque si no, no corta linea.
%\begin{lemma}\textbf{\emph{(No new reachable transitions from e' implies no new 
%information)}}
%\label{lem:noNewLoopImpliesNoNewInformation}
%Let $e \step{\l}{} e'$ be the only difference between two partial explorations, 
%$\structure$ and $\structure'$ and $e' \notin \WES \cup \LES$. 
%If $\, \forall (s \step{\l'}{} s') \ldot e' \runw{w}{\structure} s  \step{\l'}{\structure} s' $ 
%if and only if $e' \runw{w}{\structure'} s  \step{\l'}{\structure'} s' $ then 
%$e' \notin \WESS \cup \LESS$
%\end{lemma}
%
Más aún, que no haya nueva información para $e'$ implica que no hay nuevos ganadores o perdedores (Lemma~\ref{lem:E'isNoneThenAllIsNone})

\begin{lemma}\textbf{\emph{(Nuevos ganadores/perdedores solo si $e'$ es un nuevo ganador/perdedor)}}
\label{lem:E'isNoneThenAllIsNone}
Sea $e \step{\l}{} e'$ la única diferencia entre dos exploraciones parciales, $\structure$ y $\structure'$. Si $\WESS \neq \WES \Rightarrow e' \in \WESS \setminus \WES$, y si $\LESS \neq \LES \Rightarrow e' \in \LESS \setminus \LES$. ENTONCES?
%Then for all $s$ such that 
%$\exists w \ldot s \runw{w}{\structure'} e'$, if $s \notin (\WES \cup \LES)$ then $s \notin (\WESS \cup \LESS)$.
\end{lemma}


Si se cerró un nuevo loop(line~\ref{line:isLoop}), por Lemma~\ref{lem:E'isNoneThenAllIsNone} alcanza con analizar si $e' 
\in \WESS \uplus \LESS$, y por Lemma~\ref{lem:newWinnersLosersAreNonePredecessors} propagar cualquier información nueva de $e'$ a sus predecesores. 

En la linea~\ref{line:getMaxLoop} computamos $\SCC$, el conjunto de estados que pertenecen a un loop que pasa por $e \step{\l}{\structure'} e'$ y nunca por $\WES \cup \LES$. 
Intuitivamente, cualquier supervisor para $e'$ va a depender de alguno de estos loops. O, en términos del Lemma~\ref{lem:newWinnersLosersAreNonePredecessors}, para que $e'$ cambie su estado, debe ser a través de un camino de estados $\NONE$. 

En la linea~\ref{line:CanBeWinning} usamos \texttt{canBeWinningLoop}($\SCC$) para chequear si existe algún estado marcado en $\SCC$ o si es posible escapar de $\SCC$ y alcanzar un "goal" en un paso. Esto distingue entre dos posibles opciones: $e' \in \WESS$ o $e' \in \LESS$ (ver 
Lemma~\ref{lem:canBeWinningLoopWorks}). 

\begin{lemma}\textbf{\emph{(Condición necesaria/suficiente para ganar/perder)}}
\label{lem:canBeWinningLoopWorks}
Sea $e \step{\l}{} e'$ la única diferencia entre dos eploraciones parciales, 
$\structure$ y $\structure'$. Sea $\SCC$ = \emph{\texttt{getMaxLoop($e$, 
$e'$)}}.
Si $e' \in \WESS \setminus \WES$ 
entonces \emph{\texttt{canBeWinningLoop}($\SCC$)}. Además, si \\ 
\emph{\texttt{canBeWinningLoop}($\SCC$)} entonces $e' \notin \LESS$ 
\end{lemma}


Si \texttt{canBeWinningLoop()} retorna true, en la  linea~\ref{line:lookingForWinner}, 
sabemos que si $e'$ cambia su estado es porque $e' \in \WESS$. Para ver si este cambio se produce, se realiza una computación estándar de punto fijo. 
%However, we introduce an optimization, namely that there must be new 
%$\NONE$-loop (i.e., a loop over states that are  $\NONE$)
%that includes the new transition that either reaches a marked state 
%or 
%can reach a winning state in one step. 
Sin embargo, el método \texttt{findNewGoalsIn} aplica una optimización basada en Lemma~\ref{lem:newWinnersLosersAreNonePredecessors}; solo considera estados que están en un $\NONE$-loop a través de la nueva transición ($\SCC$).

Si \texttt{canBeWinningLoop()} retorna false, entonces debemos comprobar si $e' \in \LESS$.
%  By Lemma~\ref{lem:newWinnersLosersAreNonePredecessors}
  %, we only consider states that are in a $\NONE$-loop 
%via the $e \step{\l}{} e'$ (i.e, the set $\SCC$). 
Esto puede hacerse de forma más eficiente que con un punto fijo usando el Lemma~\ref{lem:findErrorsWorks} que muestra que alcanza con observar si no es posible escapar de $\SCC$ alcanzando en un paso un estado que no esté en $\LES$. 



%Hack de titulo de lemma porque si no, no corta linea.
\begin{lemma}\textbf{\emph{(findNewErrorsIn es correcto y completo)}}
\label{lem:findErrorsWorks}
Si $\SCC =$ \emph{\texttt{getMaxLoop($e$, $e'$)}} $\wedge$\\ 
$\neg$\emph{\texttt{canBeWinningLoop}}($\SCC$) y \\
$P=\emph{\texttt{findNewErrorsIn}}(\SCC)$ entonces \\
$(e' \in \LESS \Rightarrow e' \in P 
\subseteq \LESS)$ $\wedge \, (e' \notin \LESS \Rightarrow P = 
\emptyset)$
\end{lemma}



Por motivos de eficiencia, \texttt{findNewGoalsIn} y
\texttt{findNewErrorsIn} no solo verifican si $e' \in \WESS$/$e' \in 
\LESS$ sino que también agregan estados ganadores/perdedores cuando pueden. La detección completa de nuevos estados ganadores y perdedores se hace finalmente con los procesos de propagación. 

Habiendo argumentado que la propiedad~~\ref{def:invariant} es válida, la correctitud y completitud le siguen de forma natural. 

Primero, notar que el algoritmo termina cuando logra determinar que $\init{e}$ está en $\LESS$ o $\WESS$. En el segundo caso, es simple construir un supervisor basándose en la estructura de exploración
$\structure$.\hfill$\qed$

\begin{theorem}[Correctitud y Completitud]
Sea $\E = (E,A_E^C)$ un problema de control composicional según Definition~\ref{def:control-problem}. Existe una solución para $\E$ si y solo si el algoritmo DCS retorna un supervisor para $\E$.
\end{theorem}

Demostración (Correctitud y completitud):
El teorema se desprende del invariante de ciclo del algoritmo (Definition~\ref{def:invariant}), el
Lemma~\ref{lem:WESandLesMonotonicity}, y el hecho de que en el peor caso todas las transiciones son agregadas a la estructura de exploración. Entonces,  $E = \structure = 
\unexploredToBottom{\structure} = \unexploredToTop{\structure}$.


\section{Demostración de Lemas}
\begin{proof}
	
(Idea:	Para probar $\WES \subseteq \WESS$ mostramos que un controlador para un estado $s$ en $\WES$ 
	puede ser usado como un controlador para $s$ en $\WESS$. Para $\LES \subseteq 
	\LESS$, asumimos que hay un estado $s \in \LES \setminus \LESS $. Llegamos a una contradicción mostrando que el controlador que $s$ debe tener en
	$\unexploredToTop{\structure'}$ es también un controlador para $s$ en $\unexploredToTop{\structure}$.)\\
	

Si $s \in \WES $ entonces existe un controlador $\sigma$ para el problema de control $\unexploredToBottom{\structure}$. Sea $Z$ tal que $\structure \subseteq Z$. Demostraremos que $\sigma$ es un controlador para $\unexploredToBottom{Z}$. Esto requiere dos condiciones según la Definición~\ref{def:control-problem}. La primera, que $\sigma$ es controlable, es trivial ya que los conjuntos de eventos controlables y no controlables no fueron cambiados. 

Para la segunda, nonblocking, primero mostramos que $\L^\sigma(\unexploredToBottom{Z}) = 
\L^\sigma(\unexploredToBottom{\structure})$. \\

Si asumimos que $\L^\sigma(\unexploredToBottom{Z}) \not\subseteq \L^\sigma(\unexploredToBottom{\structure})$ y $w \in 
\L^\sigma(\unexploredToBottom{Z}) \setminus \L^\sigma(\unexploredToBottom{\structure})$, la corrida que verifica $w$ debe permanecer siempre en $Z$ o alcanzar eventualmente un estado $deadlock$ en $\unexploredToBottom{Z}$. En cualquier caso, sea $w_0$ el prefijo más largo en $\structure$. 
Sabemos que $w_0$ es un prefijo no vacío de $w$. Sea $\ell$ tal que $w_0.\ell$ es un prefijo de $w$. 
Por la definición de $\unexploredToBottom{\structure}$, $w_0.\ell$ alcanza un estado $deadlock$ en $\unexploredToBottom{\structure}$. Esto es una contradicción, ya que $\sigma$ es un controlador para
$\unexploredToBottom{\structure}$. 

Para mostrar que $\L^\sigma(\unexploredToBottom{Z}) \supseteq \L^\sigma(\unexploredToBottom{\structure})$, asumimos que $w \in \L^\sigma(\unexploredToBottom{\structure})$. Si $w$ también está en $\L^\sigma(\structure)$ entonces debe pertenecer a $\L^\sigma(Z)$ y $\L^\sigma(\unexploredToBottom{Z})$. De otra forma, $w = w_0.\ell$ alcanza un estado $deadlock$ en $\unexploredToBottom{\structure}$. Como $w_0$ pertenece a $\L^\sigma(\structure)$, debe pertenecer también a $\L^\sigma(Z)$. Consideramos el estado $s$ alcanzado por  $w_0$ en $E$, debe tener una transición etiquetada como $\ell$ para justificar su inclusión en $\unexploredToBottom{\structure}$. En $Z$, el estado $s$ o tiene la transición y por lo tanto $w_0.\ell \in \L^\sigma(Z) \subseteq 
\L^\sigma(\unexploredToBottom{Z})$, o no tiene la transición, pero el estado en $\unexploredToBottom{Z}$ tiene una transición $\ell$ a un estado $deadlock$, por lo tanto $w_0.\ell \in 
L^\sigma(\unexploredToBottom{Z})$.

Ahora, sabiendo que $\L^\sigma(\unexploredToBottom{Z}) = 
\L^\sigma(\unexploredToBottom{\structure})$, procedemos a $nonblocking$. Sea una palabra $w \in 
\L^\sigma(\unexploredToBottom{Z})$ que no puede ser extendida con $w'$ tal que $w.w'$ se encuentra en $L^\sigma(\unexploredToBottom{Z})$ y alcanza un estado marcado de $\unexploredToBottom{Z}$. 
Como $w$ también se encuentra en $L^\sigma(\unexploredToBottom{\structure})$ entonces, como $\sigma$ es un controlador para $\unexploredToBottom{\structure}$, existe un $w'$ tal que
$w.w' \in L^\sigma(\unexploredToBottom{\structure}) = L^\sigma(\unexploredToBottom{Z})$ y alcanza un estado marcado. Notar que la corrida para $w.w'$ siempre se encuentra en $\structure$, lo que significa que la corrida también está en $\unexploredToBottom{Z}$. Finalmente llegamos a una contradicción.\\

Para demostrar que $\LES \subseteq \LESS$, asumimos que existe un estado $s \in \LES \setminus \LESS$. Como $s \notin \LESS$, tiene un controlador $\sigma$ en $\unexploredToTop{\structure'}$, pero $s \in \LES$ por lo que no puede existir un controlador válido $\sigma'$ para $s$ en $\unexploredToTop{\structure}$. Esto es falso, más aún, mostraremos que si $\sigma$ es un controlador para $s$ en $\unexploredToTop{\structure'}$, entonces hay un controlador válido $\sigma'$ para $s$ en cualquier $\unexploredToTop{Z}$ si $Z \subseteq \structure$.

$\sigma$ es un controlador valido en $\unexploredToTop{\structure'}$, por lo que cualquier palabra en $\L^\sigma(\unexploredToTop{\structure'})$ puede ser extendida para alcanzar un estado marcado. Solo hay una cantidad finita de estados en  $\unexploredToTop{\structure'}$, por lo que deben existir $w'$ y $w''$ tal que $w.w'.w'' \in \L^\sigma(\unexploredToTop{\structure'})$, $w.w'$ llega a un estado marcado, y $w.w'.w''$ llega al mismo estado que $w.w'$.

Si $w.w'.w''$ está en $\L^\sigma(Z)$, entonces no hay nada que hacer, es claro que $\sigma$ tiene la misma forma de extender $w$ en $\unexploredToTop{Z}$. Si no, notemos que $w.w'.w'' = w_0.l.w_1$ tal que $w_0$ es el prefijo más largo de $w.w'.w''$ en $\L^\sigma(Z)$, esto significa que $w_0.l$ alcanza el estado marcado ganador $\top$, y desde ahí toda extensión de la palabra solo puede permanecer en ese mismo estado, por lo tanto, $\sigma$ también es un controlador válido en  $\unexploredToTop{Z}$.
 
\begin{flushright}
$\square$
\end{flushright}

\end{proof}


\begin{proof}
	(Idea: Si $s$ no es un predecesor de $e'$, como $e \step{l}{\structure'} e'$ es la única diferencia entre $\structure$ y $\structure'$, entonces los descendientes de $s$ son los mismos, 
	por lo tanto sus posibles controladores en $\unexploredToTop{\structure'}$ y
	$\unexploredToBottom{\structure'}$ no cambiaron. Entonces, $s \notin \WESS \cup 
	\LESS$ lo cual es una contradicción.
	
	Como paso siguiente probamos que hay al menos un camino desde $s$ a $e'$ a través de estados $\NONE$ por contradicción asumiendo que todos los caminos a $e'$ en $\structure'$ atraviesan un estado $s' \in 
	(\WES \cup \LES)$. 
	Como $s \notin \LES$, hay un controlador $\sigma$ desde $s$ en $\unexploredToTop{\structure}$. $\sigma$ no va a alcanzar estados en $\LES$, luego para todo $s'$ que alcance, debe haber un controlador $\sigma_{s'}$ para $\unexploredToBottom{\structure}$ (por lo tanto $\sigma_{s'}$ evita la nueva transición $\l$).
	Usamos $\sigma$ y $\sigma_{s'}$ para construir un controlador para $s$ en $\unexploredToTop{\structure'}$ para mostrar que $s 
	\not\in \LESS$.
	Un controlador para $s$ en $\unexploredToBottom{\structure'}$ no puede existir porque de otra forma podríamos usarlo para construir un controlador para $s$ en 
	$\unexploredToBottom{\structure}$ usando un razonamiento similar al anterior. Esto significa que $s \in 
	\WES$ contradiciendo la hipótesis. )\\


Si un estado $s$ no se encuentra en $\WES \cup \LES$ es porque tiene un controlador $\sigma$ en $\unexploredToTop{\structure}$ pero no tiene uno para $\unexploredToBottom{\structure}$. Esto depende únicamente de los descendientes de $s$, dado que estos son los únicos estados que $\sigma$ puede alcanzar. Si $s$ no es un predecesor de $e'$, y $e \step{l}{\structure'} e'$ es la única diferencia entre $\structure$ y $\structure'$ entonces los descendientes de $s$ son los mismos, por lo que los posibles controladores no tuvieron ningún cambio, y $s$ sigue siendo NONE.

Lo que no es tan claro, es que $s$ no tiene nuevos controladores posibles si tiene un camino que puede alcanzar $e'$ pero solo pasando por al menos un estado de $\WES \cup \LES$. Asumiendo que debe pasar por estados en $\WES \cup \LES$ mostramos que:

\begin{itemize}
	\item Sabiendo que $s$ tenía un controlador $\sigma$ en $\unexploredToTop{\structure}$, mostramos que $s$ tiene un controlador válido $\sigma'$ en $\unexploredToTop{\structure'}$:
	
	$\sigma'(w) = \sigma(w)$ si no existe un $w_0$ sufijo de $w$ tal que $ s \runw{w_0}{\structure'} s_i \wedge s_i \in \LES \cup \WES$. 
	
	$\sigma'(w) = \sigma_{s_i}(w_1)$ donde $w_0$ es el sufijo más corto de $w = w_0.w_1$ tal que $ s \runw{w_0}{\structure'} s_i \wedge s_i \in \WES$. $\sigma_{s_i}$ es el controlador que sabemos que  $s_i$ tiene en $\unexploredToBottom{\structure}$ ya que $s_i \in \WES$, y que cada controlador válido en $\unexploredToBottom{\structure}$ es también válido en $\unexploredToTop{\structure}$.
	
	Como $\sigma$ es un controlador válido, sabemos que no puede alcanzar estados en $\LES$.
	
	Finalmente, es claro que $\sigma'$ es un controlador válido para $s$ en $\unexploredToTop{\structure'}$. Notar que $\sigma'$ no depende de la nueva transición.
	
	
	
	
	\item Sabiendo que $s$ no tiene controlador en $\unexploredToBottom{\structure}$, mostramos que $s$ no tiene controlador en $\unexploredToBottom{\structure'}$ asumiendo que tiene uno y llegando a una contradicción:
	
	Suponemos que existe un controlador $\sigma'$ para $s$ en $\unexploredToBottom{\structure'}$, y que $e \step{l}{\structure'} e'$ es la única diferencia entre $\structure$ y $\structure'$.
	
	Con $\sigma'$ construimos $\sigma$, un controlador para $s$ en $\unexploredToBottom{\structure}$.
	
	$\sigma(w) = \sigma'(w)$ si no existe un $w_0$ que sea prefijo de $w$ y que $s \runw{w_0}{\structure} s_i \wedge s_i \in \WES \cup \LES$. Como $\sigma'$ es un controlador válido, sabemos que no puede alcanzar estados en $\LESS$. Notar que $w$ no puede alcanzar $e \step{\l}{\structure'} e'$ porque $s$ no tiene un camino de estados $\NONE$ a $e'$.
	
	Si $w=w_0.w_1$ y $s \runw{w_0}{\structure} s' \wedge s' \in \WES$ entonces $\sigma(w_0.w_1) = \sigma_{s'}(w_1)$ donde $\sigma_{s'}$ es el controlador para $s'$ en $\unexploredToBottom{\structure}$. Notar que una vez que se alcanza $s'$ siempre seguimos $\sigma_{s'}$.
	
	Como $\sigma$ nunca alcanza la nueva transición sabemos que $\sigma$ es válido en $\unexploredToBottom{\structure}$.
		
	Vemos entonces que asumiendo que existe un controlador válido $\sigma'$ para $s$ en $\unexploredToBottom{\structure'}$ estamos implicando la existencia de un controlador $\sigma$ para $s$ en $\unexploredToBottom{\structure}$. ABS! \\
	
\end{itemize}
\begin{flushright}
	$\square$
\end{flushright}
\end{proof}

\begin{proof}
	(Idea: Asumiendo $e' \notin \WESS$, usamos un testigo $s$ de $\WESS \neq 
	\WES$ para llegar a una contradicción. El estado $s$ debe tener un controlador en 
	$\WESS$ que evita $e \step{\l}{} e'$, la única diferencia entre $\unexploredToBottom{\structure}$ y $\unexploredToBottom{\structure'}$. Este controlador entonces es también un controlador para $s$ en $\WES$ llegando a un absurdo. 
	
	Asumimos $e' \notin \LESS$ y usamos un testigo $s$ de $\LESS \neq \LES$ para llegar a una contradicción. Notar que como $e' \notin \LESS$, hay un controlador $\sigma$ desde $e'$ en  
	$\unexploredToTop{\structure'}$. Como $s \notin \LES$ también debe haber un controlador $\sigma'$ desde $s$ en $\unexploredToTop{\structure}$. Construimos un nuevo controlador para $\unexploredToTop{\structure'}$ desde $s$ que funciona exactamente como $\sigma'$ pero cuando alcanza $e \step{\l}{} e'$ se comporta como $\sigma$. Este nuevo controlador prueba que $s \notin \LESS$ lo cual es una contradicción.)\\


Probamos ambas implicaciones por contradicción. 

Primero asumimos que $e' \not \in 
\WESS \setminus \WES$. Notar que como $e' \notin \WES$ entonces $e' \notin \WESS$. Como $\WESS \neq \WES$ y por la monotonicidad del (Lemma\ref{lem:WESandLesMonotonicity}) debe existir un estado $s$ tal que $s \in \WESS \setminus 
\WES$, entonces $s$ debe tener un controlador en $\WESS $. Este controlador no puede alcanzar $e'$ porque si lo hiciera, debería haber un controlador para $e'$ y comenzamos asumiendo que $e' 
\notin \WESS$. Más aún, si el controlador alcanzara $e$, entonces $\l$ debe ser controlable (si fuera no controlable, el controlador alcanzaría $e'$ lo cual ya establecimos que no es posible). Entonces, el controlador evita $e \step{\l}{} e'$ lo que significa que debe ser también un controlador para $s$ en $\unexploredToBottom{\structure}$ (i.e.,  $s \in \WES$) y alcanzamos una contradicción.

Ahora asumimos $e' \not\in \LESS \setminus \LES$. Notar que como $e' \notin \LES$ entonces $e' \notin \LESS$. Sea $s \in \LESS$ y $s \not\in \LES$. Sabemos que desde $s$ debe haber un controlador $\sigma'$ para  $\unexploredToTop{\structure}$. Este controlador puede o ser también un controlador para $\structure$ o alcanzar el estado $\top$ en $\unexploredToTop{\structure}$. En el primer caso, es también un controlador en $\structure'$ y en $\unexploredToTop{\structure'}$, una contradicción. En el segundo caso, o usa una transición que no se encuentra ni en $\structure'$ ni en $\structure$, lo que significa que en $\unexploredToTop{\structure'}$ va a alcanzar un estado ganador $\top$; o usa una transición que se encuentra en $\structure'$ pero no en $\structure$ lo que lleva a $e'$. Como sabemos que $e'$ no se encuentra en $\LESS$, entonces cuenta con un controlador en  $\unexploredToTop{\structure'}$, entonces sabemos que existe un controlador $\sigma''$ que incluye tanto a  $\sigma'$ como al controlador para $e'$. Finalmente, $\sigma''$ es un controlador para $s$ en  $\unexploredToTop{\structure'}$, pero $s \notin \LESS$, ABS!	
\begin{flushright}
	$\square$
\end{flushright}
\end{proof}

\begin{proof}
	(Idea: Para probar que $e' \in \WESS \setminus \WES$ implica \texttt{canBeWinningLoop($\SCC$)}, asumimos \\ $\neg$\texttt{canBeWinningLoop($\SCC$)} y mostramos que $e' \notin \WESS 
	\setminus \WES$. Para esto, basta con ver que si
	$\neg$\texttt{canBeWinningLoop($\SCC$)} entonces para alcanzar un estado marcado desde $e'$ se debe salir de $\SCC$ a un estado $s \notin \SCC \cup \WES$ lo que implica $s \notin \WESS$ 
	ya que $s$ no tiene ningún camino de estados $\NONE$ que llegue a $e \step{\l}{} e'$  (Lemma~\ref{lem:newWinnersLosersAreNonePredecessors}).
	Como $s$ no tiene controlador en $\unexploredToBottom{\structure'}$, es imposible que $e'$ tenga uno. 
	
	Para probar que \texttt{canBeWinningLoop($\SCC$)} implica $e' \notin \LESS$ construimos un controlador $\sigma'$ para $e'$ en $\unexploredToTop{\structure'}$ de la siguiente forma:
	Para una traza que se quede dentro de $\SCC$, solo elegimos sucesores controlables que no estén en $\LES$. Notar que no puede haber sucesores no controlables en $\LES$ ya que
	$\SCC \cap \LES= \emptyset$. Tan pronto como la traza sale de $\SCC$ a un estado $s'$ usamos el controlador para $s'$ en $\unexploredToTop{\structure'}$. 
	Como $s'$ no puede alcanzar $e \step{\l}{\structure'} 
	e'$ usando estados $\NONE$, por el 
	Lemma~\ref{lem:newWinnersLosersAreNonePredecessors}, $s'$ debe tener el controlador que necesitamos.)	\\


Sea $e' \in \WESS \setminus \WES$ pero $\neg$\texttt{canBeWinningLoop($\SCC$)}.
Existe un controlador $\sigma$ para $e'$ en $\unexploredToBottom{\structure'}$, esto significa que debe existir un camino $w$ desde $e'$ hasta un estado marcado $m$. 
Dado que $\neg$\texttt{canBeWinningLoop($\SCC$)}, no hay estados marcados en $\SCC$, $w$ debe salir de $\SCC$. 
Sea $s$ el primer estado que alcanza $w$ fuera de $\SCC$, $s$ pertenece a $\LES \cup \NONE$, y no tiene un camino de estados $\NONE$ hasta $e'$ entonces según Lemma~\ref{lem:newWinnersLosersAreNonePredecessors} $s$ va a seguir sin cambiar su estado.
Dado que $s \notin \WESS$, $s$ no tiene un controlador en $\unexploredToBottom{\structure'}$, pero $\sigma$ acepta corridas que llevan a $s$, llegamos a un absurdo.

Asumiendo \texttt{canBeWinningLoop($\SCC$)} simplemente construimos un controlador $\sigma^4$ para $e'$ en $\unexploredToTop{\structure'}$ para probar que $e' \notin \LESS$

Definimos $\sigma^4$ tal que habilita todas las transiciones no controlables (para que sea $controllable$).

$\sigma^4(w_0.w_1) = \sigma_{s'}(w_1)$ si $w_0$ es el camino más corto tal que existe $s' \notin \SCC \wedge s'\notin \LES$ y $e' \runw{w_0}{\unexploredToTop{\structure'}} s'$.
En otro caso $e' \runw{w}{\unexploredToTop{\structure'}} p \wedge p \in \SCC$ entonces para toda $\l'$ controlable, $\l' \in\sigma^4(w)$ si y solo si $\exists  p' \ldot p \step{l'}{} p' \wedge p' \notin \LES$.
%todas las transiciones que van al loop mas todas las transiciones que salen a un estado NONE (o TOP), i.e. no error. 

Usamos los controladores $\sigma_{s'}$ donde $s'$ es tal que existe $s \in \SCC$ y $s\step{\l'}{\unexploredToTop{\structure'}} s'\wedge s' \notin \SCC \wedge s'\notin \LES$. Si no existe tal $s'$, sabemos que debe haber un estado marcado en $\SCC$ y $\sigma^4$ nunca abandona el conjunto $\SCC$ ya que todos los estados alcanzables desde $\SCC$ pertenecen a $\LES$.\\

Probaremos que $\sigma^4$ es $controllable$ y $non-blocking$.

Dado que habilitamos todas las transiciones no controlables, $\sigma^4$ es trivialmente $controllable$.

Para $non-blocking$, sea $w$ compatible con $\sigma^4$, mostraremos que puede ser extendido. Si $w = w_0.w_1$ y $w_0$ es la palabra más corta tal que existe una $s' \notin \SCC \wedge s' \notin \LES$ y $e' \runw{w_0}{\unexploredToTop{\structure'}} s'$. Entonces por definición de $\sigma^4$ sabemos que  $\sigma^4(w_0.w_1.w_2) = \sigma_{s'}(w_1.w_2)$ para todo $w_2$. Ya que $\sigma_{s'}$ es $non-blocking$, existe un $w_2$ tal que $\sigma_{s'}(w_1.w_2)$ alcanza un estado marcado. Entonces $\sigma^4(w_0.w_1)$ puede ser extendido para alcanzar ese estado marcado. 

En otro caso, $w$ nunca abandona $\SCC$. Debemos probar que para todo $\l'$ tal que $w.\l'$ sea consistente con $\sigma^4$, $w.\l'$ puede ser extendido con un $w'$ para alcanzar un estado marcado. Sea $p'$ tal que $e' \runw{w.\l'}{} p'$. Si $p' \notin \SCC$ entonces $\sigma^4(w.\l') = \sigma_{p'}(\lambda)$ y, como antes, sabemos que $\sigma_{p'}$ es $non-blocking$ entonces $w.l'$ puede extenderse para llegar a un estado marcado.

Si $p' \in \SCC$, entonces $w.l'$ puede extenderse para llegar a cualquier $s$ en $\SCC$. Sabemos que o existe un estado marcado en $\SCC$ o algún estado en $\WES$ es alcanzable en un paso desde $\SCC$, de cualquier forma podemos extender $w.l'$ para llegar a un estado marcado.\\
\begin{flushright}
	$\square$
\end{flushright}
\end{proof}


\begin{proof}
	(Idea: Dividimos la prueba según la estructura del  if/then/else de \texttt{findNewErrorsIn}. 
	En el caso de que el \texttt{if} sea true, es suficiente probar que $e' \notin \LESS$. Para esto, 
	construimos un controlador $\sigma'$ para $e'$ en $\unexploredToTop{\structure'}$ de la siguiente forma: Para una traza que se queda dentro de $\SCC$, solo tomamos sucesores controlables que no estén en $\LES$. Notar que no puede haber sucesores no controlables en $\LES$ ya que
	$\SCC \cap \LES= \emptyset$. Tan pronto como la traza sale de $\SCC$ al estado $s'$ usamos el controlador de $s'$ en $\unexploredToTop{\structure'}$. 
	Como $s'$ no puede alcanzar $e \step{\l}{\structure'} 
	e'$ usando estados $\NONE$, por el 
	Lemma~\ref{lem:newWinnersLosersAreNonePredecessors}, $s'$ debe tener tal controlador. 
	
	Cuando el \texttt{if} es false, alcanza con probar que $P = \SCC \subseteq \LESS$. Alcanzamos una contradicción asumiendo que $s \in \SCC \setminus \LESS$: Si $s 
	\notin \LESS$ entonces tiene un controlador $\sigma$ que acepta una traza $w$ alcanzando un estado marcado. Como no hay estados marcados en $\SCC$, $w$ alcanza un estado $s' \notin \SCC$. Como el
	\texttt{if} era false, $s' \in \LESS$ por lo que $\sigma$ no es un controlador.)\\
	


En primer lugar, sabemos que cada estado $s' \notin \SCC$ tal que $\exists s \in \SCC \ldot s 
\step{l}{} s'$, puede o ser y seguir siendo un estado perdedor ($s' \in \LES \wedge s' \in \LESS$) o es y seguirá siendo $\NONE$ (porque $s$ no es un predecesor$-\NONE$ de un estado en $\SCC$, de otra forma $s$ estaría en $\SCC$). 

Esto significa que ningún estado $s' \notin \SCC \wedge s' \notin \LES$ puede ser forzado a un estado en $\LESS$. Entonces, si alcanzamos un estado $\NONE$ sabemos que tiene un controlador válido $\sigma_{s'}$ en $\unexploredToTop{\structure'}$.

En el caso de que la declaración $\texttt{if}$ sea verdad, probaremos que $e' \notin \LESS$:

Usamos el mismo $\sigma^4$ de la demostración Lemma \ref{lem:canBeWinningLoopWorks}. Ya sabemos que $\sigma^4$ es tanto $controllable$ como $non-blocking$ en esta situación por el Lemma anterior.\\

De otra forma entramos en el bloque $\texttt{else}$:

Si $\nexists s \in \SCC \ldot \trimlst{s \step{\l'}{\unexploredToTop{\structure'}}  s'} \wedge (s' \notin \SCC \wedge s' \notin \Errors)$ probamos que $\forall s \in \SCC$, $s \in \LESS$

Sea $\sigma'$ un controlador para $s$ en $\unexploredToTop{\structure'}$ entonces $\exists w'$ tal que  $s \runw{\lambda.w'}{\unexploredToTop{\structure'}} e_m \wedge e_m \in M_{\unexploredToTop{\structure'}}$. Ya que no hay estados marcados en $\SCC$, partiendo desde $s$ y siguiendo $w'$ eventualmente se abandona $\SCC$.

Sea $w' = w_0.w_1$ tal que $w_0$ es la palabra más corta tal que $s \runw{w_0}{\unexploredToTop{\structure'}} s' \wedge s' \notin \SCC$. Dado que $s'\in \Errors \Rightarrow s' \in \LES$ no es posible que un controlador válido $\sigma'$ acepte palabras que alcancen ese estado. ABS! Entonces no hay controlador para $s$ en $\unexploredToTop{\structure'}$ lo que implica $\forall s \in \SCC$, $s \in \LESS$.
\begin{flushright}
	$\square$
\end{flushright}
\end{proof}
