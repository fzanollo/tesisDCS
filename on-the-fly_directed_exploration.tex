%Subiría de nivel el “Nuevo algoritmo”. Trataría de explicar tempranamente la idea de construcción de núcleos de posibles zonas perdedoras o ganadoras optimistas y pesimistas y la propagación. Incluso con la formalización primero.

En esta sección presentamos el nuevo algoritmo \DCS, que realiza una exploración sobre la marcha del espacio de estados. Por medio de dicha exploración el algoritmo encuentra una solución al problema composicional de "supervisory control". También discutimos la correctitud y completitud del nuevo algoritmo \DCS. \\

\section{Nuestro enfoque}

Cabe aclarar que sólo podemos concluir que un loop es error cuando este ha sido explorado en su completitud. Esto es así debido a la naturaleza optimista de los problemas non-blocking.

Fue a causa de esta necesidad de marcar errores que decidimos diseñar un nuevo algoritmo con un invariante que consideramos clave para síntesis on-the-fly: Si con la información de lo explorado hasta el momento es posible concluir que un estado es ganador o perdedor en la planta totalmente explorada, debemos marcar ese estado antes de seguir explorando.

\textcolor{red}{Expandir esta sección}

\textcolor{red}{Poner que nunca queremos que el open quede vacío, siempre queremos concluir que el inicial es W o L.}

\section{Propuesta de nuevo algoritmo}

\lstset{escapeinside={(*@}{@*)}}
\lstset{numbers=left, numberstyle=\tiny, stepnumber=1, numbersep=5pt}
\begin{lstlisting}[language={pseudocode},label={lst:dcs},caption={On-the-fly Directed Exploration Procedure.},float=ht, frame=single]
 function DCS($\E {=} (E, A_E^C)$,$\:\heuristic$):
   $\initial$ = $\<\init{e}^0,\ldots,\init{e}^n\>$
   $\structure = \trimlst{(\{ \initial \}, A_E, \emptyset, \initial, M_E \cap \{\initial\})}$
   $\Goals = \Errors = \Witness = \emptyset$
   $\NONE = \{\initial\}$
   if (isDeadlock($\initial$)):(*@\label{line:initialDead}@*)
   $\Errors = \{\initial\}$
   $\NONE = \emptyset$
   while $\initial \not\in \Errors \cup \Goals$:
     $(e,\l,e')$ = expandNext($\heuristic$) (*@\label{line:expand}@*)
     $S_{\structure'} = S_\structure\cup  \{ e' \}$
     $\structure' = \trimlst{(S_{\structure'} , A_E, \D_\structure \cup \{ e \step{\l}{} e' 
     \}, \initial, M_E \cap S_{\structure'} )}$
     if $e' \in \Errors $: (*@\label{line:isLosing}@*)
       propagateError($\{e'\}$)
     else if $e' \in \Goals$: (*@\label{line:isWinning}@*)
       propagateGoal($\{e'\}$)
     else if canReach($e, e', \structure$): (*@\label{line:isLoop}@*)
       $\SCC$ = getMaxLoop($e$,$e'$) (*@\label{line:getMaxLoop}@*)
       if canBeWinningLoop($\SCC$): (*@\label{line:CanBeWinning}@*)
         C = findNewGoalsIn($\SCC$) (*@\label{line:lookingForWinner}@*)
         $\Witness = \Witness \cup (C \cap M_{\structure'})$
         $\Goals = \Goals \cup C$
         $\NONE = \NONE \setminus C$
         propagateGoal(C)   (*@\label{line:propagateGoal2}@*)
       else:    (*@\label{line:CanBeLosing}@*)
         P = findNewErrorsIn($\SCC$)
         $\Errors = \Errors \cup P$
         $\NONE = \NONE \setminus P$
         propagateError(P) (*@\label{line:propagateError2}@*)
     $\structure = \structure'$
 
   if $\initial \in \Goals$:
     $r = rankStates(\structure)$ (*@\label{line:controller-start}@*)
     $return \, \lambda w \ldot \{ \, \l \mid \trimlst{\initial \runw{w}{\structure} e 
     \step{\l}{\structure} e'} \wedge e' \in \Goals\,$
     $\wedge (l \in A^C_E \then \l = bestControllable(s,r,\structure) 
     )\}$(*@\label{line:controller-end}@*)
   else:
     return UNREALIZABLE
\end{lstlisting}  
%invariante del ciclo:  Inv($states(\structure), \structure)$

\lstset{numbers=none, numberstyle=\tiny, stepnumber=1, numbersep=5pt}
\begin{lstlisting}[language={pseudocode},label={lst:dcs.propagate},caption={Status propagation procedures.},float=ht, frame=single]
 function propagateGoal($newGoals$):
   $C' = \emptyset$; $C$ = ancestorsNone($newGoals$)
   while $C'\neq C$:
     $C' = C$
     $C=C \setminus \{ s \in C \mid$ 
       isForcedToLose($s,C$) $\vee$
       cannotReachGoalIn($s, C$)$\}$
   $\Goals = \Goals \cup C$
   $\NONE = \NONE \setminus C$

 procedure propagateError($newErrors$):
   $P$ = ancestorsNone($newErrors$)
   $C$ = $P$; $C' = \emptyset$
   while $C' \neq C$:
     $C' = C$
     $C=C \setminus \{ s \in C \mid$ 
     	$(\exists e \in \Errors \ldot $forcedTo($s,e,\unexploredToTop{\structure'}$)$)\ \vee$
     	cannotReachGoalIn($s, C$)$\}$
   $P = P \setminus C$
   $\Errors = \Errors\ \cup P$
   $\NONE = \NONE \setminus P$
\end{lstlisting}


\begin{lstlisting}[language={pseudocode},label={lst:dcs.gather},caption={Status confirmation.},float=ht, frame=single]
 function findNewGoalsIn($\SCC$):
   $C = \SCC$; $C' = \emptyset$
   while $C' \neq  C:$
     $C' = C$; $C'' = \emptyset$
     while $C'' \neq C:$
       $C'' = C$
       $C=C \setminus \{ s \in C \mid$ 
               isForcedToLose($s,C$) $\vee$
               cannotBeReached($s,C$) $\}$
     $C = C \setminus \{ s \in C \mid $cannotReachGoalOrMarkedIn($s, C$)$\}$
   return C
  
 function findNewErrorsIn($\SCC$):
   if ($\exists s \in \SCC \ldot \trimlst{s \step{\l'}{\unexploredToTop{\structure'}}  s'} 
   \wedge (s' \notin \SCC \wedge s' \notin \Errors)$):
     return $\emptyset$
   else: 
     return $\SCC$
\end{lstlisting}



%VErsion anterior
%   let $(e,\l,e')\in (\D_{E}\setminus\D_{\structure})$ such that $e\in S_\structure\ 
%   \wedge$
%     ($\forall (s,\l',s')\in (\D_{E}\setminus\D_{\structure})\, s\in S_\structure \ldot$ 
%        $\heuristic$($e, \l, e'$) $>=$ $\heuristic$($s, \l', s'$))

% AUX FUNCTIONS
\begin{lstlisting}[language={pseudocode},label={lst:dcs.aux},caption={auxiliary procedures.},float=ht, frame=single]
 procedure expandNext($\heuristic$):
   let $\trimlst{(e,\l,e') \ldot e\in S_\structure \wedge e\step{\l}{E} e' \wedge   
  \neg e\step{\l}{\structure} e' \wedge }$ 
           $\trimlst{(\forall s,\l',s') \ldot  s\in S_\structure \wedge s\step{\l}{E} s' \wedge   
    \neg s\step{\l}{\structure} s' \implies}$ 
                  $\trimlst{\heuristic(e, \l, e') >=\heuristic(s, \l',   s'))}$
   if isDeadlock($e'$):
     $\Errors$ = $\Errors \cup \{e'\}$
   if $e'\notin \Errors \cup \Goals$:
     $\NONE = \NONE \cup \{e'\}$
   return $(e,\l,e')$
 
 function ancestorsNone(targets):
   return $\{ \, e \in \structure' \mid \exists e' \in targets \ldot \exists w \ldot \trimlst{e \runw{w}{\structure'} e'} \wedge$
          $ \nexists s \in w(e) \ldot s \neq e' \wedge \in \Goals \cup \Errors \}$
   
 function canBeWinningLoop(loop):
   return $(\marked{loop}{\structure'}) \vee$ 
          $(\exists s \in loop \ldot $canReachInOneStep($s, \structure, \Goals$)$)$
   
 function getMaxLoop($e$,$e'$):
   return $\{s \mid \exists w, w' \ldot \trimlst{e \runw{w}{\structure'} s} \wedge \trimlst{s \runw{w'}{\structure'} e'} \wedge$ 
          $\nexists s' \ldot (s' \in w(e) \vee s' \in w'(s)) \wedge s' \neq e' \wedge s' \in \Goals \cup \Errors \}$

 function forcedTo($s,e,Z$):
   return $(\exists \l_u \in A_Z^U \ldot \trimlst{s \step{\l_u}{Z} e})\ \vee$
          $(\forall \l_c \in A_Z^C \ldot \trimlst{s \step{\l_c}{Z} e'} \then e' = e)$ 
          
 function isForcedToLose($s, C$):
    return $\exists e \ldot$ forcedTo($s,e,\unexploredToBottom{\structure'}$)$\wedge \,
    e\notin (C\cup\Goals)$
    
 function cannotBeReached($s, C$):
    return $\nexists s' \in C , \exists \l \ldot s' \step{\l}{\structure'} s$
   
 function cannotReachGoalOrMarkedIn($s, C$)
    return $\trimlst{\nexists w \ldot s \runw{w}{C} s' \wedge s'\in C\ \wedge}$ 
           $($canReachInOneStep$(s', \structure, \Goals)$
           $\vee\ s'\in\M_{\structure'})$
           
 function cannotReachGoalIn($s, C$)
    return $\trimlst{\nexists w \ldot s \runw{w}{C} s' \wedge s'\in C\ \wedge}$ 
           canReachInOneStep$(s', \structure, \Goals)$
           
 function canReach($s', s$)
    return $\exists w \ldot s' \runw{w}{\structure'} s$

 function canReachInOneStep($s, Targets$)
    return $\exists l \ldot s \step{\l}{\structure'} s' \wedge s' \in Targets$

 function isDeadlock($s$)
    return $\nexists l \ldot s \step{\l}{E} s'$
\end{lstlisting}

\begin{lstlisting}[language={pseudocode},label={lst:dcs.aux},caption={ranking procedures.},float=ht, frame=single]
 function rankStates($\structure$)
    $r = 0$; $W' = \Witness$; $W = \emptyset$
    while $W' \neq  W:$
      $\forall w \in W', $rank$(w) = r$
      $W = W \cup W'$
      $W'= \{ s \in (\Goals \setminus W) \mid \exists s' \in W \ldot s \step{}{\structure} s'\}$
      $r = r + 1$
    return rank

 function bestControllable($e, r, \structure$)
    return $\l \in A^C_E \ldot e \step{\l}{\structure} e' \wedge \nexists \l' \in A^C_E \ldot $
			$e \step{\l'}{\structure} e'', r(e'') \leq r(e')$

\end{lstlisting}


%
% function isDeadlock($e$):
%   return $\neg\exists \l \! \ldot \trimlst{e \step{\l}{E} e'}$
%
% function canReach($e, e', Z$):
%   return $\exists w \ldot \trimlst{e' \runw{w}{Z} e}$
%   
% function canReachInOneStep($s, Z, \S$):
%   return $\trimlst{\exists l \ldot s \step{\l}{Z} s' \wedge s' \in \S}$

\FloatBarrier

\section{Demostración de corectitud y completitud}
\begin{notation}
Decimos que un estado $s$ es ganador["winning"] (resp. perdedor["losing"]) en el problema $\E = 
(E, A_E^C)$ siendo $s$ el estado inicial de $E$ si hay una (resp. no hay una) solución para $\E$. Nos referimos a los estados ganadores y perdedores de $E$ cuando $A_E^C$ es inferible del contexto, también usamos $W_E$ y $L_E$ para denotar el conjunto de estados ganadores y perdedores de $\E$.
\end{notation}

El algoritmo (ver Listing~\ref{lst:dcs}) explora incrementalmente el espacio de estados de $E$ utilizando una estructura de exploración parcial ($ES$), añadiéndole una transición por vez.


\begin{definition}
[Exploración Parcial] \label{def:unexploredTo}
Sea $E = (S_E,$ $A_E, \D_E, \init{e}, M_E)$. Decimos que $ES$ es una exploración parcial de $E$ ($ES \subseteq E$) si $S_\structure \subseteq 
S_E$ y $\structure = (S_\structure,A_E, \D_\structure,\init{e},M_E 
\cap S_\structure)$, donde $ \D_\structure \subseteq (\D_E \cap 
(S_\structure \times A_E \times S_\structure))$. Escribimos $ES \subset E$ cuando $S_\structure \subset S_E$.
\end{definition}

Para explicar el algoritmo y argumentar su correctitud y completitud introducimos dos nuevos problemas de control para exploraciones parciales. Uno toma una visión optimista de la región no explorada ($\top$) asumiendo que todas las transiciones no exploradas llevan a un estado ganador. El otro toma una visión pesimista ($\bot$) asumiendo que las transiciones no exploradas llevan a estados perdedores.

\begin{definition}
[Problemas de Control $\top$ y $\bot$] \label{def:unexploredTo}

Sean $\E = (E, A_E^C)$, $E = (S_E,A_E,\D_E,\init{e},M_E)$ y $\structure = 
(S_\structure,A_E, \D_\structure,\init{e},M_E \cap S_\structure)$, y $\structure 
\subseteq E$.
\\
Definimos $\E_\top$ como $(\unexploredToTop{\structure}, A_E^C)$ donde 
$\unexploredToTop{\structure} = (S_\structure \cup \, \{\top\},A_E,\D_\top, 
\init{e}, 
(M_E \cap S_\structure)\, \cup \, \{\top\})$ y $\, \D_\top \, = \, \D_\structure 
\, 
\cup\, \{(s,\l, \top) 
\;$ $ | \; \exists s' \ldot (s, \l, s') \in (\D_E \setminus \D_\structure)\} \cup \{(\top, \l, \top) \, | \, \l \in A_E\}$ \\
Definimos $\E_\bot$ como $(\unexploredToBottom{\structure}, A_E^C)$ donde 
$\unexploredToBottom{\structure} = (S_\structure \, \cup \, 
\{\bot\},A_E,\D_\bot, 
\init{e}, M_E \cap S_\structure)$ y $\D_\bot = \D_\structure \, \cup \, \{(s,\l, 
\bot) \, | \, $ $ \exists s' \ldot (s, \l, s') \in (\D_E \setminus \D_\structure)\}$ 
\end{definition}

Usamos estos problemas de control para decidir tempranamente si un estado $s$ es ganador o perdedor en $E$ basado en lo que exploramos previamente en $\structure$. Si $s$ es ganador en $\unexploredToBottom{\structure}$ esto significa que sin importar a dónde lleven las transiciones no exploradas, $s$ también va a ser ganador en $E$. Similarmente, $s$ es perdedor en $E$ si es perdedor en $\unexploredToTop{\structure}$.
Lemma~\ref{lem:WESandLesMonotonicity} refuerza este razonamiento.


\begin{lemma}\textbf{\emph{(Monotonicidad de $\WES$ y $\LES$)}}
\label{lem:WESandLesMonotonicity}
Sean $\structure$ y $\structure'$ dos exploraciones parciales de $E$ tal que $\structure 
\subset \structure'$ entonces $\WES \subseteq \WESS$ y $\LES \subseteq 
\LESS$.
\end{lemma}

El algoritmo agrega iterativamente una transición de $E$ a $\structure$ a la vez y asegura que al final de cada iteración, los estados en $\structure$ están correcta y completamente clasificados en ganadores y perdedores si hay suficiente información de $E$ en $\structure$. Los conjuntos de estados $Errors$, 
$Goals$ y $\NONE$ se usan para este propósito.

\begin{property}[Invariante]
\label{def:invariant}
El loop principal del Algorithm~\ref{lst:dcs} tiene el siguiente invariante: 
$\structure \subseteq E \ \wedge $ $\forall s \in \structure \ldot (s\in\Goals 
\Leftrightarrow 
s \in 
\WES)$ $\wedge$  $(s\in\Errors 
\Leftrightarrow s \in \LES)$ $\wedge$  
$s\in\Errors\uplus\Goals\uplus\NONE$

% \begin{tabbing}
% ajfdlkadl\= \kill
% \> \= 
% \>$s\in\Goals \Leftrightarrow s \in \WES \wedge$ \\
% \> $s\in\Errors \Leftrightarrow s \in \LES \wedge$ \\
% \> $s\in\Errors\uplus\Goals\uplus\NONE$
% \end{tabbing}
\end{property}

La explicación del Algorithm~\ref{lst:dcs} que detallamos a continuación sirve también como un esquema de demostración para Property~\ref{def:invariant}.   

Para empezar, notar que la función \texttt{expandNext} 
(line~\ref{line:expand}) retorna una nueva transición $e \step{\l}{E} e'$ 
garantizando que $e$ ya se encontraba en $\structure$ y $e \in \NONE$. 
Esto significa que en cada iteración, hay algo de información nueva disponible para un estado que actualmente no está clasificado en ganador ni perdedor.

Si el estado $e'$ ya es clasificado como ganador en
$\unexploredToBottom{\structure}$ (line~\ref{line:isWinning}) o
perdedor en $\unexploredToTop{\structure}$ (line~\ref{line:isLosing}) 
entonces esta información necesita ser propagada a los estados en $\NONE$ para ver si pueden convertirse en ganadores en 
$\unexploredToBottom{\structure'}$ o perdedores en
$\unexploredToTop{\structure'}$. Tanto \texttt{propagateGoal} como
\texttt{propagateError} realizan un punto fijo estándar~\cite{Ramadge:1989:CDES} sobre
$\unexploredToBottom{\structure}$ y
$\unexploredToTop{\structure}$ pero solo sobre predecesores de $e'$ que están en $\NONE$. 
Lemma~\ref{lem:newWinnersLosersAreNonePredecessors} asegura la completitud de esta propagación restringida.

%Hack de titulo de lemma porque si no, no corta linea.
\begin{lemma}\textbf{\emph{(Ganadores/Perdedores nuevos tienen camino de estados-\textit{None} a transición nueva)}}
\label{lem:newWinnersLosersAreNonePredecessors}
Sea la transición $e \step{\l}{\structure} e'$ la única diferencia entre dos exploraciones parciales, $\structure$ y $\structure'$, de $E$. Si $s \notin (\WES \cup \LES)$ y $s \in (\WESS \cup \LESS)$, entonces hay $s_0, \ldots, s_n \notin (\WES \cup \LES)$ tal que $s = s_0 \wedge
s_0 \step{\l_0}{\structure}\ldots s_{n} \step{\l}{\structure} e'$.
\end{lemma}

Ya en la linea~\ref{line:isLoop} sabemos que $e'$ no es ganador en $\unexploredToBottom{\structure}$ ni perdedor en $\unexploredToTop{\structure}$, chequeamos si $e 
\step{\l}{\structure} e'$ cierra un nuevo loop. Si no es el caso, entonces no hay nada que hacer ya que $e'$ alcanza las mismas transiciones en $\structure'$ que en $\structure$. Entonces, $e' \notin 
(\WESS \cup \LESS)$ ya que cualquier supervisor para $e'$ en $\unexploredToBottom{\structure'}$ (resp. $\unexploredToTop{\structure'}$) es también un supervisor en $\unexploredToBottom{\structure}$ (resp. 
$\unexploredToTop{\structure}$) y vice versa.
%Thus, there is no new information about whether it is winning or 
%losing (Lemma~\ref{lem:noNewLoopImpliesNoNewInformation}). 
%
%%Hack de titulo de lemma porque si no, no corta linea.
%\begin{lemma}\textbf{\emph{(No new reachable transitions from e' implies no new 
%information)}}
%\label{lem:noNewLoopImpliesNoNewInformation}
%Let $e \step{\l}{} e'$ be the only difference between two partial explorations, 
%$\structure$ and $\structure'$ and $e' \notin \WES \cup \LES$. 
%If $\, \forall (s \step{\l'}{} s') \ldot e' \runw{w}{\structure} s  \step{\l'}{\structure} s' $ 
%if and only if $e' \runw{w}{\structure'} s  \step{\l'}{\structure'} s' $ then 
%$e' \notin \WESS \cup \LESS$
%\end{lemma}
%
Más aún, que no haya nueva información para $e'$ implica que no hay nuevos ganadores o perdedores (Lemma~\ref{lem:E'isNoneThenAllIsNone})

\begin{lemma}\textbf{\emph{(Nuevos ganadores/perdedores solo si $e'$ es un nuevo ganador/perdedor)}}
\label{lem:E'isNoneThenAllIsNone}
Sea $e \step{\l}{} e'$ la única diferencia entre dos exploraciones parciales, $\structure$ y $\structure'$. Si $\WESS \neq \WES \Rightarrow e' \in \WESS \setminus \WES$, y si $\LESS \neq \LES \Rightarrow e' \in \LESS \setminus \LES$. ENTONCES?
%Then for all $s$ such that 
%$\exists w \ldot s \runw{w}{\structure'} e'$, if $s \notin (\WES \cup \LES)$ then $s \notin (\WESS \cup \LESS)$.
\end{lemma}


Si se cerró un nuevo loop(line~\ref{line:isLoop}), por Lemma~\ref{lem:E'isNoneThenAllIsNone} alcanza con analizar si $e' 
\in \WESS \uplus \LESS$, y por Lemma~\ref{lem:newWinnersLosersAreNonePredecessors} propagar cualquier información nueva de $e'$ a sus predecesores. 

En la linea~\ref{line:getMaxLoop} computamos $\SCC$, el conjunto de estados que pertenecen a un loop que pasa por $e \step{\l}{\structure'} e'$ y nunca por $\WES \cup \LES$. 
Intuitivamente, cualquier supervisor para $e'$ va a depender de alguno de estos loops. O, en términos del Lemma~\ref{lem:newWinnersLosersAreNonePredecessors}, para que $e'$ cambie su estado, debe ser a través de un camino de estados $\NONE$. 

En la linea~\ref{line:CanBeWinning} usamos \texttt{canBeWinningLoop}($\SCC$) para chequear si existe algún estado marcado en $\SCC$ o si es posible escapar de $\SCC$ y alcanzar un "goal" en un paso. Esto distingue entre dos posibles opciones: $e' \in \WESS$ o $e' \in \LESS$ (ver 
Lemma~\ref{lem:canBeWinningLoopWorks}). 

\begin{lemma}\textbf{\emph{(Condición necesaria/suficiente para ganar/perder)}}
\label{lem:canBeWinningLoopWorks}
Sea $e \step{\l}{} e'$ la única diferencia entre dos eploraciones parciales, 
$\structure$ y $\structure'$. Sea $\SCC$ = \emph{\texttt{getMaxLoop($e$, 
$e'$)}}.
Si $e' \in \WESS \setminus \WES$ 
entonces \emph{\texttt{canBeWinningLoop}($\SCC$)}. Además, si \\ 
\emph{\texttt{canBeWinningLoop}($\SCC$)} entonces $e' \notin \LESS$ 
\end{lemma}


Si \texttt{canBeWinningLoop()} retorna true, en la  linea~\ref{line:lookingForWinner}, 
sabemos que si $e'$ cambia su estado es porque $e' \in \WESS$. Para ver si este cambio se produce, se realiza una computación de punto fijo basada en la solución del problema monolítico~(\ref{chpt:algoMono}). 
Sin embargo, el método \texttt{findNewGoalsIn} aplica una optimización basada en Lemma~\ref{lem:newWinnersLosersAreNonePredecessors}; solo considera estados que están en un $\NONE$-loop a través de la nueva transición ($\SCC$).

Si \texttt{canBeWinningLoop()} retorna false, entonces debemos comprobar si $e' \in \LESS$.
%  By Lemma~\ref{lem:newWinnersLosersAreNonePredecessors}
  %, we only consider states that are in a $\NONE$-loop 
%via the $e \step{\l}{} e'$ (i.e, the set $\SCC$). 
Esto puede hacerse de forma más eficiente que con un punto fijo usando el Lemma~\ref{lem:findErrorsWorks} que muestra que alcanza con observar si no es posible escapar de $\SCC$ alcanzando en un paso un estado que no esté en $\LES$. 



%Hack de titulo de lemma porque si no, no corta linea.
\begin{lemma}\textbf{\emph{(findNewErrorsIn es correcto y completo)}}
\label{lem:findErrorsWorks}
Si $\SCC =$ \emph{\texttt{getMaxLoop($e$, $e'$)}} $\wedge$\\ 
$\neg$\emph{\texttt{canBeWinningLoop}}($\SCC$) y \\
$P=\emph{\texttt{findNewErrorsIn}}(\SCC)$ entonces \\
$(e' \in \LESS \Rightarrow e' \in P 
\subseteq \LESS)$ $\wedge \, (e' \notin \LESS \Rightarrow P = 
\emptyset)$
\end{lemma}



Por motivos de eficiencia, \texttt{findNewGoalsIn} y
\texttt{findNewErrorsIn} no solo verifican si $e' \in \WESS$/$e' \in 
\LESS$ sino que también agregan estados ganadores/perdedores cuando pueden. La detección completa de nuevos estados ganadores y perdedores se hace finalmente con los procesos de propagación. 

Habiendo argumentado que la propiedad~~\ref{def:invariant} es válida, la correctitud y completitud le siguen de forma natural. 

Primero, notar que el algoritmo termina cuando logra determinar que $\init{e}$ está en $\LESS$ o $\WESS$. En el segundo caso, es simple construir un supervisor basándose en la estructura de exploración
$\structure$.\hfill$\qed$

\begin{theorem}[Correctitud y Completitud]
Sea $\E = (E,A_E^C)$ un problema de control composicional según Definition~\ref{def:control-problem}. Existe una solución para $\E$ si y solo si el algoritmo DCS retorna un supervisor para $\E$.
\end{theorem}

Demostración (Correctitud y completitud):
El teorema se desprende del invariante de ciclo del algoritmo (Definition~\ref{def:invariant}), el
Lemma~\ref{lem:WESandLesMonotonicity}, y el hecho de que en el peor caso todas las transiciones son agregadas a la estructura de exploración. Entonces,  $E = \structure = 
\unexploredToBottom{\structure} = \unexploredToTop{\structure}$.


\section{Demostración de Lemas}
\paragraph*{Proof Lemma \ref{lem:WESandLesMonotonicity}}


\textbf{Proof Sketch}
To prove $\WES \subseteq \WESS$ we show that a supervisor for a 
state $s$ in $\WES$ 
can be used as a supervisor from $s$ in $\WESS$. For $\LES \subseteq 
\LESS$, we assume 
there is a state $s \in \LES \setminus \LESS $. We reach a 
contradiction by showing that the 
supervisor that $s$ must have in 
$\unexploredToTop{\structure'}$ is also a 
supervisor for $s$ in $\unexploredToTop{\structure}$.
\textbf{End Proof Sketch}


If $s \in \WES $ then there exists a supervisor $\sigma$ for the control problem 
$\unexploredToBottom{\structure}$. Let $Z$ such that $\structure \subseteq Z$. we will 
show that $\sigma$ is a supervisor for $\unexploredToBottom{Z}$.  This requires showing two 
conditions as per Definition~\ref{def:control-problem}. The first, that $\sigma$ is 
controllable, is trivial as the set of controllable and uncontrollable events have not changed. 

For the second, nonblocking, we first prove that $\L^\sigma(\unexploredToBottom{Z}) = 
\L^\sigma(\unexploredToBottom{\structure})$. 

Assume that $\L^\sigma(\unexploredToBottom{Z}) \not\subseteq \L^\sigma(\unexploredToBottom{\structure})$.
If $w \in 
\L^\sigma(\unexploredToBottom{Z}) \setminus \L^\sigma(\unexploredToBottom{\structure})$. 
The run that witnesses $w$ must either always be in $Z$ or eventually reach a deadlock state 
in $\unexploredToBottom{Z}$. In either case, let $w_0$ be the longest prefix in $\structure$. 
We know $w_0$ is a proper prefix of $w$. Let $\ell$ such that $w_0.\ell$ is a prefix of $w$. 
By definition of $\unexploredToBottom{\structure}$, $w_0.\ell$ reaches a deadlock state in  
$\unexploredToBottom{\structure}$. This is a contradiction as $\sigma$ is a supervisor for 
$\unexploredToBottom{\structure}$. 

To show that $\L^\sigma(\unexploredToBottom{Z}) \supseteq \L^\sigma(\unexploredToBottom{\structure})$, assume $w \in \L^\sigma(\unexploredToBottom{\structure})$. If $w$ is also in $\L^\sigma(\structure)$ it is also in $\L^\sigma(Z)$ and $\L^\sigma(\unexploredToBottom{Z})$. Otherwise, $w = w_0.\ell$ hits a 
deadlock state in $\unexploredToBottom{\structure}$. As $w_0$ in $\L^\sigma(\structure)$, 
it is also in $\L^\sigma(Z)$. Consider the state $s$ reached by $w_0$ in $E$, it must have a 
transition labelled $\ell$ to justify its addition to $\unexploredToBottom{\structure}$. In, $Z$ 
state $s$ either has the transition and thus  $w_0.\ell \in \L^\sigma(Z) \subseteq 
\L^\sigma(\unexploredToBottom{Z})$, or it does not have the transition but then the state in 
$\unexploredToBottom{Z})$ has an $\ell$ transition to a deadlock state, hence $w_0.\ell \in 
L^\sigma(\unexploredToBottom{Z})$.

Now, assume a word $w \in 
\L^\sigma(\unexploredToBottom{Z})$ that cannot be extended with $w'$ such that $w.w'$ is 
in $L^\sigma(\unexploredToBottom{Z})$ and reaches a marked state of $\unexploredToBottom{Z}$. 
As  $w$ is also in $L^\sigma(\unexploredToBottom{\structure})$ then, as
$\sigma$ is a supervisor for $\unexploredToBottom{\structure}$, there is a $w'$ such that 
$w.w' \in L^\sigma(\unexploredToBottom{\structure}) = L^\sigma(\unexploredToBottom{Z})$ 
and hits a marked state. Note that the run for $w.w'$ is always in $\structure$, which means 
that the run is also in $\unexploredToBottom{Z}$. Thus we reach a contradiction.\\

To see that $\LES \subseteq \LESS$, assume there is a state $s \in \LES \setminus \LESS$. Since $s \notin \LESS$, it has a supervisor $\sigma$ in $\unexploredToTop{\structure'}$, but $s \in \LES$ so there can't exists a valid supervisor $\sigma'$ for $s$ in $\unexploredToTop{\structure}$. This is false, even more, we show that if $\sigma$ is a supervisor for $s$ in $\unexploredToTop{\structure'}$, then there is a supervisor $\sigma'$ for $s$ in any $\unexploredToTop{Z}$ such that $Z \subseteq \structure$.

$\sigma$ is a valid supervisor in $\unexploredToTop{\structure'}$, so every word in $\L^\sigma(\unexploredToTop{\structure'})$ can be extended to reach a marked state, and there are only finite states in $\unexploredToTop{\structure'}$, there must be a $w'$ and $w''$ such that $w.w'.w'' \in \L^\sigma(\unexploredToTop{\structure'})$, $w.w'$ reaches a marked state, and $w.w'.w''$ reaches the same state as $w.w'$.

If $w.w'.w''$ is in $\L^\sigma(Z)$, then there is nothing to be done, it is clear that $\sigma$ still has a way to extend $w$ in $\unexploredToTop{Z}$. Otherwise, we note that $w.w'.w'' = w_0.l.w_1$ such that $w_0$ is the longest prefix of $w.w'.w''$ in $\L^\sigma(Z)$, this means that $w_0.l$ reaches the marked winning state $\top$, and from there every extension of the word can only remain in this state, thus, $\sigma$ is also a valid controller for $\unexploredToTop{Z}$.
 
\begin{flushright}
$\square$
\end{flushright}


\paragraph*{Proof Lemma 
\ref{lem:newWinnersLosersAreNonePredecessors}}
\textbf{{Begin Proof Sketch}}

We first show there is a path from $s$ to $e'$ by assuming there is none. If $s \notin 
\WES \cup \LES$ then it has a supervisor $\sigma$ in 
$\unexploredToTop{\structure}$ but it does not have one in 
$\unexploredToBottom{\structure}$. This depends entirely on the descendants of 
$s$, since those are the only states that supervisors can reach in $\structure$. If $s$ 
is not a predecessor of $e'$, as $e \step{l}{\structure'} e'$ is the only difference 
between $\structure$ and $\structure'$, then the descendants of $s$ are the same, 
thus its possible supervisors in $\unexploredToTop{\structure'}$ and 
$\unexploredToBottom{\structure'}$ are unchanged. Thus, $s \notin \WESS \cup 
\LESS$ which is a contradiction.

We then prove that there is at least one path from $s$ to $e'$ via $\NONE$ states by 
contradiction assuming that all paths to $e'$ in $\structure'$ cross a state $s' \in 
(\WES \cup \LES)$. A supervisor $\sigma$ from $s$ in 
$\unexploredToTop{\structure}$ will not reach states in $\LES$, 
thus for all $s'$ it crosses they will have a supervisors $\sigma_{s'}$ for 
$\unexploredToBottom{\structure}$. We use $\sigma$ and $\sigma_{s'}$  to 
build a supervisor for $s$ in $\unexploredToTop{\structure'}$ to show that $s 
\not\in \LESS$.
A supervisor for $s$ in $\unexploredToBottom{\structure'}$ cannot exist because 
otherwise we can use it to build a supervisor for $s$ in 
$\unexploredToBottom{\structure}$ using a similar reasoning as before. This means 
that $s \in 
\WES$ which contradicts the hypothesis. 






%Possible supervisors for a state $s$ depend only on its descendants, thus if the descendants of $s$ haven't changed between to partial explorations $s$ has no new options for supervisors and it's status can't change.
%
%What is not so clear, is that $s$ has no new possible supervisors if it has a path to reach $e'$ but only passing through states in $\WES \cup \LES$. Let $s'$ be the first state of the path between $s$ and $e'$ that is not $\NONE$.
%
%We can see that if $s' \in \WES$ then once we reach it, we can simply follow $\sigma_{s'}$, its supervisor on $\unexploredToBottom{\structure}$. Note that $\sigma_{s'}$ does not use $e \step{l}{\structure'} e'$. 
%
%If $s' \in \LES$, no valid supervisor should reach $s'$ and thus won't reach $e \step{l}{\structure'} e'$. 
%
%Then, whether $s$ has a supervisor in $\unexploredToTop{\structure'}$ or it doesn't have one in $\unexploredToBottom{\structure'}$ does not depend on $e \step{l}{\structure'} e'$. Finally $s$ is not in $\WESS \cup \LESS$.

\textbf{{End Proof Sketch}}

If a state $s$ is not in $\WES \cup \LES$ it is because it has a  supervisor $\sigma$ in $\unexploredToTop{\structure}$ but it doesn't have one in $\unexploredToBottom{\structure}$. This depends entirely on the descendants of $s$, since those are the only states that $\sigma$ can reach. If $s$ is not a predecessor of $e'$, and $e \step{l}{\structure'} e'$ is the only difference between $\structure$ and $\structure'$ then the descendants of $s$ are the same, thus its possible supervisors haven't changed, and $s$ is still NONE.

What is not so clear, is that $s$ has no new possible supervisors if it has a path to reach $e'$ but only by passing through at least one state in $\WES \cup \LES$. Assuming it must pass through states in $\WES \cup \LES$ we show that:

\begin{itemize}
	\item knowing $s$ had a supervisor $\sigma$ in $\unexploredToTop{\structure}$, we show $s$ has a valid supervisor $\sigma'$ in $\unexploredToTop{\structure'}$:
	
	$\sigma'(w) = \sigma(w)$ if there is no $w_0$ suffix of $w$ such that $ s \runw{w_0}{\structure'} s_i \wedge s_i \in \LES \cup \WES$. 
	
	$\sigma'(w) = \sigma_{s_i}(w_1)$ where $w_0$ is the shortest suffix of $w = w_0.w_1$ such that $ s \runw{w_0}{\structure'} s_i \wedge s_i \in \WES$. $\sigma_{s_i}$ is the supervisor we know $s_i$ had in $\unexploredToBottom{\structure}$ because $s_i \in \WES$, and every supervisor valid in $\unexploredToBottom{\structure}$ is also valid in $\unexploredToTop{\structure}$.
	
	Since $\sigma$ is a valid supervisor, we know it will not reach states in $\LES$.
	
	Finally it is clear that $\sigma'$ is a valid supervisor for $s$ in $\unexploredToTop{\structure'}$. Note that $\sigma'$ does not depend of the new transition.
	
	
	
	
	\item Knowing $s$ had no supervisor in $\unexploredToBottom{\structure}$, we show that $s$ has no supervisor in $\unexploredToBottom{\structure'}$ by assuming there is one and reaching a contradiction:
	
	Suppose there is a supervisor $\sigma'$ for $s$ in $\unexploredToBottom{\structure'}$, and that $e \step{l}{\structure'} e'$ is the only difference between $\structure$ and $\structure'$.
	
	With $\sigma'$ we will build $\sigma$, a supervisor for $s$ in $\unexploredToBottom{\structure}$.
	
	$\sigma(w) = \sigma'(w)$ if there is no $w_0$ which is a prefix of $w$ and $s \runw{w_0}{\structure} s_i \wedge s_i \in \WES \cup \LES$. Since $\sigma'$ is a valid supervisor, we know it will not reach states in $\LESS$. Note that $w$ can't reach $e \step{\l}{\structure'} e'$ because $s$ has no path of $\NONE$ states to $e'$.
	
	If $w=w_0.w_1$ and $s \runw{w_0}{\structure} s' \wedge s' \in \WES$ then $\sigma(w_0.w_1) = \sigma_{s'}(w_1)$ where $\sigma_{s'}$ is the supervisor for $s'$ in $\unexploredToBottom{\structure}$. Note that once we reach $s'$ we always follow $\sigma_{s'}$.
	
	Since $\sigma$ never reaches the new transition we know that $\sigma$ is valid in $\unexploredToBottom{\structure}$.
		
	We see then that assuming there is a valid supervisor $\sigma'$ for $s$ in $\unexploredToBottom{\structure'}$ implies that there is a valid supervisor $\sigma$ for $s$ in $\unexploredToBottom{\structure}$. ABS! \\
	
\end{itemize}


\paragraph*{Proof Lemma \ref{lem:E'isNoneThenAllIsNone}}
\textbf{Begin Sketch}
Assuming $e' \notin \WESS$, we use a witness $s$ to $\WESS \neq 
\WES$ to reach a contradiction. State $s$ must have a supervisor in 
$\WESS$  that avoids $e \step{\l}{} e'$ which is the only difference 
between $\unexploredToBottom{\structure}$ and 
$\unexploredToBottom{\structure'}$. This supervisor is then also a 
supervisor for $s$ in $\WES$ reaching a contradiction. 

If we assume $e' \notin \LESS$.  We use a witness $s$ to $\LESS 
\neq 
\LES$ to reach a contradiction. Note that as $e' \notin \LESS$, there 
is a supervisor $\sigma$ from $e'$ in  
$\unexploredToTop{\structure'}$.  As $s \notin \LES$ it also must 
have a 
supervisor $\sigma'$ in  $\unexploredToTop{\structure}$. We build 
a 
new supervisor for $\unexploredToTop{\structure'}$ from $s$ that 
works just like $\sigma'$ but when it reaches $e \step{\l}{} e'$ it 
behaves as $\sigma$. This new supervisor proves that $s \in \LESS$ 
which is a contradiction. 
\textbf{End Sketch}

We prove both implications by contradiction. 

First we assume $e' \not\in 
\WESS \setminus \WES$. Note that as $e' \notin \WES$ then $e' 
\notin 
\WESS$.  As $\WESS \neq \WES$ by monotonicity 
(Lemma\ref{lem:WESandLesMonotonicity}) there must be a state $s$ 
such that $s \in \WESS \setminus 
\WES$, so $s$ must have a supervisor in $\WESS $. This supervisor 
cannot reach $e'$ because if it did, 
then there would be a supervisor for $e'$ and we had assumed $e' 
\notin \WESS$. Furthermore, if the supervisor reaches $e$, then 
$\l$ 
must be controllable (it if were uncontrollable, the supervisor would 
reach $e'$ which we established is not possible). Thus, the 
supervisor 
avoids $e \step{\l}{} e'$ altogether which means that it is also a 
supervisor for $s$ in $\unexploredToBottom{\structure}$ (i.e.,  $s 
\in 
\WES$) reaching a contradiction.


%
%There must be a 
%supervisor from $s$ that avoids 
%transition $e \step{\l}{} e'$ (which is the only difference between 
%$\structure$ and $\structure'$). We know this because there is no 
%supervisor for $e'$ in $\unexploredToBottom{\structure'}$, 
%otherwise $e'$ would be in $\WESS$. 
%Thus, if $e$ is reachable using that supervisor, the state is 
%controllable and the transition can be avoided. But this means that 
%there is a supervisor for $s$ in 
%$\unexploredToBottom{\structure}$ 
%and finally $s \in \WES$, a contradiction.

Now we assume $e' \not\in 
\LESS \setminus \LES$. Note that as $e' \notin \LES$ then $e' 
\notin 
\LESS$.  
 Let $s \in \LESS$ and $s \not\in \LES$. We know that from $s$ 
 there is a supervisor $\sigma'$ for 
 $\unexploredToTop{\structure}$. This supervisor either is also a 
 supervisor for $\structure$ or reaches the $\top$ state in 
 $\unexploredToTop{\structure}$. In the first case, it is also a 
 supervisor $\structure'$ and in $\unexploredToTop{\structure'}$, 
 reaching a contradiction. In the second case, it either uses a 
 transition that is neither in $\structure'$ nor $\structure$, which 
 means that in $\unexploredToTop{\structure'}$ it will lead to a 
 winning $\top$ state; or it uses the transition that is in 
 $\structure'$ but not $\structure$ which leads to $e'$. Since we 
 know $e'$ is not in $\LESS$, then it has a supervisor for it in 
 $\unexploredToTop{\structure'}$, then we know that there exists a 
 supervisor $\sigma''$ that includes both $\sigma'$ and the 
 supervisor for $e'$. Finally, $\sigma''$ is a supervisor for $s$ in 
 $\unexploredToTop{\structure'}$, but $s \notin \LESS$, ABS!	



\paragraph*{Proof Lemma \ref{lem:canBeWinningLoopWorks}}
\textbf{{Begin Proof Sketch}}
To prove $e' \in \WESS \setminus \WES$ implies  \texttt{canBeWinningLoop($\SCC$)}, we  
assume $\neg$\texttt{canBeWinningLoop($\SCC$)} and show that $e' \notin \WESS 
\setminus \WES$. For this, it suffices to see that if 
$\neg$\texttt{canBeWinningLoop($\SCC$)} then to reach a marked state from $e'$ it must 
exit $\SCC$ to a state $s \notin \SCC$. This state must be such that $s \notin \WESS$. As 
$s$ has no 
supervisor in $\unexploredToBottom{\structure'}$, it is impossible for $e'$ to have one. 


To prove that \texttt{canBeWinningLoop($\SCC$)} implies $e' \notin \LESS$ we build a supervisor $\sigma'$ for $e'$ in $\unexploredToTop{\structure'}$ as 
follows: 
For a trace that stays within $\SCC$, we only choose controllable successors 
that are not 
in $\LES$. Note that there cannot be uncontrollable successors in $\LES$ because 
$\SCC \cap \LES= \emptyset$.   As soon as a trace exits $\SCC$ at a state $s'$ we 
use the supervisor for $s'$ in $\unexploredToTop{\structure'}$. 
As $s'$ cannot reach $e \step{\l}{\structure'} 
e'$ using $\NONE$ states, by 
Lemma~\ref{lem:newWinnersLosersAreNonePredecessors}, $s'$ must have such a 
supervisor. 

\textbf{{End Proof Sketch}}

Supose $e' \in \WESS \setminus \WES$ but $\neg$\texttt{canBeWinningLoop($\SCC$)}.
There exists a supervisor $\sigma$ for $e'$ in $\unexploredToBottom{\structure'}$, this means there must be a path $w$ from $e'$ to a marked state $m$. 
Since $\neg$\texttt{canBeWinningLoop($\SCC$)}, there are no marked state in $\SCC$, $w$ must leave $\SCC$. 
Let $s$ be the first state reached by $w$ not in $\SCC$, $s$ is in $\LES \cup \NONE$, and doesn't have a $\NONE$ path to $e'$ then by Lemma~\ref{lem:newWinnersLosersAreNonePredecessors} $s$ will still not change it status.
Since $s \notin \WESS$, $s$ doesn't have a supervisor in $\unexploredToBottom{\structure'}$, but $\sigma$ accepts runs that lead to $s$, absurd.



Assuming \texttt{canBeWinningLoop($\SCC$)} we simply build a supervisor $\sigma^4$ for $e'$ in $\unexploredToTop{\structure'}$ to prove that $e' \notin \LESS$


We define $\sigma^4$ so that it enables all uncontrollable transitions (to be $controllable$).

$\sigma^4(w_0.w_1) = \sigma_{s'}(w_1)$ if $w_0$ is the shortest word such that there is an $s' \notin \SCC \wedge s'\notin \LES$ and $e' \runw{w_0}{\unexploredToTop{\structure'}} s'$.
Else if $e' \runw{w}{\unexploredToTop{\structure'}} p \wedge p \in \SCC$ then for all controllable $\l'$, $\l' \in\sigma^4(w)$ iff $\exists  p' \ldot p \step{l'}{} p' \wedge p' \notin \LES$.
%todas las transiciones que van al loop mas todas las transiciones que salen a un estado NONE (o TOP), i.e. no error. 

We use the supervisors $\sigma_{s'}$ where $s'$ is such that exists $s \in \SCC$ and $s\step{\l'}{\unexploredToTop{\structure'}} s'\wedge s' \notin \SCC \wedge s'\notin \LES$. If none such $s'$ exists, we know there is a marked state in $\SCC$ and $\sigma^4$ never leaves $\SCC$ since all reachable states from $\SCC$ are in $\LES$.



We will prove that $\sigma^4$ is $controllable$ and $non-blocking$.

Since we enabled all uncontrollable transitions, $\sigma^4$ is trivially $controllable$.

For $non-blocking$, assume $w$ compatible with $\sigma^4$, we will show that it can be extended. If $w = w_0.w_1$ where $w_0$ is the shortest word such that there is an $s' \notin \SCC \wedge s' \notin \LES$ and $e' \runw{w_0}{\unexploredToTop{\structure'}} s'$. Then by definition of $\sigma^4$ we have that $\sigma^4(w_0.w_1.w_2) = \sigma_{s'}(w_1.w_2)$ for all $w_2$. As $\sigma_{s'}$ is $non-blocking$, there is a $w_2$ such that $\sigma_{s'}(w_1.w_2)$ hits a marked state. So $\sigma^4(w_0.w_1)$ can be extended to hit the marked state. 

Otherwise, $w$ never exits $\SCC$. We want to prove that for every $\l'$ such that $w.\l'$ is consistent with $\sigma^4$, $w.\l'$ is extendable with a $w'$ to reach a marked state. Let $p'$ be such that $e' \runw{w.\l'}{} p'$. If $p' \notin \SCC$ then $\sigma^4(w.\l') = \sigma_{p'}(\lambda)$ and, as before, we know $\sigma_{p'}$ is $non-blocking$ so $w.l'$ is extendable to reach a marked state.\\
If $p' \in \SCC$, then  $w.l'$ can be extended to reach any state $s$ in the $\SCC$. We know that either there is a marked state in $\SCC$ or a state in $\WES$ reachable in one step from $\SCC$, either way we can extend $w.l'$ to reach a marked state.\\



\paragraph*{Proof Lemma \ref{lem:findErrorsWorks}}
\textbf{{Begin Proof Sketch}}
We split the proof based on the if/then/else structure of \texttt{findSomeNewErrors}. 
In the case of \texttt{if} being true, it suffices to prove that $e' \notin \LESS$. For this, 
we build a supervisor $\sigma'$ for $e'$ in $\unexploredToTop{\structure'}$ as 
follows: For a trace that stays within $\SCC$, we only choose controllable successors 
that are not 
in $\LES$. Note that there cannot be uncontrollable successors in $\LES$ because 
$\SCC \cap \LES= \emptyset$.   As soon as a trace exits $\SCC$ at a state $s'$ we 
use the supervisor for $s'$ in $\unexploredToTop{\structure'}$. 
As $s'$ cannot reach $e \step{\l}{\structure'} 
e'$ using $\NONE$ states, by 
Lemma~\ref{lem:newWinnersLosersAreNonePredecessors}, $s'$ must have such a 
supervisor. 
 
When the \texttt{if} is false, it suffices to prove $P = \SCC \subseteq \LESS$. We do 
this by assuming $s \in \SCC \setminus \LESS$ and reaching a contradiction. If $s 
\notin LESS$ it has a supervisor $\sigma$ which admits a trace to a marked state. As 
there are no marked states in $\SCC$, the trace reaches a state $s' \not \SCC$ As the 
\texttt{if} was false, $s' \in \LESS$ which means $\sigma$ is not a supervisor.
\textbf{{End Proof Sketch}}


First, we know that every state $s' \notin \SCC$ such that $\exists s \in \SCC \ldot s 
\step{l}{} s'$, either is and will be a losing state ($s' \in LES \wedge s' \in LESS$) or is 
and will be $\NONE$ (because $s$ isn't a $NONE-$Predecessor of any state in 
$\SCC$, otherwise $s$ would be in $\SCC$). \\
This means that any state $s' \notin \SCC \wedge s' \notin \LES$ can not be forced to a state in $\LESS$. Thus, if we reach a $\NONE$ state we know it has a supervisor $\sigma_{s'}$ in $\unexploredToTop{\structure'}$.

In the case that the $\texttt{if}$ statement is true, we prove that $e' \notin \LESS$:

We use the same $\sigma^4$ as in Proof Lemma \ref{lem:canBeWinningLoopWorks}. We already know $\sigma^4$ is both $controllable$ and $non-blocking$ in this situation because of the previos Lemma.\\

Otherwise we enter the $\texttt{else}$ block:

If $\nexists s \in \SCC \ldot \trimlst{s \step{\l'}{\unexploredToTop{\structure'}}  s'} \wedge (s' \notin \SCC \wedge s' \notin \Errors)$ we prove that $\forall s \in \SCC$, $s \in \LESS$

Suppose there is a supervisor $\sigma'$ for $s$ in $\unexploredToTop{\structure'}$ then $\exists w'$ such that $s \runw{\lambda.w'}{\unexploredToTop{\structure'}} e_m \wedge e_m \in M_{\unexploredToTop{\structure'}}$. Since there's no marked state in $\SCC$ then eventually $s$ following $w'$ leaves $\SCC$. \\
Let $w' = w_0.w_1$ such that $w_0$ is the shortest word so that $s \runw{w_0}{\unexploredToTop{\structure'}} s' \wedge s' \notin \SCC$. But since $s'\in \Errors \Rightarrow s' \in \LES$ then it is not possible for a valid supervisor $\sigma'$ to allow reaching that state. ABS! Then there's no supervisor for $s$ in $\unexploredToTop{\structure'}$ implying $\forall s \in \SCC$, $s \in \LESS$.
