En esta sección presentamos el nuevo algoritmo \DCS, que realiza una exploración sobre la marcha del espacio de estados. Por medio de dicha exploración el algoritmo encuentra una solución al problema composicional de "supervisory control". También discutimos la correctitud y completitud del nuevo algoritmo \DCS. \\

\section{Problemas encontrados}
El anterior algoritmo de exploración tenía falencias en cuanto a agregar estados al conjunto $\Errors$. Esto se debía a que no sacaba conclusión alguna al haber explorado todo un subgrafo, por ende al propagar información desde otra rama se podría llegar a un resultado erroneo. Para comprender mejor observar la figura (INGRESE FIGURA) donde desde el estado e tenemos dos sub-ramas a explorar. Si se mira primero la de abajo y no lo marcamos como error entonces al mirar la de arriba diremos que es goal y propagaremos dicha información, equivocadamente, más allá de e.

EXPLICAR QUE NUESTRO ALGORITMO ES AGNÓSTICO A LA HEURÍSICA (COMO DEBERÍA)

\section{Propuesta de nuevo algoritmo}

\lstset{escapeinside={(*@}{@*)}}
\lstset{numbers=left, numberstyle=\tiny, stepnumber=1, numbersep=5pt}
\begin{lstlisting}[language={pseudocode},label={lst:dcs},caption={Algoritmo de exploración dirigida on-the-fly.},float=ht, frame=single]
 function DCS($\E {=} (E, A_E^C)$,$\:\heuristic$):
   $\initial$ = $\<\init{e}^0,\ldots,\init{e}^n\>$
   $\structure = \trimlst{(\{ \initial \}, A_E, \emptyset, \initial, M_E \cap \{\initial\})}$
   $\Goals = \Errors = \Witness = \emptyset$
   $\NONE = \{\initial\}$
   if (isDeadlock($\initial$)):(*@\label{line:initialDead}@*)
   $\Errors = \{\initial\}$
   $\NONE = \emptyset$
   while $\initial \not\in \Errors \cup \Goals$:
     $(e,\l,e')$ = expandNext($\heuristic$) (*@\label{line:expand}@*)
     $S_{\structure'} = S_\structure\cup  \{ e' \}$
     $\structure' = \trimlst{(S_{\structure'} , A_E, \D_\structure \cup \{ e \step{\l}{} e' 
     \}, \initial, M_E \cap S_{\structure'} )}$
     if $e' \in \Errors $: (*@\label{line:isLosing}@*)
       propagateError($\{e'\}$)
     else if $e' \in \Goals$: (*@\label{line:isWinning}@*)
       propagateGoal($\{e'\}$)
     else if canReach($e, e', \structure$): (*@\label{line:isLoop}@*)
       $\SCC$ = getMaxLoop($e$,$e'$) (*@\label{line:getMaxLoop}@*)
       if canBeWinningLoop($\SCC$): (*@\label{line:CanBeWinning}@*)
         C = findNewGoalsIn($\SCC$) (*@\label{line:lookingForWinner}@*)
         $\Witness = \Witness \cup (C \cap M_{\structure'})$
         $\Goals = \Goals \cup C$
         $\NONE = \NONE \setminus C$
         propagateGoal(C)   (*@\label{line:propagateGoal2}@*)
       else:    (*@\label{line:CanBeLosing}@*)
         P = findNewErrorsIn($\SCC$)
         $\Errors = \Errors \cup P$
         $\NONE = \NONE \setminus P$
         propagateError(P) (*@\label{line:propagateError2}@*)
     $\structure = \structure'$
 
   if $\initial \in \Goals$:
     $r = rankStates(\structure)$ (*@\label{line:controller-start}@*)
     $return \, \lambda w \ldot \{ \, \l \mid \trimlst{\initial \runw{w}{\structure} e 
     \step{\l}{\structure} e'} \wedge e' \in \Goals\,$
     $\wedge (l \in A^C_E \then \l = bestControllable(s,r,\structure) 
     )\}$(*@\label{line:controller-end}@*)
   else:
     return UNREALIZABLE
\end{lstlisting}  
%invariante del ciclo:  Inv($states(\structure), \structure)$

\lstset{numbers=none, numberstyle=\tiny, stepnumber=1, numbersep=5pt}
\begin{lstlisting}[language={pseudocode},label={lst:dcs.propagate},caption={Algoritmos de propagación.},float=ht, frame=single]
 function propagateGoal($newGoals$):
   $C' = \emptyset$; $C$ = ancestorsNone($newGoals$)
   while $C'\neq C$:
     $C' = C$
     $C=C \setminus \{ s \in C \mid$ 
       forcedTo($s, S_{\unexploredToBottom{\structure'}}\setminus (C \cup \Goals),\unexploredToBottom{\structure'}$) $\vee$
       cannotReachGoalIn($s, C$)$\}$
   $\Goals = \Goals \cup C$
   $\NONE = \NONE \setminus C$

 procedure propagateError($newErrors$):
   $P$ = ancestorsNone($newErrors$)
   $C$ = $P$; $C' = \emptyset$
   while $C' \neq C$:
     $C' = C$
     $C=C \setminus \{ s \in C \mid$ 
     	forcedTo($s,\Errors,\unexploredToTop{\structure'}$) $\vee$
     	cannotReachGoalIn($s, C$)$\}$
   $P = P \setminus C$
   $\Errors = \Errors\ \cup P$
   $\NONE = \NONE \setminus P$
\end{lstlisting}


\begin{lstlisting}[language={pseudocode},label={lst:dcs.gather},caption={Confirmación de clasificaciones},float=ht, frame=single]
 function findNewGoalsIn($\SCC$):
   $C = \SCC$; $C' = \emptyset$
   while $C' \neq  C:$
     $C' = C$; $C'' = \emptyset$
     while $C'' \neq C:$
       $C'' = C$
       $C=C \setminus \{ s \in C \mid$ 
               forcedTo($s, S_{\unexploredToBottom{\structure'}}\setminus (C \cup \Goals),\unexploredToBottom{\structure'}$) $\vee$
               cannotBeReached($s,C$) $\}$
     $C = C \setminus \{ s \in C \mid $cannotReachGoalOrMarkedIn($s, C$)$\}$
   return C
  
 function findNewErrorsIn($\SCC$):
   if ($\exists s \in \SCC \ldot \trimlst{s \step{\l'}{\unexploredToTop{\structure'}}  s'} 
   \wedge (s' \notin \SCC \wedge s' \notin \Errors)$):
     return $\emptyset$
   else: 
     return $\SCC$
\end{lstlisting}



%VErsion anterior
%   let $(e,\l,e')\in (\D_{E}\setminus\D_{\structure})$ such that $e\in S_\structure\ 
%   \wedge$
%     ($\forall (s,\l',s')\in (\D_{E}\setminus\D_{\structure})\, s\in S_\structure \ldot$ 
%        $\heuristic$($e, \l, e'$) $>=$ $\heuristic$($s, \l', s'$))

% AUX FUNCTIONS
\begin{lstlisting}[language={pseudocode},label={lst:dcs.aux},caption={Métodos auxiliares},float=ht, frame=single]
 procedure expandNext($\heuristic$):
   let $\trimlst{(e,\l,e') \ldot e\in S_\structure \wedge e\step{\l}{E} e' \wedge   
       \neg e\step{\l}{\structure} e' \wedge e \in \NONE \wedge}$ 
       $\trimlst{\forall (s,\l',s') \ldot  s\in S_\structure \wedge s\step{\l}{E} s' \wedge \neg s\step{\l}{\structure} s' \wedge s \in \NONE}$ 
         $\implies \trimlst{\heuristic(e, \l, e') >=\heuristic(s, \l',   s'))}$
   if isDeadlock($e'$):
     $\Errors$ = $\Errors \cup \{e'\}$
   if $e'\notin \Errors \cup \Goals$:
     $\NONE = \NONE \cup \{e'\}$
   return $(e,\l,e')$
 
 function ancestorsNone(targets):
   return $\{ \, e \in \structure' \mid \exists e' \in targets \ldot \exists w \ldot \trimlst{e \runw{w}{\structure'} e'} \wedge$
          $ \nexists s \in w(e) \ldot s \neq e' \wedge \in \Goals \cup \Errors \}$
   
 function canBeWinningLoop(loop):
   return $(\marked{loop}{\structure'}) \vee$ 
          $(\exists s \in loop \ldot $canReachInOneStep($s, \structure, \Goals$)$)$
   
 function getMaxLoop($e$,$e'$):
   return $\{s \mid \exists w, w' \ldot \trimlst{e \runw{w}{\structure'} s} \wedge \trimlst{s \runw{w'}{\structure'} e'} \wedge$ 
          $\nexists s' \ldot (s' \in w(e) \vee s' \in w'(s)) \wedge s' \neq e' \wedge s' \in \Goals \cup \Errors \}$

 function forcedTo($s,Dest,Z$):
   return $(\exists \l_u \in A_Z^U \ldot \exists e \in Dest \ldot \trimlst{s \step{\l_u}{Z} e})\ \vee$
          $(\forall \l \in A_Z \  (\trimlst{s \step{\l}{Z} e} \Rightarrow e \in Dest))$    
    
 function cannotBeReached($s, C$):
    return $\nexists s' \in C , \exists \l \ldot s' \step{\l}{\structure'} s$
   
 function cannotReachGoalOrMarkedIn($s, C$)
    return $\trimlst{\nexists w \ldot s \runw{w}{C} s' \wedge s'\in C\ \wedge}$ 
           $($canReachInOneStep$(s', \structure, \Goals)$
           $\vee\ s'\in\M_{\structure'})$
           
 function cannotReachGoalIn($s, C$)
    return $\trimlst{\nexists w \ldot s \runw{w}{C} s' \wedge s'\in C\ \wedge}$ 
           canReachInOneStep$(s', \structure, \Goals)$
           
 function canReach($s', s$)
    return $\exists w \ldot s' \runw{w}{\structure'} s$

 function canReachInOneStep($s, Targets$)
    return $\exists l \ldot s \step{\l}{\structure'} s' \wedge s' \in Targets$

 function isDeadlock($s$)
    return $\nexists l \ldot s \step{\l}{E} s'$
\end{lstlisting}

\begin{lstlisting}[language={pseudocode},label={lst:dcs.aux},caption={Métodos de ranking},float=ht, frame=single]
 function rankStates($\structure$)
    $r = 0$; $W' = \Witness$; $W = \emptyset$
    while $W' \neq  W:$
      $\forall w \in W', $rank$(w) = r$
      $W = W \cup W'$
      $W'= \{ s \in (\Goals \setminus W) \mid \exists s' \in W \ldot s \step{}{\structure} s'\}$
      $r = r + 1$
    return rank

 function bestControllable($e, r, \structure$)
    return $\l \in A^C_E \ldot e \step{\l}{\structure} e' \wedge \nexists \l' \in A^C_E \ldot $
			$e \step{\l'}{\structure} e'', r(e'') \leq r(e')$

\end{lstlisting}


%
% function isDeadlock($e$):
%   return $\neg\exists \l \! \ldot \trimlst{e \step{\l}{E} e'}$
%
% function canReach($e, e', Z$):
%   return $\exists w \ldot \trimlst{e' \runw{w}{Z} e}$
%   
% function canReachInOneStep($s, Z, \S$):
%   return $\trimlst{\exists l \ldot s \step{\l}{Z} s' \wedge s' \in \S}$

\section{Demostración de corectitud y completitud}
\begin{notation}
Decimos que un estado $s$ es ganador["winning"] (resp. perdedor["losing"]) en el problema $\E = 
(E, A_E^C)$ siendo $s$ el estado inicial de $E$ si hay una (resp. no hay una) solución para $\E$. Nos referimos a los estados ganadores y perdedores de $E$ cuando $A_E^C$ es inferible del contexto, también usamos $W_E$ y $L_E$ para denotar el conjunto de estados ganadores y perdedores de $\E$.
\end{notation}

El algoritmo (ver Listing~\ref{lst:dcs}) explora incrementalmente el espacio de estados de $E$ utilizando una estructura de exploración parcial ($ES$), añadiéndole una transición por vez.


\begin{definition}
[Exploración Parcial] \label{def:unexploredTo}
Sea $E = (S_E,$ $A_E, \D_E, \init{e}, M_E)$. Decimos que $ES$ es una exploración parcial de $E$ ($ES \subseteq E$) si $S_\structure \subseteq 
S_E$ y $\structure = (S_\structure,A_E, \D_\structure,\init{e},M_E 
\cap S_\structure)$, donde $ \D_\structure \subseteq (\D_E \cap 
(S_\structure \times A_E \times S_\structure))$. Escribimos $ES \subset E$ cuando $S_\structure \subset S_E$.
\end{definition}

Para explicar el algoritmo y argumentar su correctitud y completitud introducimos dos nuevos problemas de control para exploraciones parciales. Uno toma una visión optimista de la región no explorada ($\top$) asumiendo que todas las transiciones no exploradas llevan a un estado ganador. El otro toma una visión pesimista ($\bot$) asumiendo que las transiciones no exploradas llevan a estados perdedores.

\begin{definition}
[Problemas de Control $\top$ y $\bot$] \label{def:unexploredTo}

Sean $\E = (E, A_E^C)$, $E = (S_E,A_E,\D_E,\init{e},M_E)$ y $\structure = 
(S_\structure,A_E, \D_\structure,\init{e},M_E \cap S_\structure)$, y $\structure 
\subseteq E$.
\\
Definimos $\E_\top$ como $(\unexploredToTop{\structure}, A_E^C)$ donde 
$\unexploredToTop{\structure} = (S_\structure \cup \, \{\top\},A_E,\D_\top, 
\init{e}, 
(M_E \cap S_\structure)\, \cup \, \{\top\})$ y $\, \D_\top \, = \, \D_\structure 
\, 
\cup\, \{(s,\l, \top) 
\;$ $ | \; \exists s' \ldot (s, \l, s') \in (\D_E \setminus \D_\structure)\} \cup \{(\top, \l, \top) \, | \, \l \in A_E\}$ \\
Definimos $\E_\bot$ como $(\unexploredToBottom{\structure}, A_E^C)$ donde 
$\unexploredToBottom{\structure} = (S_\structure \, \cup \, 
\{\bot\},A_E,\D_\bot, 
\init{e}, M_E \cap S_\structure)$ y $\D_\bot = \D_\structure \, \cup \, \{(s,\l, 
\bot) \, | \, $ $ \exists s' \ldot (s, \l, s') \in (\D_E \setminus \D_\structure)\}$ 
\end{definition}

Usamos estos problemas de control para decidir tempranamente si un estado $s$ es ganador o perdedor en $E$ basado en lo que exploramos previamente en $\structure$. Si $s$ es ganador en $\unexploredToBottom{\structure}$ esto significa que sin importar a dónde lleven las transiciones no exploradas, $s$ también va a ser ganador en $E$. Similarmente, $s$ es perdedor en $E$ si es perdedor en $\unexploredToTop{\structure}$.
Lemma~\ref{lem:WESandLesMonotonicity} refuerza este razonamiento.


\begin{lemma}\textbf{\emph{(Monotonicidad de $\WES$ y $\LES$)}}
\label{lem:WESandLesMonotonicity}
Sean $\structure$ y $\structure'$ dos exploraciones parciales de $E$ tal que $\structure 
\subset \structure'$ entonces $\WES \subseteq \WESS$ y $\LES \subseteq 
\LESS$.
\end{lemma}

El algoritmo agrega iterativamente una transición de $E$ a $\structure$ a la vez y asegura que al final de cada iteración, los estados en $\structure$ están correcta y completamente clasificados en ganadores y perdedores si hay suficiente información de $E$ en $\structure$. Los conjuntos de estados $Errors$, 
$Goals$ y $\NONE$ se usan para este propósito.

\begin{property}[Invariante]
\label{def:invariant}
El loop principal del Algorithm~\ref{lst:dcs} tiene el siguiente invariante: 
$\structure \subseteq E \ \wedge $ $\forall s \in \structure \ldot (s\in\Goals 
\Leftrightarrow 
s \in 
\WES)$ $\wedge$  $(s\in\Errors 
\Leftrightarrow s \in \LES)$ $\wedge$  
$s\in\Errors\uplus\Goals\uplus\NONE$

% \begin{tabbing}
% ajfdlkadl\= \kill
% \> \= 
% \>$s\in\Goals \Leftrightarrow s \in \WES \wedge$ \\
% \> $s\in\Errors \Leftrightarrow s \in \LES \wedge$ \\
% \> $s\in\Errors\uplus\Goals\uplus\NONE$
% \end{tabbing}
\end{property}

La explicación del Algorithm~\ref{lst:dcs} que detallamos a continuación sirve también como un esquema de demostración para Property~\ref{def:invariant}.   

Para empezar, notar que la función \texttt{expandNext} 
(line~\ref{line:expand}) retorna una nueva transición $e \step{\l}{E} e'$ 
garantizando que $e$ ya se encontraba en $\structure$ y $e \in \NONE$. 
Esto significa que en cada iteración, hay algo de información nueva disponible para un estado que actualmente no está clasificado en ganador ni perdedor.

Si el estado $e'$ ya es clasificado como ganador en
$\unexploredToBottom{\structure}$ (line~\ref{line:isWinning}) o
perdedor en $\unexploredToTop{\structure}$ (line~\ref{line:isLosing}) 
entonces esta información necesita ser propagada a los estados en $\NONE$ para ver si pueden convertirse en ganadores en 
$\unexploredToBottom{\structure'}$ o perdedores en
$\unexploredToTop{\structure'}$. Tanto \texttt{propagateGoal} como
\texttt{propagateError} realizan un punto fijo estándar~\cite{Ramadge:1989:CDES} sobre
$\unexploredToBottom{\structure}$ y
$\unexploredToTop{\structure}$ pero solo sobre predecesores de $e'$ que están en $\NONE$. 
Lemma~\ref{lem:newWinnersLosersAreNonePredecessors} asegura la completitud de esta propagación restringida.

%Hack de titulo de lemma porque si no, no corta linea.
\begin{lemma}\textbf{\emph{(Ganadores/Perdedores nuevos tienen camino de estados-\textit{None} a transición nueva)}}
\label{lem:newWinnersLosersAreNonePredecessors}
Sea la transición $e \step{\l}{\structure} e'$ la única diferencia entre dos exploraciones parciales, $\structure$ y $\structure'$, de $E$. Si $s \notin (\WES \cup \LES)$ y $s \in (\WESS \cup \LESS)$, entonces hay $s_0, \ldots, s_n \notin (\WES \cup \LES)$ tal que $s = s_0 \wedge
s_0 \step{\l_0}{\structure}\ldots s_{n} \step{\l}{\structure} e'$.
\end{lemma}

Ya en la linea~\ref{line:isLoop} sabemos que $e'$ no es ganador en $\unexploredToBottom{\structure}$ ni perdedor en $\unexploredToTop{\structure}$, chequeamos si $e 
\step{\l}{\structure} e'$ cierra un nuevo loop. Si no es el caso, entonces no hay nada que hacer ya que $e'$ alcanza las mismas transiciones en $\structure'$ que en $\structure$. Entonces, $e' \notin 
(\WESS \cup \LESS)$ ya que cualquier supervisor para $e'$ en $\unexploredToBottom{\structure'}$ (resp. $\unexploredToTop{\structure'}$) es también un supervisor en $\unexploredToBottom{\structure}$ (resp. 
$\unexploredToTop{\structure}$) y vice versa.
%Thus, there is no new information about whether it is winning or 
%losing (Lemma~\ref{lem:noNewLoopImpliesNoNewInformation}). 
%
%%Hack de titulo de lemma porque si no, no corta linea.
%\begin{lemma}\textbf{\emph{(No new reachable transitions from e' implies no new 
%information)}}
%\label{lem:noNewLoopImpliesNoNewInformation}
%Let $e \step{\l}{} e'$ be the only difference between two partial explorations, 
%$\structure$ and $\structure'$ and $e' \notin \WES \cup \LES$. 
%If $\, \forall (s \step{\l'}{} s') \ldot e' \runw{w}{\structure} s  \step{\l'}{\structure} s' $ 
%if and only if $e' \runw{w}{\structure'} s  \step{\l'}{\structure'} s' $ then 
%$e' \notin \WESS \cup \LESS$
%\end{lemma}
%
Más aún, que no haya nueva información para $e'$ implica que no hay nuevos ganadores o perdedores (Lemma~\ref{lem:E'isNoneThenAllIsNone})

\begin{lemma}\textbf{\emph{(Nuevos ganadores/perdedores solo si $e'$ es un nuevo ganador/perdedor)}}
\label{lem:E'isNoneThenAllIsNone}
Sea $e \step{\l}{} e'$ la única diferencia entre dos exploraciones parciales, $\structure$ y $\structure'$. Si $\WESS \neq \WES \Rightarrow e' \in \WESS \setminus \WES$, y si $\LESS \neq \LES \Rightarrow e' \in \LESS \setminus \LES$. ENTONCES?
%Then for all $s$ such that 
%$\exists w \ldot s \runw{w}{\structure'} e'$, if $s \notin (\WES \cup \LES)$ then $s \notin (\WESS \cup \LESS)$.
\end{lemma}


Si se cerró un nuevo loop(line~\ref{line:isLoop}), por Lemma~\ref{lem:E'isNoneThenAllIsNone} alcanza con analizar si $e' 
\in \WESS \uplus \LESS$, y por Lemma~\ref{lem:newWinnersLosersAreNonePredecessors} propagar cualquier información nueva de $e'$ a sus predecesores. 

En la linea~\ref{line:getMaxLoop} computamos $\SCC$, el conjunto de estados que pertenecen a un loop que pasa por $e \step{\l}{\structure'} e'$ y nunca por $\WES \cup \LES$. 
Intuitivamente, cualquier supervisor para $e'$ va a depender de alguno de estos loops. O, en términos del Lemma~\ref{lem:newWinnersLosersAreNonePredecessors}, para que $e'$ cambie su estado, debe ser a través de un camino de estados $\NONE$. 

En la linea~\ref{line:CanBeWinning} usamos \texttt{canBeWinningLoop}($\SCC$) para chequear si existe algún estado marcado en $\SCC$ o si es posible escapar de $\SCC$ y alcanzar un "goal" en un paso. Esto distingue entre dos posibles opciones: $e' \in \WESS$ o $e' \in \LESS$ (ver 
Lemma~\ref{lem:canBeWinningLoopWorks}). 

\begin{lemma}\textbf{\emph{(Condición necesaria/suficiente para ganar/perder)}}
\label{lem:canBeWinningLoopWorks}
Sea $e \step{\l}{} e'$ la única diferencia entre dos eploraciones parciales, 
$\structure$ y $\structure'$. Sea $\SCC$ = \emph{\texttt{getMaxLoop($e$, 
$e'$)}}.
Si $e' \in \WESS \setminus \WES$ 
entonces \emph{\texttt{canBeWinningLoop}($\SCC$)}. Además, si \\ 
\emph{\texttt{canBeWinningLoop}($\SCC$)} entonces $e' \notin \LESS$ 
\end{lemma}


Si \texttt{canBeWinningLoop()} retorna true, en la  linea~\ref{line:lookingForWinner}, 
sabemos que si $e'$ cambia su estado es porque $e' \in \WESS$. Para ver si este cambio se produce, se realiza una computación estándar de punto fijo. 
%However, we introduce an optimization, namely that there must be new 
%$\NONE$-loop (i.e., a loop over states that are  $\NONE$)
%that includes the new transition that either reaches a marked state 
%or 
%can reach a winning state in one step. 
Sin embargo, el método \texttt{findNewGoalsIn} aplica una optimización basada en Lemma~\ref{lem:newWinnersLosersAreNonePredecessors}; solo considera estados que están en un $\NONE$-loop a través de la nueva transición ($\SCC$).

Si \texttt{canBeWinningLoop()} retorna false, entonces debemos comprobar si $e' \in \LESS$.
%  By Lemma~\ref{lem:newWinnersLosersAreNonePredecessors}
  %, we only consider states that are in a $\NONE$-loop 
%via the $e \step{\l}{} e'$ (i.e, the set $\SCC$). 
Esto puede hacerse de forma más eficiente que con un punto fijo usando el Lemma~\ref{lem:findErrorsWorks} que muestra que alcanza con observar si no es posible escapar de $\SCC$ alcanzando en un paso un estado que no esté en $\LES$. 



%Hack de titulo de lemma porque si no, no corta linea.
\begin{lemma}\textbf{\emph{(findNewErrorsIn es correcto y completo)}}
\label{lem:findErrorsWorks}
Si $\SCC =$ \emph{\texttt{getMaxLoop($e$, $e'$)}} $\wedge$\\ 
$\neg$\emph{\texttt{canBeWinningLoop}}($\SCC$) y \\
$P=\emph{\texttt{findNewErrorsIn}}(\SCC)$ entonces \\
$(e' \in \LESS \Rightarrow e' \in P 
\subseteq \LESS)$ $\wedge \, (e' \notin \LESS \Rightarrow P = 
\emptyset)$
\end{lemma}



Por motivos de eficiencia, \texttt{findNewGoalsIn} y
\texttt{findNewErrorsIn} no solo verifican si $e' \in \WESS$/$e' \in 
\LESS$ sino que también agregan estados ganadores/perdedores cuando pueden. La detección completa de nuevos estados ganadores y perdedores se hace finalmente con los procesos de propagación. 

Habiendo argumentado que la propiedad~~\ref{def:invariant} es válida, la correctitud y completitud le siguen de forma natural. 

Primero, notar que el algoritmo termina cuando logra determinar que $\init{e}$ está en $\LESS$ o $\WESS$. En el segundo caso, es simple construir un supervisor basándose en la estructura de exploración
$\structure$.\hfill$\qed$

\begin{theorem}[Correctitud y Completitud]
Sea $\E = (E,A_E^C)$ un problema de control composicional según Definition~\ref{def:control-problem}. Existe una solución para $\E$ si y solo si el algoritmo DCS retorna un supervisor para $\E$.
\end{theorem}

Demostración (Correctitud y completitud):
El teorema se desprende del invariante de ciclo del algoritmo (Definition~\ref{def:invariant}), el
Lemma~\ref{lem:WESandLesMonotonicity}, y el hecho de que en el peor caso todas las transiciones son agregadas a la estructura de exploración. Entonces,  $E = \structure = 
\unexploredToBottom{\structure} = \unexploredToTop{\structure}$.

