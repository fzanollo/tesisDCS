\begin{notation}
Decimos que un estado $s$ es ganador["winning"] (resp. perdedor["losing"]) en el problema $\E = 
(E, A_E^C)$ siendo $s$ el estado inicial de $E$ si hay una (resp. no hay una) solución para $\E$. Nos referimos a los estados ganadores y perdedores de $E$ cuando $A_E^C$ es inferible del contexto, también usamos $W_E$ y $L_E$ para denotar el conjunto de estados ganadores y perdedores de $\E$.
\end{notation}

El algoritmo (ver Listing~\ref{lst:dcs}) explora incrementalmente el espacio de estados de $E$ utilizando una estructura de exploración parcial ($ES$), añadiéndole una transición por vez.


\begin{definition}
[Exploración Parcial] \label{def:unexploredTo}
Sea $E = (S_E,$ $A_E, \D_E, \init{e}, M_E)$. Decimos que $ES$ es una exploración parcial de $E$ ($ES \subseteq E$) si $S_\structure \subseteq 
S_E$ y $\structure = (S_\structure,A_E, \D_\structure,\init{e},M_E 
\cap S_\structure)$, donde $ \D_\structure \subseteq (\D_E \cap 
(S_\structure \times A_E \times S_\structure))$. Escribimos $ES \subset E$ cuando $S_\structure \subset S_E$.
\end{definition}

Para explicar el algoritmo y argumentar su correctitud y completitud introducimos dos nuevos problemas de control para exploraciones parciales. Uno toma una visión optimista de la región no explorada ($\top$) asumiendo que todas las transiciones no exploradas llevan a un estado ganador. El otro toma una visión pesimista ($\bot$) asumiendo que las transiciones no exploradas llevan a estados perdedores.

\begin{definition}
[Problemas de Control $\top$ y $\bot$] \label{def:unexploredTo}

Sean $\E = (E, A_E^C)$, $E = (S_E,A_E,\D_E,\init{e},M_E)$ y $\structure = 
(S_\structure,A_E, \D_\structure,\init{e},M_E \cap S_\structure)$, y $\structure 
\subseteq E$.
\\
Definimos $\E_\top$ como $(\unexploredToTop{\structure}, A_E^C)$ donde 
$\unexploredToTop{\structure} = (S_\structure \cup \, \{\top\},A_E,\D_\top, 
\init{e}, 
(M_E \cap S_\structure)\, \cup \, \{\top\})$ y $\, \D_\top \, = \, \D_\structure 
\, 
\cup\, \{(s,\l, \top) 
\;$ $ | \; \exists s' \ldot (s, \l, s') \in (\D_E \setminus \D_\structure)\} \cup \{(\top, \l, \top) \, | \, \l \in A_E\}$ \\
Definimos $\E_\bot$ como $(\unexploredToBottom{\structure}, A_E^C)$ donde 
$\unexploredToBottom{\structure} = (S_\structure \, \cup \, 
\{\bot\},A_E,\D_\bot, 
\init{e}, M_E \cap S_\structure)$ y $\D_\bot = \D_\structure \, \cup \, \{(s,\l, 
\bot) \, | \, $ $ \exists s' \ldot (s, \l, s') \in (\D_E \setminus \D_\structure)\}$ 
\end{definition}

Usamos estos problemas de control para decidir tempranamente si un estado $s$ es ganador o perdedor en $E$ basado en lo que exploramos previamente en $\structure$. Si $s$ es ganador en $\unexploredToBottom{\structure}$ esto significa que sin importar a dónde lleven las transiciones no exploradas, $s$ también va a ser ganador en $E$. Similarmente, $s$ es perdedor en $E$ si es perdedor en $\unexploredToTop{\structure}$.
Lemma~\ref{lem:WESandLesMonotonicity} refuerza este razonamiento.


\begin{lemma}\textbf{\emph{(Monotonicidad de $\WES$ y $\LES$)}}
\label{lem:WESandLesMonotonicity}
Sean $\structure$ y $\structure'$ dos exploraciones parciales de $E$ tal que $\structure 
\subset \structure'$ entonces $\WES \subseteq \WESS$ y $\LES \subseteq 
\LESS$.
\end{lemma}

El algoritmo agrega iterativamente una transición de $E$ a $\structure$ a la vez y asegura que al final de cada iteración, los estados en $\structure$ están correcta y completamente clasificados en ganadores y perdedores si hay suficiente información de $E$ en $\structure$. Los conjuntos de estados $Errors$, 
$Goals$ y $\NONE$ se usan para este propósito.

\begin{property}[Invariante]
\label{def:invariant}
El loop principal del Algorithm~\ref{lst:dcs} tiene el siguiente invariante: 
$\structure \subseteq E \ \wedge $ $\forall s \in \structure \ldot (s\in\Goals 
\Leftrightarrow 
s \in 
\WES)$ $\wedge$  $(s\in\Errors 
\Leftrightarrow s \in \LES)$ $\wedge$  
$s\in\Errors\uplus\Goals\uplus\NONE$

% \begin{tabbing}
% ajfdlkadl\= \kill
% \> \= 
% \>$s\in\Goals \Leftrightarrow s \in \WES \wedge$ \\
% \> $s\in\Errors \Leftrightarrow s \in \LES \wedge$ \\
% \> $s\in\Errors\uplus\Goals\uplus\NONE$
% \end{tabbing}
\end{property}

La explicación del Algorithm~\ref{lst:dcs} que detallamos a continuación sirve también como un esquema de demostración para Property~\ref{def:invariant}.   

Para empezar, notar que la función \texttt{expandNext} 
(line~\ref{line:expand}) retorna una nueva transición $e \step{\l}{E} e'$ 
garantizando que $e$ ya se encontraba en $\structure$ y $e \in \NONE$. 
Esto significa que en cada iteración, hay algo de información nueva disponible para un estado que actualmente no está clasificado en ganador ni perdedor.

Si el estado $e'$ ya es clasificado como ganador en
$\unexploredToBottom{\structure}$ (line~\ref{line:isWinning}) o
perdedor en $\unexploredToTop{\structure}$ (line~\ref{line:isLosing}) 
entonces esta información necesita ser propagada a los estados en $\NONE$ para ver si pueden convertirse en ganadores en 
$\unexploredToBottom{\structure'}$ o perdedores en
$\unexploredToTop{\structure'}$. Tanto \texttt{propagateGoal} como
\texttt{propagateError} realizan un punto fijo estándar~\cite{Ramadge:1989:CDES} sobre
$\unexploredToBottom{\structure}$ y
$\unexploredToTop{\structure}$ pero solo sobre predecesores de $e'$ que están en $\NONE$. 
Lemma~\ref{lem:newWinnersLosersAreNonePredecessors} asegura la completitud de esta propagación restringida.

%Hack de titulo de lemma porque si no, no corta linea.
\begin{lemma}\textbf{\emph{(Ganadores/Perdedores nuevos tienen camino de estados-\textit{None} a transición nueva)}}
\label{lem:newWinnersLosersAreNonePredecessors}
Sea la transición $e \step{\l}{\structure} e'$ la única diferencia entre dos exploraciones parciales, $\structure$ y $\structure'$, de $E$. Si $s \notin (\WES \cup \LES)$ y $s \in (\WESS \cup \LESS)$, entonces hay $s_0, \ldots, s_n \notin (\WES \cup \LES)$ tal que $s = s_0 \wedge
s_0 \step{\l_0}{\structure}\ldots s_{n} \step{\l}{\structure} e'$.
\end{lemma}

Ya en la linea~\ref{line:isLoop} sabemos que $e'$ no es ganador en $\unexploredToBottom{\structure}$ ni perdedor en $\unexploredToTop{\structure}$, chequeamos si $e 
\step{\l}{\structure} e'$ cierra un nuevo loop. Si no es el caso, entonces no hay nada que hacer ya que $e'$ alcanza las mismas transiciones en $\structure'$ que en $\structure$. Entonces, $e' \notin 
(\WESS \cup \LESS)$ ya que cualquier supervisor para $e'$ en $\unexploredToBottom{\structure'}$ (resp. $\unexploredToTop{\structure'}$) es también un supervisor en $\unexploredToBottom{\structure}$ (resp. 
$\unexploredToTop{\structure}$) y vice versa.
%Thus, there is no new information about whether it is winning or 
%losing (Lemma~\ref{lem:noNewLoopImpliesNoNewInformation}). 
%
%%Hack de titulo de lemma porque si no, no corta linea.
%\begin{lemma}\textbf{\emph{(No new reachable transitions from e' implies no new 
%information)}}
%\label{lem:noNewLoopImpliesNoNewInformation}
%Let $e \step{\l}{} e'$ be the only difference between two partial explorations, 
%$\structure$ and $\structure'$ and $e' \notin \WES \cup \LES$. 
%If $\, \forall (s \step{\l'}{} s') \ldot e' \runw{w}{\structure} s  \step{\l'}{\structure} s' $ 
%if and only if $e' \runw{w}{\structure'} s  \step{\l'}{\structure'} s' $ then 
%$e' \notin \WESS \cup \LESS$
%\end{lemma}
%
Más aún, que no haya nueva información para $e'$ implica que no hay nuevos ganadores o perdedores (Lemma~\ref{lem:E'isNoneThenAllIsNone})

\begin{lemma}\textbf{\emph{(Nuevos ganadores/perdedores solo si $e'$ es un nuevo ganador/perdedor)}}
\label{lem:E'isNoneThenAllIsNone}
Sea $e \step{\l}{} e'$ la única diferencia entre dos exploraciones parciales, $\structure$ y $\structure'$. Si $\WESS \neq \WES \Rightarrow e' \in \WESS \setminus \WES$, y si $\LESS \neq \LES \Rightarrow e' \in \LESS \setminus \LES$. ENTONCES?
%Then for all $s$ such that 
%$\exists w \ldot s \runw{w}{\structure'} e'$, if $s \notin (\WES \cup \LES)$ then $s \notin (\WESS \cup \LESS)$.
\end{lemma}


Si se cerró un nuevo loop(line~\ref{line:isLoop}), por Lemma~\ref{lem:E'isNoneThenAllIsNone} alcanza con analizar si $e' 
\in \WESS \uplus \LESS$, y por Lemma~\ref{lem:newWinnersLosersAreNonePredecessors} propagar cualquier información nueva de $e'$ a sus predecesores. 

En la linea~\ref{line:getMaxLoop} computamos $\SCC$, el conjunto de estados que pertenecen a un loop que pasa por $e \step{\l}{\structure'} e'$ y nunca por $\WES \cup \LES$. 
Intuitivamente, cualquier supervisor para $e'$ va a depender de alguno de estos loops. O, en términos del Lemma~\ref{lem:newWinnersLosersAreNonePredecessors}, para que $e'$ cambie su estado, debe ser a través de un camino de estados $\NONE$. 

En la linea~\ref{line:CanBeWinning} usamos \texttt{canBeWinningLoop}($\SCC$) para chequear si existe algún estado marcado en $\SCC$ o si es posible escapar de $\SCC$ y alcanzar un "goal" en un paso. Esto distingue entre dos posibles opciones: $e' \in \WESS$ o $e' \in \LESS$ (ver 
Lemma~\ref{lem:canBeWinningLoopWorks}). 

\begin{lemma}\textbf{\emph{(Condición necesaria/suficiente para ganar/perder)}}
\label{lem:canBeWinningLoopWorks}
Sea $e \step{\l}{} e'$ la única diferencia entre dos eploraciones parciales, 
$\structure$ y $\structure'$. Sea $\SCC$ = \emph{\texttt{getMaxLoop($e$, 
$e'$)}}.
Si $e' \in \WESS \setminus \WES$ 
entonces \emph{\texttt{canBeWinningLoop}($\SCC$)}. Además, si \\ 
\emph{\texttt{canBeWinningLoop}($\SCC$)} entonces $e' \notin \LESS$ 
\end{lemma}


Si \texttt{canBeWinningLoop()} retorna true, en la  linea~\ref{line:lookingForWinner}, 
sabemos que si $e'$ cambia su estado es porque $e' \in \WESS$. Para ver si este cambio se produce, se realiza una computación estándar de punto fijo. 
%However, we introduce an optimization, namely that there must be new 
%$\NONE$-loop (i.e., a loop over states that are  $\NONE$)
%that includes the new transition that either reaches a marked state 
%or 
%can reach a winning state in one step. 
Sin embargo, el método \texttt{findNewGoalsIn} aplica una optimización basada en Lemma~\ref{lem:newWinnersLosersAreNonePredecessors}; solo considera estados que están en un $\NONE$-loop a través de la nueva transición ($\SCC$).

Si \texttt{canBeWinningLoop()} retorna false, entonces debemos comprobar si $e' \in \LESS$.
%  By Lemma~\ref{lem:newWinnersLosersAreNonePredecessors}
  %, we only consider states that are in a $\NONE$-loop 
%via the $e \step{\l}{} e'$ (i.e, the set $\SCC$). 
Esto puede hacerse de forma más eficiente que con un punto fijo usando el Lemma~\ref{lem:findErrorsWorks} que muestra que alcanza con observar si no es posible escapar de $\SCC$ alcanzando en un paso un estado que no esté en $\LES$. 



%Hack de titulo de lemma porque si no, no corta linea.
\begin{lemma}\textbf{\emph{(findNewErrorsIn es correcto y completo)}}
\label{lem:findErrorsWorks}
Si $\SCC =$ \emph{\texttt{getMaxLoop($e$, $e'$)}} $\wedge$\\ 
$\neg$\emph{\texttt{canBeWinningLoop}}($\SCC$) y \\
$P=\emph{\texttt{findNewErrorsIn}}(\SCC)$ entonces \\
$(e' \in \LESS \Rightarrow e' \in P 
\subseteq \LESS)$ $\wedge \, (e' \notin \LESS \Rightarrow P = 
\emptyset)$
\end{lemma}



Por motivos de eficiencia, \texttt{findNewGoalsIn} y
\texttt{findNewErrorsIn} no solo verifican si $e' \in \WESS$/$e' \in 
\LESS$ sino que también agregan estados ganadores/perdedores cuando pueden. La detección completa de nuevos estados ganadores y perdedores se hace finalmente con los procesos de propagación. 

Habiendo argumentado que la propiedad~~\ref{def:invariant} es válida, la correctitud y completitud le siguen de forma natural. 

Primero, notar que el algoritmo termina cuando logra determinar que $\init{e}$ está en $\LESS$ o $\WESS$. En el segundo caso, es simple construir un supervisor basándose en la estructura de exploración
$\structure$.\hfill$\qed$

\begin{theorem}[Correctitud y Completitud]
Sea $\E = (E,A_E^C)$ un problema de control composicional según Definition~\ref{def:control-problem}. Existe una solución para $\E$ si y solo si el algoritmo DCS retorna un supervisor para $\E$.
\end{theorem}

Demostración (Correctitud y completitud):
El teorema se desprende del invariante de ciclo del algoritmo (Definition~\ref{def:invariant}), el
Lemma~\ref{lem:WESandLesMonotonicity}, y el hecho de que en el peor caso todas las transiciones son agregadas a la estructura de exploración. Entonces,  $E = \structure = 
\unexploredToBottom{\structure} = \unexploredToTop{\structure}$.
