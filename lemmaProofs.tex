\begin{proof}
	
(Idea:	Para probar $\WES \subseteq \WESS$ mostramos que un supervisor para un estado $s$ en $\WES$ 
	puede ser usado como un supervisor para $s$ en $\WESS$. Para $\LES \subseteq 
	\LESS$, asumimos que hay un estado $s \in \LES \setminus \LESS $. Llegamos a una contradicción mostrando que el supervisor que $s$ debe tener en
	$\unexploredToTop{\structure'}$ es también un supervisor para $s$ en $\unexploredToTop{\structure}$.)\\
	

Si $s \in \WES $ entonces existe un supervisor $\sigma$ para el problema de control $\unexploredToBottom{\structure}$. Sea $Z$ tal que $\structure \subseteq Z$. Demostraremos que $\sigma$ es un supervisor para $\unexploredToBottom{Z}$. Esto requiere dos condiciones según la Definición~\ref{def:control-problem}. La primera, que $\sigma$ es controlable, es trivial ya que los conjuntos de eventos controlables y no controlables no fueron cambiados. 

Para la segunda, nonblocking, primero mostramos que $\L^\sigma(\unexploredToBottom{Z}) = 
\L^\sigma(\unexploredToBottom{\structure})$. \\

Si asumimos que $\L^\sigma(\unexploredToBottom{Z}) \not\subseteq \L^\sigma(\unexploredToBottom{\structure})$ y $w \in 
\L^\sigma(\unexploredToBottom{Z}) \setminus \L^\sigma(\unexploredToBottom{\structure})$, la corrida que verifica $w$ debe permanecer siempre en $Z$ o alcanzar eventualmente un estado $deadlock$ en $\unexploredToBottom{Z}$. En cualquier caso, sea $w_0$ el prefijo más largo en $\structure$. 
Sabemos que $w_0$ es un prefijo no vacío de $w$. Sea $\ell$ tal que $w_0.\ell$ es un prefijo de $w$. 
Por la definición de $\unexploredToBottom{\structure}$, $w_0.\ell$ alcanza un estado $deadlock$ en $\unexploredToBottom{\structure}$. Esto es una contradicción, ya que $\sigma$ es un supervisor para
$\unexploredToBottom{\structure}$. 

Para mostrar que $\L^\sigma(\unexploredToBottom{Z}) \supseteq \L^\sigma(\unexploredToBottom{\structure})$, asumimos que $w \in \L^\sigma(\unexploredToBottom{\structure})$. Si $w$ también está en $\L^\sigma(\structure)$ entonces debe pertenecer a $\L^\sigma(Z)$ y $\L^\sigma(\unexploredToBottom{Z})$. De otra forma, $w = w_0.\ell$ alcanza un estado $deadlock$ en $\unexploredToBottom{\structure}$. Como $w_0$ pertenece a $\L^\sigma(\structure)$, pertenecer también a $\L^\sigma(Z)$. Consideramos el estado $s$ alcanzado por  $w_0$ en $E$, debe tener una transición etiquetada como $\ell$ para justificar su inclusión en $\unexploredToBottom{\structure}$. En $Z$, el estado $s$ o tiene la transición y por lo tanto $w_0.\ell \in \L^\sigma(Z) \subseteq 
\L^\sigma(\unexploredToBottom{Z})$, o no tiene la transición, pero el estado en $\unexploredToBottom{Z})$ tiene una transición $\ell$ a un estado $deadlock$, por lo tanto $w_0.\ell \in 
L^\sigma(\unexploredToBottom{Z})$.

Ahora, sabiendo que $\L^\sigma(\unexploredToBottom{Z}) = 
\L^\sigma(\unexploredToBottom{\structure})$, procedemos a $nonblocking$. Sea una palabra $w \in 
\L^\sigma(\unexploredToBottom{Z})$ que no puede ser extendida con $w'$ tal que $w.w'$ se encuentra en $L^\sigma(\unexploredToBottom{Z})$ y alcanza un estado marcado de $\unexploredToBottom{Z}$. 
Como $w$ también se encuentra en $L^\sigma(\unexploredToBottom{\structure})$ entonces, como $\sigma$ es un supervisor para $\unexploredToBottom{\structure}$, existe un $w'$ tal que
$w.w' \in L^\sigma(\unexploredToBottom{\structure}) = L^\sigma(\unexploredToBottom{Z})$ y alcanza un estado marcado. Notar que la corrida para $w.w'$ siempre se encuentra en $\structure$, lo que significa que la corrida también está en $\unexploredToBottom{Z}$. Finalmente llegamos a una contradicción.\\

Para demostrar que $\LES \subseteq \LESS$, asumimos que existe un estado $s \in \LES \setminus \LESS$. Como $s \notin \LESS$, tiene un supervisor $\sigma$ en $\unexploredToTop{\structure'}$, pero $s \in \LES$ por lo que no puede existitr un supervisor válido $\sigma'$ para $s$ en $\unexploredToTop{\structure}$. Esto es falso, más aún, mostraremos que si $\sigma$ es un supervisor para $s$ en $\unexploredToTop{\structure'}$, entonces hay un supervisor válido $\sigma'$ para $s$ en cualquier $\unexploredToTop{Z}$ si $Z \subseteq \structure$.

$\sigma$ es un supervisor valido en $\unexploredToTop{\structure'}$, por lo que cualquier palabra en $\L^\sigma(\unexploredToTop{\structure'})$ puede ser extendida para alcanzar un estado marcado. Solo hay una cantidad finita de estados en  $\unexploredToTop{\structure'}$, por lo que deben existir $w'$ y $w''$ tal que $w.w'.w'' \in \L^\sigma(\unexploredToTop{\structure'})$, $w.w'$ llega a un estado marcado, y $w.w'.w''$ llega al mismo estado que $w.w'$.

Si $w.w'.w''$ está en $\L^\sigma(Z)$, entonces no hay nada que hacer, es claro que $\sigma$ tiene la misma forma de extender $w$ en $\unexploredToTop{Z}$. Si no, notemos que $w.w'.w'' = w_0.l.w_1$ tal que $w_0$ es el prefijo más largo de $w.w'.w''$ en $\L^\sigma(Z)$, esto significa que $w_0.l$ alcanza el estado marcado ganador $\top$, y desde ahí toda extensión de la palabra solo puede permanecer en ese mismo estado, por lo tanto, $\sigma$ también es un controlador válido en  $\unexploredToTop{Z}$.
 
\begin{flushright}
$\square$
\end{flushright}

\end{proof}


\begin{proof}
	(Idea: Si $s$ no es un predecesor de $e'$, como $e \step{l}{\structure'} e'$ es la única diferencia entre $\structure$ y $\structure'$, entonces los decendientes de $s$ son los mismos, 
	por lo tanto sus posibles supervisores en $\unexploredToTop{\structure'}$ y
	$\unexploredToBottom{\structure'}$ no cambiaron. Entonces, $s \notin \WESS \cup 
	\LESS$ lo cual es una contradicción.
	
	Como paso siguiente probamos que hay al menos un camino desde $s$ a $e'$ a través de estados $\NONE$ por contradicción asumiendo que todos los caminos a $e'$ en $\structure'$ atraviesan un estado $s' \in 
	(\WES \cup \LES)$. Un supervisor $\sigma$ de $s$ en 
	$\unexploredToTop{\structure}$ no va a alcanzar estados en $\LES$, 
	por lo tanto todo $s'$ que alcance va a tener un supervisor $\sigma_{s'}$ para 
	$\unexploredToBottom{\structure}$. Usamos $\sigma$ y $\sigma_{s'}$ para construir un supervisor para $s$ en $\unexploredToTop{\structure'}$ para mostrar que $s 
	\not\in \LESS$.
	Un supervisor para $s$ en $\unexploredToBottom{\structure'}$ no puede existir porque de otra forma podríamos usarlo para construir un supervisor para $s$ en 
	$\unexploredToBottom{\structure}$ usando un razonamiento similar al anterior. Esto significa que $s \in 
	\WES$ contradiciendo la hipótesis. )\\


Si un estado $s$ no se encuentra en $\WES \cup \LES$ es porque tiene un supervisor $\sigma$ en $\unexploredToTop{\structure}$ pero no tiene uno para $\unexploredToBottom{\structure}$. Esto depende únicamente de los descendientes de $s$, dado que éstos son los únicos estados que $\sigma$ puede alcanzar. Si $s$ no es un predecesor de $e'$, y $e \step{l}{\structure'} e'$ es la única diferencia entre $\structure$ y $\structure'$ entonces los decendientes de $s$ son los mismos, por lo que los posibles supervisores no tuvieron ningún cambio, y $s$ sigue siendo NONE.

Lo que no es tan claro, es que $s$ no tiene nuevos supervisores posibles si tiene un camino que puede alcanzar $e'$ pero solo pasando por al menos un estado de $\WES \cup \LES$. Asumiendo que debe pasar por estados en $\WES \cup \LES$ mostramos que:

\begin{itemize}
	\item Sabiendo que $s$ tenía un supervisor $\sigma$ en $\unexploredToTop{\structure}$, mostramos que $s$ tiene un supervisor válido $\sigma'$ en $\unexploredToTop{\structure'}$:
	
	$\sigma'(w) = \sigma(w)$ si no existe un $w_0$ sufijo de $w$ tal que $ s \runw{w_0}{\structure'} s_i \wedge s_i \in \LES \cup \WES$. 
	
	$\sigma'(w) = \sigma_{s_i}(w_1)$ donde $w_0$ es el sufijo más corto de $w = w_0.w_1$ tal que $ s \runw{w_0}{\structure'} s_i \wedge s_i \in \WES$. $\sigma_{s_i}$ es el supervisor que sabemos que  $s_i$ tiene en $\unexploredToBottom{\structure}$ ya que $s_i \in \WES$, y que cada supervisor válido en $\unexploredToBottom{\structure}$ es también válido en $\unexploredToTop{\structure}$.
	
	Como $\sigma$ es un supervisor válido, sabemos que no puede alcanzar estados en $\LES$.
	
	Finalmente, es claro que $\sigma'$ es un supervisor válido para $s$ en $\unexploredToTop{\structure'}$. Notar que $\sigma'$ no depende de la nueva transición.
	
	
	
	
	\item Sabiendo que $s$ no tiene supervisor en $\unexploredToBottom{\structure}$, mostramos que $s$ no tiene supervisor en $\unexploredToBottom{\structure'}$ asumiendo que tiene uno y llegando a una contradicción:
	
	Suponemos que existe un supervisor $\sigma'$ para $s$ en $\unexploredToBottom{\structure'}$, y que $e \step{l}{\structure'} e'$ es la única diferencia entre $\structure$ y $\structure'$.
	
	Con $\sigma'$ construimos $\sigma$, un supervisor para $s$ en $\unexploredToBottom{\structure}$.
	
	$\sigma(w) = \sigma'(w)$ si no existe un $w_0$ que sea prefijo de $w$ y que $s \runw{w_0}{\structure} s_i \wedge s_i \in \WES \cup \LES$. Como $\sigma'$ es un supervisor válido, sabemos que no puede alcanzar estados en $\LESS$. Notar que $w$ no puede alcanzar $e \step{\l}{\structure'} e'$ porque $s$ no tiene un camino de estados $\NONE$ a $e'$.
	
	Si $w=w_0.w_1$ y $s \runw{w_0}{\structure} s' \wedge s' \in \WES$ entonces $\sigma(w_0.w_1) = \sigma_{s'}(w_1)$ donde $\sigma_{s'}$ es el supervisor para $s'$ en $\unexploredToBottom{\structure}$. Notar que una vez que se alcanza $s'$ siempre seguimos $\sigma_{s'}$.
	
	Como $\sigma$ nunca alcanza la nueva transición sabemos que $\sigma$ es válido en $\unexploredToBottom{\structure}$.
		
	Vemos entonces que asumiendo que existe un supervisor válido $\sigma'$ para $s$ en $\unexploredToBottom{\structure'}$ estamos implicando la existencia de un supervisor $\sigma$ para $s$ en $\unexploredToBottom{\structure}$. ABS! \\
	
\end{itemize}
\begin{flushright}
	$\square$
\end{flushright}
\end{proof}

\begin{proof}
	(Idea: Asumiendo $e' \notin \WESS$, usamos un testigo $s$ de $\WESS \neq 
	\WES$ para llegar a una contradicción. El estado $s$ debe tener un supervisor en 
	$\WESS$ que evita $e \step{\l}{} e'$, la única diferencia entre $\unexploredToBottom{\structure}$ y $\unexploredToBottom{\structure'}$. Este supervisor entonces es también un supervisor para $s$ en $\WES$ llegando a un absurdo. 
	
	Si asumimos $e' \notin \LESS$.  usamos un testigo $s$ de $\LESS \neq \LES$ para llegar a una contradicción. Notar que como $e' \notin \LESS$, hay un supervisor $\sigma$ desde $e'$ en  
	$\unexploredToTop{\structure'}$. Como $s \notin \LES$ también debe haber un supervisor $\sigma'$ en $\unexploredToTop{\structure}$. Construimos un nuevo supervisor para $\unexploredToTop{\structure'}$ desde $s$ que funciona exactamente como $\sigma'$ pero cuando alcanza $e \step{\l}{} e'$ se comporta como $\sigma$. Este nuevo supervisor prueba que $s \in \LESS$ lo cual es una contradicción.)\\


Probamos ambas implicaciones por contradicción. 

Primero asumimos que $e' \not \in 
\WESS \setminus \WES$. Notar que como $e' \notin \WES$ entonces $e' \notin \WESS$. Como $\WESS \neq \WES$ y por la monotonicidad del (Lemma\ref{lem:WESandLesMonotonicity}) debe existir un estado $s$ tal que $s \in \WESS \setminus 
\WES$, entonces $s$ debe tener un supervisor en $\WESS $. Este supervisor no puede alcanzar $e'$ porque si lo hiciera, debería haber un supervisor para $e'$ y comenzamos asumiendo que $e' 
\notin \WESS$. Más aún, si el supervisor alcanzara $e$, entonces $\l$ debe ser controlable (si fuera no controlable, el supervisor alcanzaría $e'$ lo cual ya establecimos no es posible). Entonces, el supervisor evita $e \step{\l}{} e'$ lo que significa que debe ser también un supervisor para $s$ en $\unexploredToBottom{\structure}$ (i.e.,  $s \in \WES$) y alcanzamos una contradicción.

Ahora asumimos $e' \not\in \LESS \setminus \LES$. Notar que como $e' \notin \LES$ entonces $e' \notin \LESS$. Sea $s \in \LESS$ y $s \not\in \LES$. Sabemos que desde $s$ debe haber un supervisor $\sigma'$ para  $\unexploredToTop{\structure}$. Este supervisor puede o ser también un supervisor para $\structure$ o alcanzar el estado $\top$ en $\unexploredToTop{\structure}$. En el primer caso, es también un supervisor en $\structure'$ y en $\unexploredToTop{\structure'}$, una contradicción. En el segundo caso, o usa una transición que no se encuentra ni en $\structure'$ ni en $\structure$, lo que significa que en $\unexploredToTop{\structure'}$ va a alcanzar un estado ganador $\top$; o usa una transición que se encuentra en $\structure'$ pero no en $\structure$ lo que lleva a $e'$. Como sabemos que $e'$ no se encuentra en $\LESS$, entonces cuenta con un supervisor en  $\unexploredToTop{\structure'}$, entonces sabemos que existe un supervisor $\sigma''$ que incluye tanto a  $\sigma'$ como al supervisor para $e'$. Finalmente, $\sigma''$ es un supervisor para $s$ en  $\unexploredToTop{\structure'}$, pero $s \notin \LESS$, ABS!	
\begin{flushright}
	$\square$
\end{flushright}
\end{proof}

\begin{proof}
	(Idea: Para probar que $e' \in \WESS \setminus \WES$ implica \texttt{canBeWinningLoop($\SCC$)}, asumimos \\ $\neg$\texttt{canBeWinningLoop($\SCC$)} y mostramos que $e' \notin \WESS 
	\setminus \WES$. Para esto, basta con ver que si
	$\neg$\texttt{canBeWinningLoop($\SCC$)} entonces para alcanzar un estado marcado desde $e'$ se debe salir de $\SCC$ a un estado $s \notin \SCC \cup \WES$ lo que implica $s \notin \WESS$ 
	ya que $s$ no tiene ningún camino de estados none que llegue a $e \step{\l}{} e'$  (Lemma~\ref{lem:newWinnersLosersAreNonePredecessors}).
	Como $s$ no tiene supervisor en $\unexploredToBottom{\structure'}$, es imposible que $e'$ tenga uno. 
	
	Para probar que \texttt{canBeWinningLoop($\SCC$)} implica $e' \notin \LESS$ construimos un supervisor $\sigma'$ para $e'$ en $\unexploredToTop{\structure'}$ de la siguiente forma:
	Para una traza que se quede dentro de $\SCC$, solo elegimos sucesores controlables que no estén en $\LES$. Notar que no puede haber sucesores no controlables en $\LES$ ya que
	$\SCC \cap \LES= \emptyset$. Tan pronto como la traza sale de $\SCC$ a un estado $s'$ usamos el supervisor para $s'$ en $\unexploredToTop{\structure'}$. 
	Como $s'$ no puede alcanzar $e \step{\l}{\structure'} 
	e'$ usando estados $\NONE$, por el 
	Lemma~\ref{lem:newWinnersLosersAreNonePredecessors}, $s'$ debe tener el supervisor que necesitamos.)	\\


Sea $e' \in \WESS \setminus \WES$ pero $\neg$\texttt{canBeWinningLoop($\SCC$)}.
Existe un supervisor $\sigma$ para $e'$ en $\unexploredToBottom{\structure'}$, esto significa que debe existir un camino $w$ desde $e'$ hasta un estado marcado $m$. 
Dado que $\neg$\texttt{canBeWinningLoop($\SCC$)}, no hay estados marcados en $\SCC$, $w$ debe salir de $\SCC$. 
Sea $s$ el primer estado que alcanza $w$ fuera de $\SCC$, $s$ pertenece a $\LES \cup \NONE$, y no tiene un camino de estados $\NONE$ hasta $e'$ entonces según Lemma~\ref{lem:newWinnersLosersAreNonePredecessors} $s$ va a seguir sin cambiar su estado.
Dado que $s \notin \WESS$, $s$ no tiene un supervisor en $\unexploredToBottom{\structure'}$, pero $\sigma$ acepta corridas que llevan a $s$, llegamos a un absurdo.

Asumiendo \texttt{canBeWinningLoop($\SCC$)} simplemente construimos un supervisor $\sigma^4$ para $e'$ en $\unexploredToTop{\structure'}$ para probar que $e' \notin \LESS$

Definimos $\sigma^4$ tal que habilita todas las transiciones no controlables (para que sea $controllable$).

$\sigma^4(w_0.w_1) = \sigma_{s'}(w_1)$ si $w_0$ es el camino más corto tal que existe $s' \notin \SCC \wedge s'\notin \LES$ y $e' \runw{w_0}{\unexploredToTop{\structure'}} s'$.
En otro caso $e' \runw{w}{\unexploredToTop{\structure'}} p \wedge p \in \SCC$ entonces para toda $\l'$ controlable, $\l' \in\sigma^4(w)$ si y solo si $\exists  p' \ldot p \step{l'}{} p' \wedge p' \notin \LES$.
%todas las transiciones que van al loop mas todas las transiciones que salen a un estado NONE (o TOP), i.e. no error. 

Usamos los supervisores $\sigma_{s'}$ donde $s'$ es tal que existe $s \in \SCC$ y $s\step{\l'}{\unexploredToTop{\structure'}} s'\wedge s' \notin \SCC \wedge s'\notin \LES$. Si no existe tal $s'$, sabemos que debe haber un estado marcado en $\SCC$ y $\sigma^4$ nunca abandona el conjunto $\SCC$ ya que todos los estados alcanzables desde $\SCC$ pertenecen a $\LES$.\\

Probaremos que $\sigma^4$ es $controllable$ y $non-blocking$.

Dado que habilitamos todas las transiciones no controlables, $\sigma^4$ es trivialmente $controllable$.

Para $non-blocking$, sea $w$ compatible con $\sigma^4$, mostraremos que puede ser extendido. Si $w = w_0.w_1$ y $w_0$ es la palabra más corta tal que existe una $s' \notin \SCC \wedge s' \notin \LES$ y $e' \runw{w_0}{\unexploredToTop{\structure'}} s'$. Entonces por definición de $\sigma^4$ sabemos que  $\sigma^4(w_0.w_1.w_2) = \sigma_{s'}(w_1.w_2)$ para todo $w_2$. Ya que $\sigma_{s'}$ es $non-blocking$, existe un $w_2$ tal que $\sigma_{s'}(w_1.w_2)$ alcanza un estado marcado. Entonces $\sigma^4(w_0.w_1)$ puede ser extendido para alcanzar ese estado marcado. 

En otro caso, $w$ nunca abandona $\SCC$. Debemos probar que para todo $\l'$ tal que $w.\l'$ sea consistente con $\sigma^4$, $w.\l'$ puede ser extendido con un $w'$ para alcanzar un estado marcado. Sea $p'$ tal que $e' \runw{w.\l'}{} p'$. Si $p' \notin \SCC$ entonces $\sigma^4(w.\l') = \sigma_{p'}(\lambda)$ y, como antes, sabemos que $\sigma_{p'}$ es $non-blocking$ entonces $w.l'$ puede extenderse para llegar a un estado marcado.

Si $p' \in \SCC$, entonces $w.l'$ puede extenderse para llegar a cualquier $s$ en $\SCC$. Sabemos que o existe un estado marcado en $\SCC$ o algún estado en $\WES$ es alcanzable en un paso desde $\SCC$, de cualquier forma podemos extender $w.l'$ para llegar a un estado marcado.\\
\begin{flushright}
	$\square$
\end{flushright}
\end{proof}


\begin{proof}
	(Idea: Dividimos la prueba según la estructura del  if/then/else de \texttt{findNewErrorsIn}. 
	En el caso de que el \texttt{if} sea true, es suficiente probar que $e' \notin \LESS$. Para esto, 
	construimos un supervisor $\sigma'$ para $e'$ en $\unexploredToTop{\structure'}$ de la siguiente forma: Para una traza que se queda dentro de $\SCC$, solo tomamos sucesores controlables que no estén en $\LES$. Notar que no puede haber sucesores no controlables en $\LES$ ya que
	$\SCC \cap \LES= \emptyset$. Tan pronto como la traza sale de $\SCC$ al estado $s'$ usamos el supervisor de $s'$ en $\unexploredToTop{\structure'}$. 
	Como $s'$ no puede alcanzar $e \step{\l}{\structure'} 
	e'$ usando estados $\NONE$, por el 
	Lemma~\ref{lem:newWinnersLosersAreNonePredecessors}, $s'$ debe tener tal supervisor. 
	
	Cuando el \texttt{if} es false, alcanza con probar que $P = \SCC \subseteq \LESS$. Alcanzamos una contradicción asumiendo que $s \in \SCC \setminus \LESS$: Si $s 
	\notin \LESS$ entonces tiene un supervisor $\sigma$ que acepta una traza $w$ alcanzando un estado marcado. Como no hay estados marcados en $\SCC$, $w$ alcanza un estado $s' \notin \SCC$. Como el
	\texttt{if} era false, $s' \in \LESS$ por lo que $\sigma$ no es un supervisor.)\\
	


En primer lugar, sabemos que cada estado $s' \notin \SCC$ tal que $\exists s \in \SCC \ldot s 
\step{l}{} s'$, puede o ser y seguir siendo un estado perdedor ($s' \in LES \wedge s' \in LESS$) o es y seguirá siendo $\NONE$ (porque $s$ no es un predecesor$-NONE$ de un estado en $\SCC$, de otra forma $s$ estaría en $\SCC$). 

Esto significa que ningún estado $s' \notin \SCC \wedge s' \notin \LES$ puede ser forzado a un estado en $\LESS$. Entonces, si alcanzamos un estado $\NONE$ sabemos que tiene un supervisor válido $\sigma_{s'}$ en $\unexploredToTop{\structure'}$.

En el caso de que la declaración $\texttt{if}$ sea verdad, probaremos que $e' \notin \LESS$:

Usamos el mismo $\sigma^4$ de la demostración Lemma \ref{lem:canBeWinningLoopWorks}. Ya sabemos que $\sigma^4$ es tanto $controllable$ como $non-blocking$ en esta situación por el Lemma anterior.\\

De otra forma entramos en el bloque $\texttt{else}$:

Si $\nexists s \in \SCC \ldot \trimlst{s \step{\l'}{\unexploredToTop{\structure'}}  s'} \wedge (s' \notin \SCC \wedge s' \notin \Errors)$ probamos que $\forall s \in \SCC$, $s \in \LESS$

Sea $\sigma'$ un supervisor para $s$ en $\unexploredToTop{\structure'}$ entonces $\exists w'$ tal que  $s \runw{\lambda.w'}{\unexploredToTop{\structure'}} e_m \wedge e_m \in M_{\unexploredToTop{\structure'}}$. Ya que no hay estados marcados en $\SCC$, partiendo desde $s$ y siguiendo $w'$ eventualmente se abandona $\SCC$.

Sea $w' = w_0.w_1$ tal que $w_0$ es la palabra más corta tal que $s \runw{w_0}{\unexploredToTop{\structure'}} s' \wedge s' \notin \SCC$. Dado que $s'\in \Errors \Rightarrow s' \in \LES$ no es posible que un supervisor válido $\sigma'$ acepte palabras que alcancen ese estado. ABS! Entonces no hay supervisor para $s$ en $\unexploredToTop{\structure'}$ lo que implica $\forall s \in \SCC$, $s \in \LESS$.
\begin{flushright}
	$\square$
\end{flushright}
\end{proof}