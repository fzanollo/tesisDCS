\paragraph*{Proof Lemma \ref{lem:WESandLesMonotonicity}}


\textbf{Proof Sketch}
To prove $\WES \subseteq \WESS$ we show that a supervisor for a 
state $s$ in $\WES$ 
can be used as a supervisor from $s$ in $\WESS$. For $\LES \subseteq 
\LESS$, we assume 
there is a state $s \in \LES \setminus \LESS $. We reach a 
contradiction by showing that the 
supervisor that $s$ must have in 
$\unexploredToTop{\structure'}$ is also a 
supervisor for $s$ in $\unexploredToTop{\structure}$.
\textbf{End Proof Sketch}


If $s \in \WES $ then there exists a supervisor $\sigma$ for the control problem 
$\unexploredToBottom{\structure}$. Let $Z$ such that $\structure \subseteq Z$. we will 
show that $\sigma$ is a supervisor for $\unexploredToBottom{Z}$.  This requires showing two 
conditions as per Definition~\ref{def:control-problem}. The first, that $\sigma$ is 
controllable, is trivial as the set of controllable and uncontrollable events have not changed. 

For the second, nonblocking, we first prove that $\L^\sigma(\unexploredToBottom{Z}) = 
\L^\sigma(\unexploredToBottom{\structure})$. 

Assume that $\L^\sigma(\unexploredToBottom{Z}) \not\subseteq \L^\sigma(\unexploredToBottom{\structure})$.
If $w \in 
\L^\sigma(\unexploredToBottom{Z}) \setminus \L^\sigma(\unexploredToBottom{\structure})$. 
The run that witnesses $w$ must either always be in $Z$ or eventually reach a deadlock state 
in $\unexploredToBottom{Z}$. In either case, let $w_0$ be the longest prefix in $\structure$. 
We know $w_0$ is a proper prefix of $w$. Let $\ell$ such that $w_0.\ell$ is a prefix of $w$. 
By definition of $\unexploredToBottom{\structure}$, $w_0.\ell$ reaches a deadlock state in  
$\unexploredToBottom{\structure}$. This is a contradiction as $\sigma$ is a supervisor for 
$\unexploredToBottom{\structure}$. 

To show that $\L^\sigma(\unexploredToBottom{Z}) \supseteq \L^\sigma(\unexploredToBottom{\structure})$, assume $w \in \L^\sigma(\unexploredToBottom{\structure})$. If $w$ is also in $\L^\sigma(\structure)$ it is also in $\L^\sigma(Z)$ and $\L^\sigma(\unexploredToBottom{Z})$. Otherwise, $w = w_0.\ell$ hits a 
deadlock state in $\unexploredToBottom{\structure}$. As $w_0$ in $\L^\sigma(\structure)$, 
it is also in $\L^\sigma(Z)$. Consider the state $s$ reached by $w_0$ in $E$, it must have a 
transition labelled $\ell$ to justify its addition to $\unexploredToBottom{\structure}$. In, $Z$ 
state $s$ either has the transition and thus  $w_0.\ell \in \L^\sigma(Z) \subseteq 
\L^\sigma(\unexploredToBottom{Z})$, or it does not have the transition but then the state in 
$\unexploredToBottom{Z})$ has an $\ell$ transition to a deadlock state, hence $w_0.\ell \in 
L^\sigma(\unexploredToBottom{Z})$.

Now, assume a word $w \in 
\L^\sigma(\unexploredToBottom{Z})$ that cannot be extended with $w'$ such that $w.w'$ is 
in $L^\sigma(\unexploredToBottom{Z})$ and reaches a marked state of $\unexploredToBottom{Z}$. 
As  $w$ is also in $L^\sigma(\unexploredToBottom{\structure})$ then, as
$\sigma$ is a supervisor for $\unexploredToBottom{\structure}$, there is a $w'$ such that 
$w.w' \in L^\sigma(\unexploredToBottom{\structure}) = L^\sigma(\unexploredToBottom{Z})$ 
and hits a marked state. Note that the run for $w.w'$ is always in $\structure$, which means 
that the run is also in $\unexploredToBottom{Z}$. Thus we reach a contradiction.\\

To see that $\LES \subseteq \LESS$, assume there is a state $s \in \LES \setminus \LESS$. Since $s \notin \LESS$, it has a supervisor $\sigma$ in $\unexploredToTop{\structure'}$, but $s \in \LES$ so there can't exists a valid supervisor $\sigma'$ for $s$ in $\unexploredToTop{\structure}$. This is false, even more, we show that if $\sigma$ is a supervisor for $s$ in $\unexploredToTop{\structure'}$, then there is a supervisor $\sigma'$ for $s$ in any $\unexploredToTop{Z}$ such that $Z \subseteq \structure$.

$\sigma$ is a valid supervisor in $\unexploredToTop{\structure'}$, so every word in $\L^\sigma(\unexploredToTop{\structure'})$ can be extended to reach a marked state, and there are only finite states in $\unexploredToTop{\structure'}$, there must be a $w'$ and $w''$ such that $w.w'.w'' \in \L^\sigma(\unexploredToTop{\structure'})$, $w.w'$ reaches a marked state, and $w.w'.w''$ reaches the same state as $w.w'$.

If $w.w'.w''$ is in $\L^\sigma(Z)$, then there is nothing to be done, it is clear that $\sigma$ still has a way to extend $w$ in $\unexploredToTop{Z}$. Otherwise, we note that $w.w'.w'' = w_0.l.w_1$ such that $w_0$ is the longest prefix of $w.w'.w''$ in $\L^\sigma(Z)$, this means that $w_0.l$ reaches the marked winning state $\top$, and from there every extension of the word can only remain in this state, thus, $\sigma$ is also a valid controller for $\unexploredToTop{Z}$.
 
\begin{flushright}
$\square$
\end{flushright}


\paragraph*{Proof Lemma 
\ref{lem:newWinnersLosersAreNonePredecessors}}
\textbf{{Begin Proof Sketch}}

We first show there is a path from $s$ to $e'$ by assuming there is none. If $s \notin 
\WES \cup \LES$ then it has a supervisor $\sigma$ in 
$\unexploredToTop{\structure}$ but it does not have one in 
$\unexploredToBottom{\structure}$. This depends entirely on the descendants of 
$s$, since those are the only states that supervisors can reach in $\structure$. If $s$ 
is not a predecessor of $e'$, as $e \step{l}{\structure'} e'$ is the only difference 
between $\structure$ and $\structure'$, then the descendants of $s$ are the same, 
thus its possible supervisors in $\unexploredToTop{\structure'}$ and 
$\unexploredToBottom{\structure'}$ are unchanged. Thus, $s \notin \WESS \cup 
\LESS$ which is a contradiction.

We then prove that there is at least one path from $s$ to $e'$ via $\NONE$ states by 
contradiction assuming that all paths to $e'$ in $\structure'$ cross a state $s' \in 
(\WES \cup \LES)$. A supervisor $\sigma$ from $s$ in 
$\unexploredToTop{\structure}$ will not reach states in $\LES$, 
thus for all $s'$ it crosses they will have a supervisors $\sigma_{s'}$ for 
$\unexploredToBottom{\structure}$. We use $\sigma$ and $\sigma_{s'}$  to 
build a supervisor for $s$ in $\unexploredToTop{\structure'}$ to show that $s 
\not\in \LESS$.
A supervisor for $s$ in $\unexploredToBottom{\structure'}$ cannot exist because 
otherwise we can use it to build a supervisor for $s$ in 
$\unexploredToBottom{\structure}$ using a similar reasoning as before. This means 
that $s \in 
\WES$ which contradicts the hypothesis. 






%Possible supervisors for a state $s$ depend only on its descendants, thus if the descendants of $s$ haven't changed between to partial explorations $s$ has no new options for supervisors and it's status can't change.
%
%What is not so clear, is that $s$ has no new possible supervisors if it has a path to reach $e'$ but only passing through states in $\WES \cup \LES$. Let $s'$ be the first state of the path between $s$ and $e'$ that is not $\NONE$.
%
%We can see that if $s' \in \WES$ then once we reach it, we can simply follow $\sigma_{s'}$, its supervisor on $\unexploredToBottom{\structure}$. Note that $\sigma_{s'}$ does not use $e \step{l}{\structure'} e'$. 
%
%If $s' \in \LES$, no valid supervisor should reach $s'$ and thus won't reach $e \step{l}{\structure'} e'$. 
%
%Then, whether $s$ has a supervisor in $\unexploredToTop{\structure'}$ or it doesn't have one in $\unexploredToBottom{\structure'}$ does not depend on $e \step{l}{\structure'} e'$. Finally $s$ is not in $\WESS \cup \LESS$.

\textbf{{End Proof Sketch}}

If a state $s$ is not in $\WES \cup \LES$ it is because it has a  supervisor $\sigma$ in $\unexploredToTop{\structure}$ but it doesn't have one in $\unexploredToBottom{\structure}$. This depends entirely on the descendants of $s$, since those are the only states that $\sigma$ can reach. If $s$ is not a predecessor of $e'$, and $e \step{l}{\structure'} e'$ is the only difference between $\structure$ and $\structure'$ then the descendants of $s$ are the same, thus its possible supervisors haven't changed, and $s$ is still NONE.

What is not so clear, is that $s$ has no new possible supervisors if it has a path to reach $e'$ but only by passing through at least one state in $\WES \cup \LES$. Assuming it must pass through states in $\WES \cup \LES$ we show that:

\begin{itemize}
	\item knowing $s$ had a supervisor $\sigma$ in $\unexploredToTop{\structure}$, we show $s$ has a valid supervisor $\sigma'$ in $\unexploredToTop{\structure'}$:
	
	$\sigma'(w) = \sigma(w)$ if there is no $w_0$ suffix of $w$ such that $ s \runw{w_0}{\structure'} s_i \wedge s_i \in \LES \cup \WES$. 
	
	$\sigma'(w) = \sigma_{s_i}(w_1)$ where $w_0$ is the shortest suffix of $w = w_0.w_1$ such that $ s \runw{w_0}{\structure'} s_i \wedge s_i \in \WES$. $\sigma_{s_i}$ is the supervisor we know $s_i$ had in $\unexploredToBottom{\structure}$ because $s_i \in \WES$, and every supervisor valid in $\unexploredToBottom{\structure}$ is also valid in $\unexploredToTop{\structure}$.
	
	Since $\sigma$ is a valid supervisor, we know it will not reach states in $\LES$.
	
	Finally it is clear that $\sigma'$ is a valid supervisor for $s$ in $\unexploredToTop{\structure'}$. Note that $\sigma'$ does not depend of the new transition.
	
	
	
	
	\item Knowing $s$ had no supervisor in $\unexploredToBottom{\structure}$, we show that $s$ has no supervisor in $\unexploredToBottom{\structure'}$ by assuming there is one and reaching a contradiction:
	
	Suppose there is a supervisor $\sigma'$ for $s$ in $\unexploredToBottom{\structure'}$, and that $e \step{l}{\structure'} e'$ is the only difference between $\structure$ and $\structure'$.
	
	With $\sigma'$ we will build $\sigma$, a supervisor for $s$ in $\unexploredToBottom{\structure}$.
	
	$\sigma(w) = \sigma'(w)$ if there is no $w_0$ which is a prefix of $w$ and $s \runw{w_0}{\structure} s_i \wedge s_i \in \WES \cup \LES$. Since $\sigma'$ is a valid supervisor, we know it will not reach states in $\LESS$. Note that $w$ can't reach $e \step{\l}{\structure'} e'$ because $s$ has no path of $\NONE$ states to $e'$.
	
	If $w=w_0.w_1$ and $s \runw{w_0}{\structure} s' \wedge s' \in \WES$ then $\sigma(w_0.w_1) = \sigma_{s'}(w_1)$ where $\sigma_{s'}$ is the supervisor for $s'$ in $\unexploredToBottom{\structure}$. Note that once we reach $s'$ we always follow $\sigma_{s'}$.
	
	Since $\sigma$ never reaches the new transition we know that $\sigma$ is valid in $\unexploredToBottom{\structure}$.
		
	We see then that assuming there is a valid supervisor $\sigma'$ for $s$ in $\unexploredToBottom{\structure'}$ implies that there is a valid supervisor $\sigma$ for $s$ in $\unexploredToBottom{\structure}$. ABS! \\
	
\end{itemize}


\paragraph*{Proof Lemma \ref{lem:E'isNoneThenAllIsNone}}
\textbf{Begin Sketch}
Assuming $e' \notin \WESS$, we use a witness $s$ to $\WESS \neq 
\WES$ to reach a contradiction. State $s$ must have a supervisor in 
$\WESS$  that avoids $e \step{\l}{} e'$ which is the only difference 
between $\unexploredToBottom{\structure}$ and 
$\unexploredToBottom{\structure'}$. This supervisor is then also a 
supervisor for $s$ in $\WES$ reaching a contradiction. 

If we assume $e' \notin \LESS$.  We use a witness $s$ to $\LESS 
\neq 
\LES$ to reach a contradiction. Note that as $e' \notin \LESS$, there 
is a supervisor $\sigma$ from $e'$ in  
$\unexploredToTop{\structure'}$.  As $s \notin \LES$ it also must 
have a 
supervisor $\sigma'$ in  $\unexploredToTop{\structure}$. We build 
a 
new supervisor for $\unexploredToTop{\structure'}$ from $s$ that 
works just like $\sigma'$ but when it reaches $e \step{\l}{} e'$ it 
behaves as $\sigma$. This new supervisor proves that $s \in \LESS$ 
which is a contradiction. 
\textbf{End Sketch}

We prove both implications by contradiction. 

First we assume $e' \not\in 
\WESS \setminus \WES$. Note that as $e' \notin \WES$ then $e' 
\notin 
\WESS$.  As $\WESS \neq \WES$ by monotonicity 
(Lemma\ref{lem:WESandLesMonotonicity}) there must be a state $s$ 
such that $s \in \WESS \setminus 
\WES$, so $s$ must have a supervisor in $\WESS $. This supervisor 
cannot reach $e'$ because if it did, 
then there would be a supervisor for $e'$ and we had assumed $e' 
\notin \WESS$. Furthermore, if the supervisor reaches $e$, then 
$\l$ 
must be controllable (it if were uncontrollable, the supervisor would 
reach $e'$ which we established is not possible). Thus, the 
supervisor 
avoids $e \step{\l}{} e'$ altogether which means that it is also a 
supervisor for $s$ in $\unexploredToBottom{\structure}$ (i.e.,  $s 
\in 
\WES$) reaching a contradiction.


%
%There must be a 
%supervisor from $s$ that avoids 
%transition $e \step{\l}{} e'$ (which is the only difference between 
%$\structure$ and $\structure'$). We know this because there is no 
%supervisor for $e'$ in $\unexploredToBottom{\structure'}$, 
%otherwise $e'$ would be in $\WESS$. 
%Thus, if $e$ is reachable using that supervisor, the state is 
%controllable and the transition can be avoided. But this means that 
%there is a supervisor for $s$ in 
%$\unexploredToBottom{\structure}$ 
%and finally $s \in \WES$, a contradiction.

Now we assume $e' \not\in 
\LESS \setminus \LES$. Note that as $e' \notin \LES$ then $e' 
\notin 
\LESS$.  
 Let $s \in \LESS$ and $s \not\in \LES$. We know that from $s$ 
 there is a supervisor $\sigma'$ for 
 $\unexploredToTop{\structure}$. This supervisor either is also a 
 supervisor for $\structure$ or reaches the $\top$ state in 
 $\unexploredToTop{\structure}$. In the first case, it is also a 
 supervisor $\structure'$ and in $\unexploredToTop{\structure'}$, 
 reaching a contradiction. In the second case, it either uses a 
 transition that is neither in $\structure'$ nor $\structure$, which 
 means that in $\unexploredToTop{\structure'}$ it will lead to a 
 winning $\top$ state; or it uses the transition that is in 
 $\structure'$ but not $\structure$ which leads to $e'$. Since we 
 know $e'$ is not in $\LESS$, then it has a supervisor for it in 
 $\unexploredToTop{\structure'}$, then we know that there exists a 
 supervisor $\sigma''$ that includes both $\sigma'$ and the 
 supervisor for $e'$. Finally, $\sigma''$ is a supervisor for $s$ in 
 $\unexploredToTop{\structure'}$, but $s \notin \LESS$, ABS!	



\paragraph*{Proof Lemma \ref{lem:canBeWinningLoopWorks}}
\textbf{{Begin Proof Sketch}}
To prove $e' \in \WESS \setminus \WES$ implies  \texttt{canBeWinningLoop($\SCC$)}, we  
assume $\neg$\texttt{canBeWinningLoop($\SCC$)} and show that $e' \notin \WESS 
\setminus \WES$. For this, it suffices to see that if 
$\neg$\texttt{canBeWinningLoop($\SCC$)} then to reach a marked state from $e'$ it must 
exit $\SCC$ to a state $s \notin \SCC$. This state must be such that $s \notin \WESS$. As 
$s$ has no 
supervisor in $\unexploredToBottom{\structure'}$, it is impossible for $e'$ to have one. 


To prove that \texttt{canBeWinningLoop($\SCC$)} implies $e' \notin \LESS$ we build a supervisor $\sigma'$ for $e'$ in $\unexploredToTop{\structure'}$ as 
follows: 
For a trace that stays within $\SCC$, we only choose controllable successors 
that are not 
in $\LES$. Note that there cannot be uncontrollable successors in $\LES$ because 
$\SCC \cap \LES= \emptyset$.   As soon as a trace exits $\SCC$ at a state $s'$ we 
use the supervisor for $s'$ in $\unexploredToTop{\structure'}$. 
As $s'$ cannot reach $e \step{\l}{\structure'} 
e'$ using $\NONE$ states, by 
Lemma~\ref{lem:newWinnersLosersAreNonePredecessors}, $s'$ must have such a 
supervisor. 

\textbf{{End Proof Sketch}}

Supose $e' \in \WESS \setminus \WES$ but $\neg$\texttt{canBeWinningLoop($\SCC$)}.
There exists a supervisor $\sigma$ for $e'$ in $\unexploredToBottom{\structure'}$, this means there must be a path $w$ from $e'$ to a marked state $m$. 
Since $\neg$\texttt{canBeWinningLoop($\SCC$)}, there are no marked state in $\SCC$, $w$ must leave $\SCC$. 
Let $s$ be the first state reached by $w$ not in $\SCC$, $s$ is in $\LES \cup \NONE$, and doesn't have a $\NONE$ path to $e'$ then by Lemma~\ref{lem:newWinnersLosersAreNonePredecessors} $s$ will still not change it status.
Since $s \notin \WESS$, $s$ doesn't have a supervisor in $\unexploredToBottom{\structure'}$, but $\sigma$ accepts runs that lead to $s$, absurd.



Assuming \texttt{canBeWinningLoop($\SCC$)} we simply build a supervisor $\sigma^4$ for $e'$ in $\unexploredToTop{\structure'}$ to prove that $e' \notin \LESS$


We define $\sigma^4$ so that it enables all uncontrollable transitions (to be $controllable$).

$\sigma^4(w_0.w_1) = \sigma_{s'}(w_1)$ if $w_0$ is the shortest word such that there is an $s' \notin \SCC \wedge s'\notin \LES$ and $e' \runw{w_0}{\unexploredToTop{\structure'}} s'$.
Else if $e' \runw{w}{\unexploredToTop{\structure'}} p \wedge p \in \SCC$ then for all controllable $\l'$, $\l' \in\sigma^4(w)$ iff $\exists  p' \ldot p \step{l'}{} p' \wedge p' \notin \LES$.
%todas las transiciones que van al loop mas todas las transiciones que salen a un estado NONE (o TOP), i.e. no error. 

We use the supervisors $\sigma_{s'}$ where $s'$ is such that exists $s \in \SCC$ and $s\step{\l'}{\unexploredToTop{\structure'}} s'\wedge s' \notin \SCC \wedge s'\notin \LES$. If none such $s'$ exists, we know there is a marked state in $\SCC$ and $\sigma^4$ never leaves $\SCC$ since all reachable states from $\SCC$ are in $\LES$.



We will prove that $\sigma^4$ is $controllable$ and $non-blocking$.

Since we enabled all uncontrollable transitions, $\sigma^4$ is trivially $controllable$.

For $non-blocking$, assume $w$ compatible with $\sigma^4$, we will show that it can be extended. If $w = w_0.w_1$ where $w_0$ is the shortest word such that there is an $s' \notin \SCC \wedge s' \notin \LES$ and $e' \runw{w_0}{\unexploredToTop{\structure'}} s'$. Then by definition of $\sigma^4$ we have that $\sigma^4(w_0.w_1.w_2) = \sigma_{s'}(w_1.w_2)$ for all $w_2$. As $\sigma_{s'}$ is $non-blocking$, there is a $w_2$ such that $\sigma_{s'}(w_1.w_2)$ hits a marked state. So $\sigma^4(w_0.w_1)$ can be extended to hit the marked state. 

Otherwise, $w$ never exits $\SCC$. We want to prove that for every $\l'$ such that $w.\l'$ is consistent with $\sigma^4$, $w.\l'$ is extendable with a $w'$ to reach a marked state. Let $p'$ be such that $e' \runw{w.\l'}{} p'$. If $p' \notin \SCC$ then $\sigma^4(w.\l') = \sigma_{p'}(\lambda)$ and, as before, we know $\sigma_{p'}$ is $non-blocking$ so $w.l'$ is extendable to reach a marked state.\\
If $p' \in \SCC$, then  $w.l'$ can be extended to reach any state $s$ in the $\SCC$. We know that either there is a marked state in $\SCC$ or a state in $\WES$ reachable in one step from $\SCC$, either way we can extend $w.l'$ to reach a marked state.\\



\paragraph*{Proof Lemma \ref{lem:findErrorsWorks}}
\textbf{{Begin Proof Sketch}}
We split the proof based on the if/then/else structure of \texttt{findSomeNewErrors}. 
In the case of \texttt{if} being true, it suffices to prove that $e' \notin \LESS$. For this, 
we build a supervisor $\sigma'$ for $e'$ in $\unexploredToTop{\structure'}$ as 
follows: For a trace that stays within $\SCC$, we only choose controllable successors 
that are not 
in $\LES$. Note that there cannot be uncontrollable successors in $\LES$ because 
$\SCC \cap \LES= \emptyset$.   As soon as a trace exits $\SCC$ at a state $s'$ we 
use the supervisor for $s'$ in $\unexploredToTop{\structure'}$. 
As $s'$ cannot reach $e \step{\l}{\structure'} 
e'$ using $\NONE$ states, by 
Lemma~\ref{lem:newWinnersLosersAreNonePredecessors}, $s'$ must have such a 
supervisor. 
 
When the \texttt{if} is false, it suffices to prove $P = \SCC \subseteq \LESS$. We do 
this by assuming $s \in \SCC \setminus \LESS$ and reaching a contradiction. If $s 
\notin LESS$ it has a supervisor $\sigma$ which admits a trace to a marked state. As 
there are no marked states in $\SCC$, the trace reaches a state $s' \not \SCC$ As the 
\texttt{if} was false, $s' \in \LESS$ which means $\sigma$ is not a supervisor.
\textbf{{End Proof Sketch}}


First, we know that every state $s' \notin \SCC$ such that $\exists s \in \SCC \ldot s 
\step{l}{} s'$, either is and will be a losing state ($s' \in LES \wedge s' \in LESS$) or is 
and will be $\NONE$ (because $s$ isn't a $NONE-$Predecessor of any state in 
$\SCC$, otherwise $s$ would be in $\SCC$). \\
This means that any state $s' \notin \SCC \wedge s' \notin \LES$ can not be forced to a state in $\LESS$. Thus, if we reach a $\NONE$ state we know it has a supervisor $\sigma_{s'}$ in $\unexploredToTop{\structure'}$.

In the case that the $\texttt{if}$ statement is true, we prove that $e' \notin \LESS$:

We use the same $\sigma^4$ as in Proof Lemma \ref{lem:canBeWinningLoopWorks}. We already know $\sigma^4$ is both $controllable$ and $non-blocking$ in this situation because of the previos Lemma.\\

Otherwise we enter the $\texttt{else}$ block:

If $\nexists s \in \SCC \ldot \trimlst{s \step{\l'}{\unexploredToTop{\structure'}}  s'} \wedge (s' \notin \SCC \wedge s' \notin \Errors)$ we prove that $\forall s \in \SCC$, $s \in \LESS$

Suppose there is a supervisor $\sigma'$ for $s$ in $\unexploredToTop{\structure'}$ then $\exists w'$ such that $s \runw{\lambda.w'}{\unexploredToTop{\structure'}} e_m \wedge e_m \in M_{\unexploredToTop{\structure'}}$. Since there's no marked state in $\SCC$ then eventually $s$ following $w'$ leaves $\SCC$. \\
Let $w' = w_0.w_1$ such that $w_0$ is the shortest word so that $s \runw{w_0}{\unexploredToTop{\structure'}} s' \wedge s' \notin \SCC$. But since $s'\in \Errors \Rightarrow s' \in \LES$ then it is not possible for a valid supervisor $\sigma'$ to allow reaching that state. ABS! Then there's no supervisor for $s$ in $\unexploredToTop{\structure'}$ implying $\forall s \in \SCC$, $s \in \LESS$.