%- En conclusiones listaría los desafíos que tuvieron
% a) Entender el algoritmo de daniel
% b) Entender el algoritmo estandar
% c) Entender la implementación de daniel. Muy distinta. y el codigo de MTSA en general
% d) Entender control supervisado (composicional y no composicional)
% 
% y contribuciones:
% 
% 1) Tests de regression y harness que muestran los bugs
% 2) Nuevo algoritmo
% 3) Prueba
% 4) Experimentación mostrando que no hay perdida significativa de performance.

Al empezar con el proyecto y leer sobre control supervisado descubrimos que hay todo un mundo detrás. Primero debimos aprender sobre los algoritmos composicionales y no composicionales. Luego entender el algoritmo estándar y parte de su implementación para finalmente poder arrancar con el algoritmo on-the-fly. Éste último lo debimos entender a la perfección, para poder descubrir y solucionar los diversos problemas. 

MTSA es un proyecto con gran trayectoria y muchos avances en diversos frentes hechos por diferentes personas y grupos de investigación; como tal su código puede ser muy complejo, teniendo partes escritas incluso en versiones antiguas de java.\\

Pese a estos desafíos, logramos las siguientes contribuciones: 
\begin{itemize}
	\item Una batería de tests de regresión como una adición permanente al proyecto de MTSA para garantizar la continua correctitud de su feature de síntesis de controladores con exploración heurística.
	
	\item Un nuevo algoritmo de exploración, cuya correctitud es agnóstica a la heurística utilizada.
	
	\item Una prueba de la correctitud y completitud del algoritmo presentado.
	
	\item Resultados experimentales para comprobar que las modificaciones a la exploración siguen manteniendo la buena performance de la técnica.
\end{itemize}

%TODO TRABAJO A FUTURO
% casos de tests generados automáticamente por medio de mutaciones?
