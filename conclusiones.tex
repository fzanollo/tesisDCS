%- En conclusiones listaría los desafíos que tuvieron
% a) Entender el algoritmo de daniel
% b) Entender el algoritmo estandar
% c) Entender la implementación de daniel. Muy distinta. y el codigo de MTSA en general
% d) Entender control supervisado (composicional y no composicional)
% 
% y contribuciones:
% 
% 1) Tests de regression y harness que muestran los bugs
% 2) Nuevo algoritmo
% 3) Prueba
% 4) Experimentación mostrando que no hay perdida significativa de performance.

Al empezar con el proyecto y leer sobre control supervisado descubrimos que hay todo un mundo detrás. Primero debimos aprender sobre los algoritmos composicionales y no composicionales. Luego entender el algoritmo estándar y parte de su implementación para finalmente poder arrancar con el algoritmo on-the-fly. Éste último lo debimos entendera la perfección, para poder descubrir y solucionar los diversos problemas. 

MTSA es un proyecto con gran trayectoria y muchos avances en diversos frentes hechos por diferentes personas y grupos de investigación; como tal su código puede ser muy complejo, teniendo partes escritas incluso en versiones antiguas de java.
