\documentclass{beamer}
\usetheme{Darmstadt}
\usecolortheme{seahorse}
%-------------------------------------------------------
\usepackage[spanish]{babel} % Para separar correctamente las palabras
\usepackage[utf8]{inputenc} % Este paquete permite poner acentos y eñes usando codificación
\usepackage[most]{tcolorbox}

\usepackage{multicol}
\usepackage{listings}
\lstset{
    language=C,
    backgroundcolor=\color{black!5}, % set backgroundcolor
    basicstyle=\footnotesize,% basic font setting
}

\newcommand<>{\reveal}[1]{\mbox{}\visible#2{#1}}

\title[DCS non-blocking]{Directed Controller Synthesis for Non-Maximal Blocking Requirements}
\author[Duran, Zanollo]{Matias Duran, Florencia Zanollo}
\date[12/5/21]{12 de Mayo de 2021}
%-------------------------------------------------------
% INTRO
% intro sobre qué queremos (programa que soluciona problemas por nosotros)
% qué toma de input (autómatas, modelo de la realidad, en general son varios)
% detalles de los autómatas (estado inicial, marcados, etc) con ejemplo de avión
% la composición explota, los algor clásicos rompen acá (i.e. decir que existe algor monolítico)
% idea general: componer mientras exploras
% [buscar] ejemplo simple para explicar composición (mostrar planta completa)

% DEFINICIONES (ANTECEDENTES)
% def pasos y corridas
% controlador
% vemos en particular problemas non-blocking (i.e. loop para ganadores, no-controlables no joden)
% estados ganadores y perdedores (estados desde donde hay una estrategia==controlador ganadora/perdedora)

% ON-THE-FLY
% explicar exploración parcial (lo visto hasta ahora)
% estados ganadores y perdedores ahora (antes era facil, ahora tenemos que ver a dónde van)
% el concepto de ``frontera'' (en el sentido de transiciones existentes sin explorar) top, bottom
% actualizamos estado solo cuando estamos seguros (ganador en bottom / perdedor en top)
% exploramos de a una transición [e -l-> e']
% esto actualiza antecesores de e
% idea central de que ganador necesita loop

% TESTS
% implementación en MTSA, explicación mini de la herramienta
% la exploración es guiada por una heurísitca, hay varias, dos las hizo dany
% hicimos tests, medio TDD

% BENCHMARK
% usamos mismo benchmark que dany
% resultados vs DCS version 1.0, monolítico, y la nuestra (DCS2)

% QUÉ VIMOS (CONCLUSIONES)
% entender el problema
% soluciones existentes, idea nueva
% nuevo algoritmo
% ``demo'' de corrección y completitud
% implementación
% batería de tests
% benchmark confirma que mantiene eficiencia
% esta idea se puede aplicar a otro tipo de problemas (que no sean non-blocking), ej GR1 vuelvan prontos

%-------------------------------------------------------
\begin{document}
%-------------------------------------------------------
\begin{frame}[plain]

\titlepage

\end{frame}
%-------------------------------------------------------
\begin{frame}[plain]{Índice}
    \tableofcontents
\end{frame}
%-------------------------------------------------------
\section{Introducción}% INTRO
% intro sobre qué queremos (programa que soluciona problemas por nosotros)
% qué toma de input (autómatas, modelo de la realidad, en general son varios)
% detalles de los autómatas (estado inicial, marcados, etc) con ejemplo de avión
% la composición explota, los algor clásicos rompen acá (i.e. decir que existe algor monolítico)
% idea general: componer mientras exploras y terminar al tener conclusión (con suerte antes de ver todo)
% [buscar] ejemplo simple para explicar composición (mostrar planta completa)
%-------------------------------------------------------
\begin{frame}{Objetivo}
    \begin{block}{}
    ``The good thing about computers is that they do what you tell them to do. The bad news is that they do what you tell them to do.''\hfill – Ted Nelson 
    \end{block}
    
    \pause
    ¿Es posible hacer que una computadora se diga a sí misma qué hacer?
    
    \pause
    \begin{block}{Síntesis Automática de Controladores}
     Se le brinda a un programa las reglas y objetivos a cumplir, éste sintetiza una estrategia para ganar (si existe) conocida con el nombre de controlador.
    \end{block}

\end{frame}
%-------------------------------------------------------
\begin{frame}{Control de Eventos Discretos}
    \begin{itemize}
     \item Es una de las áreas que estudia problemas de síntesis.
     \item El problema es modelado usando autómatas finitos (o máquinas de estados finitos).
     \pause
     \item Estos autómatas modelan la parte que nos interesa de la realidad.
     \item Se suele partir el modelo en pequeñas partes, más simples de abstraer, y luego se componen para formar el objeto de interés.
    \end{itemize}
\end{frame}
%-------------------------------------------------------
\begin{frame}{Autómatas} 
    [EJEMPLO DE AUTÓMATA]

    detalles de los autómatas finitos (estado inicial, marcados (ver si va), etc)
\end{frame}
%-------------------------------------------------------
\begin{frame}{Composición} 
    ejemplo simple para explicar composición (mostrar planta completa)
    explicar por arriba cómo se compone o cuál es la idea (la planta refleja el comportamiento del todo) (reglas para sincronización entre las partes)
\end{frame}
%-------------------------------------------------------
\begin{frame}{Explosión de estados}
    la composición explota [img de una planta gigante?], los algor clásicos rompen acá (i.e. decir que existe algor monolítico)
\end{frame}
%-------------------------------------------------------
\begin{frame}{Idea general de este trabajo}
    Partiendo de la versión presentada en la tesis doctoral de Daniel Ciolek, queremos componer mientras exploras y terminar al tener conclusión (con suerte antes de ver todo).
    
    Dicha versión presentaba errores a la hora de clasificar los estados, nuestra intención es, no sólo arreglar los errores, sino dar una demostración formal al respecto.
    
\end{frame}
%-------------------------------------------------------

%-------------------------------------------------------
%-------------------------------------------------------
\end{document}
