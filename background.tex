A continuación definimos formalmente el problema composicional de síntesis de controlador nonblocking.

\begin{definition}[Autómata Determinístico] \label{def:automata}
	Un \k{autómata determinístico} es una tupla $T = (S_T, A_T, \D_T, \init{t}, M_T)$, donde:
	\begin{itemize*}[label=]
		
		\item $S_T$ es un \k{conjunto finito de estados};
		
		\item $A_T$ es el \k{conjunto de eventos} del autómata;
		
		\item $\D_T \C (S_T \x A_T \x S_T)$ es una \k{función de transición};
		
		\item $\init{t} \in S_T$ es el \k{estado inicial}; y
		
		\item $M_T \C S_T$ es un conjunto de \k{estados marcados}.
		
	\end{itemize*}
	
\end{definition}

\begin{notation}[Pasos y corridas] \label{not:paso}
	
	$\!\!$Notamos $(t,\l,t') {\in}\!\! \D_T$ como $t \step{\l}{T} t'$ y lo llamamos \k{paso}.
	A su vez, una \k{corrida} de una palabra $w = \l_0,\ldots,\l_k$ en $T$, es una secuencia de pasos tal que $t_i \step{\l_i}{T} t_{i+1}$ para todo $0 \leq i \leq k$, notado como $t_0 \runw{w}{T} t_{k+1}$.
	
\end{notation}


\begin{notation}
	Decimos que un estado $s$ es alcanzado por una palabra $w$ en una corrida comenzando en el estado $s'$, anotado como $s \in w(s')$, cuando $\exists w_0 \ldot \exists w_1 \ldot w = w_0.w_1$ 
	y $s' \runw{w_0}{E} s$
\end{notation}

Los autómatas definen un lenguaje, un conjunto de palabras, que aceptan. Dado un conjunto de eventos $A$, notamos con $A^*$ al conjunto de palabras finitas de eventos de $A$. El lenguaje generado por un autómata $T$ (notado como $\L(T)$) es el conjunto de palabras formadas por sus eventos que cumplen $\D_T$. Formalmente, si $w \in A_T^*$, entonces $w \in \L(T)$ si y solo si existe una corrida para $w$ comenzando desde el estado inicial $\init{t}$ de $T$, que notamos $\init{t} \runw{w}{T} t_{k+1}$.

Generalmente los estados marcados se utilizan para indicar el logro de una tarea. Por lo tanto, consideramos como lenguaje \textit{marcado} (o \textit{aceptado}) por $T$ (lo cual notamos $\L_m(T)$) al conjunto de palabras cuya corrida termina en un estado marcado. Formalmente, sea $w \in \L(T)$, entonces $w \in \L_m(T)$ si y solo si hay una corrida de $w$ que comienza en el estado inicial $\init{t}$ y alcanza un estado marcado $t_m \in M_T$, que notamos $\init{t} \runw{w}{T} t_m$.

Este último concepto es relevante para la definición de la composición paralela. Es un parte fundamental del trabajo para definir las palabras aceptadas por el problema \texttt{nonblocking} ya que el lenguaje aceptado por la planta compuesta es el mismo que el aceptado por cada uno de los componentes por separado.

\begin{definition} [Composición Paralela] \label{def:parcomp}
	La \k{composición paralela} $(\|)$ de dos autómatas $T$ y $Q$ es un operador simétrico y asociativo que produce un autómata $T \| Q = (S_T {\times} S_Q, A_T {\cup} 
	A_Q, \D_{T\|Q}, \<\init{t},\init{q}\>, M_T {\times} M_Q)$, donde $\D_{T\|Q}$ es la menor relación que satisface las siguientes reglas (omitimos la versión simétrica de la primera regla):
	
	\begin{normalsize}
		\centering
		\vspace{-18pt}
		\hspace{-50pt}
		\begin{minipage}{0.30\linewidth}
			\[ 
			\frac{t \step{\l}{T} t'}{\<t,q\> {\step{\l}{T\|Q}} \<t'\!,\!q\> }{{\scriptstyle \l \in A_{T} {\setminus} A_{Q}}} 
			\]
		\end{minipage} 
		\hspace{40pt}
		%\begin{minipage}{0.30\linewidth}
		%\[ 
		%\frac{q \step{\l}{Q} q'}{\<t,q\> \step{\l}{T\|Q} \<t,q'\>}{{\scriptstyle \l \in A_{\!Q} 
		%{\setminus} A_{T}}} 
		%\]
		%\end{minipage} \\[-6pt]
		%\hspace{-20pt}
		\begin{minipage}{0.30\linewidth}
			\[ 
			\frac{t \step{\l}{T} t' \quad q \step{\l}{Q} q'}{\<t,q\> \step{\l}{T\|Q} \<t',q'\>}{{\scriptstyle \l \in A_{T} {\cap} A_{Q}}}
			\]
		\end{minipage} \\[15pt]
	\end{normalsize}
\end{definition}

Hay ciertas características de la composición paralela a destacar. En primer lugar, la última regla realiza la sincronización entre componentes. Segundo, que el número de estados en $T_0 \| ... \| T_n$ puede crecer exponencialmente con respecto al número de autómatas a componer. Tercero, que el lenguaje aceptado por la composición contiene las palabras que alcanzan estados marcados en todos los componentes \textit{simultaneamente}.

\section{Controlador objetivo}

Dado un (conjunto de) autómata(s) y una partición de sus eventos en dos subconjuntos: $controlables$ y $no controlables$, lo que buscamos es un $controlador$ (director) que restrinja el vocabulario aceptado de forma de mantener un camino posible a los estados marcados del autómata.

Un controlador observa las transiciones no controlables y deshabilita alguna transiciones controlables para generar una planta restringida. Una palabra $w$ pertenece al lenguaje generado por $T$ restringido por una función del controlador $\sigma : A_T^* \into 2^{A_T}$ (anotado como $\L^\sigma(T)$) si cada prefijo de $w$ ``sobrevive'' a $\sigma$.
Formalmente, sea $w = \l_0,\ldots,\l_k$ una palabra en $\L(T)$, entonces $w \in \L^\sigma(T)$ si y solo si para todo $0 \leq i \leq k$:
$
\init{t} \run{\,\l_0}{\l_i}{T} t_{i+1} \wedge \l_i \in \sigma(\l_0,\ldots,\l_{i-1})
$

\begin{definition}[Problema de Control Safe y Non-Blocking] \label{def:control-problem}
	Un \k{Problema de Control} con objetivos \k{Safe} y \k{Non-Blocking} composicional es una tupla $\E = (E, A_E^C)$, donde $E$ es un conjunto de autómatas $\{E_0,\ldots,E_n\}$ (podemos abusar la notación y usar $E = (S_E,A_E,\D_E,\init{e},M_E)$ para referirnos a la composición $E_0\|\ldots\|E_n$), y $A_E^C \C A_E$ es el conjunto de eventos controlables (i.e., $A_E^U = A_E \setminus A_E^C$ es el conjunto de eventos no controlables).
	Una solución para $\E$ es un supervisor $\sigma : A_E^* \into 2^{A_E}$, tal que $\sigma$ es:
	\begin{itemize}[itemsep=4pt,topsep=-8pt]
		
		\item \k{Controlable}: $A_E^U \C \sigma(w)$ con $w \in A_E^*$; y
		
		\item \k{Safe y Nonblocking}: para cada palabra $w \in \L^\sigma(E)$ existe una palabra no vacía $w' \in A_E^*$ tal que, la concatenación $ww' \in \L^\sigma(E)$ y $\init{e} \runw{\;ww'}{E} e_m$ con $e_m \in M_E$ (i.e., un estado marcado de $E$).
		
	\end{itemize}
	
\end{definition}

En resumen, un supervisor $\sigma$ es controlable si solo deshabilita eventos controlables, y es safe y nonblocking si restringe el lenguaje generado a palabras que siempre puedan extenderse para llegar a un estado marcado. Notar que no es necesario efectivamente alcanzar un estado marcado, puede haber eventos no controlables que lo eviten, pero siempre debe haber una extensión válida para alcanzar tal estado marcado.

Sabiendo que el objetivo del problema es sintetizar un controlador, queremos distinguir estados \textit{ganadores} (resp. \textit{perdedores}), aquellos estados que podemos incluir (resp. no podemos incluir) en un controlador. 

\begin{notation}
	Decimos que un estado $s$ es ganador["winning"] (resp. perdedor, errores, ["losing"]) en el problema $\E = (E, A_E^C)$ 
	si hay (resp. no hay) una solución para $(E_s, A_E^C)$ donde $E_s$ es el resultado de cambiar el estado inicial de $E$ a $s$.
	Nos referimos como controlador para $s$ en $E$ a una solución de $(E_s, A_E^C)$.
	Nos referimos a los estados ganadores y perdedores de $E$ cuando $A_E^C$ es inferible del contexto, también usamos $W_E$ y $L_E$ para denotar el conjunto de estados ganadores y perdedores de $\E$.
\end{notation}

Podemos pensar en un controlador non-blocking como un jugador optimista. Se encarga de no perder, y solo requiere conocer un futuro camino posible para llegar a un estado ganador.

%Es clave entender que en el problema a tratar, la posición de "tablas" del ajedrez, en la que ambos jugadores repiten sus jugadas 50 veces, se considera ganadora si todavía hay opción de dar un jaque mate. Si repetimos nuestras jugadas y todavía tengo dos torres considero que gané el partido, porque eventualmente mi oponente podría cansarse y dejarme ganar. Si repetimos nuestras jugadas pero solo tengo mi rey, no hay forma de dar mate, no puedo extender esta "palabra", esta partida, de forma de dar mate, y considero que perdí.

Es importante notar que como se busca que cualquier palabra sea extendible a otro estado marcado, lo que se busca es pasar por algún estado marcado infinitas veces. Es decir, un estado 'e' marcado que tenga un camino para que el jugador pueda volver controlablemente al mismo estado 'e'.

Por esto, las estructuras claves que analizamos en nuestro algoritmo son los ciclos ($loops$), ya que los primeros estados ganadores son aquellos que están en un loop controlable con un estado marcado dentro. Luego anotamos como ganadores también a cualquier estado que controlablemente alcanza un estado ganador.

Los ciclos también son esenciales para encontrar los estados perdedores, ya que la única forma de que un estado sea perdedor es que no pueda alcanzar un estado ganador. En otras palabras, los estados perdedores son aquellos que forman parte de un loop que no tiene estados marcados ni transiciones salientes.

%De forma más concreta, en nuestro algoritmo, un ciclo del cual el jugador no puede escapar, pero desde el cual existe un camino hacia un estado ganador, se considera ganador.

\section{Director} \label{chpt:director}
En particular, buscamos como solución al problema de control, controladores que sean directores, como en \textcolor{red}{[REFS Huang]}. Un director se destaca por habilitar a lo sumo un evento controlable en cada punto de la ejecución. 

\begin{definition}[Director] \label{def:director}
	Dado un controlador $\sigma : A_E^* \into 2^{A_E}$ de un problema de control $\E$, decimos que $\sigma$ es un director si $\forall w \in A_E^*$, $\|\sigma(w) \cap A_E^C\| \leq 1$.	
\end{definition}

Esto es en contraste con las soluciones tradicionales de Discrete Event Control y sus herramientas, como \textcolor{red}{poner a SUP?} que presentan supervisores maximales. Los supervisores deben habilitar todos los eventos controlables que sean válidos en algún controlador que cumpla el objetivo del problema. Es decir que un director será un controlador que cumpla el mismo objetivo que un supervisor, pero restringiendo las palabras posibles a un subconjunto del lenguaje aceptado por el supervisor.

El foco en la construcción de directores tiene las siguientes razones:

\begin{itemize}
	\item Los directores pueden ser más apropiados en contextos donde el controlador \textit{ejecuta} las acciones controlables \textcolor{red}{REF}.
	
	\item La construcción de directores puede requerir una menor exploración de la planta que la construcción de un supervisor. Esto se sinergiza y potencia las ganancias en tiempo de la técnica al explorar la composición on-the-fly y permite componer una proporción menor de la planta.
	
	\item En el caso de poder sintetizar un director explorando menos de la planta, se podría usar tanto para controlar la planta como para probar la controlabilidad de un problema donde herramientas de construcción de supervisores fallan por tener que explorar en mayor medida un problema de gran tamaño.
	
	\item Hay hasta la fecha una falta de herramientas disponibles para la síntesis de directores
\end{itemize}

Notar que en \textcolor{red}{[REF 15 paper]} se prueba que un director existe si y solo si un supervisor maximal existe.

\section{Algoritmo monolítico} \label{chpt:algoMono}

Una solución a este problema, anteriormente estudiada~\cite{Ehlers:EECS-2013-162} se basa en un menor punto fijo. Simplemente se comienza con el conjunto de los estados que no tienen ningún camino para alcanzar un estado marcado. Luego en cada iteración se agrega al conjunto de los estados perdedores todos aquellos que en un paso son forzados al conjunto de la iteración anterior. 

Si al concluir el punto fijo el estado inicial no se encuentra en el conjunto entonces existe un controlador para el problema en cuestión y para construirlo se deben evitar las transiciones controlables que llevan al conjunto de estados perdedores.

Presentamos en el listing~\ref{lst:classical} una simplificación del algoritmo monólitico (que resuelve toda la planta ya compuesta).

\begin{lstlisting}[language={pseudocode},label={lst:classical},caption={Algoritmo Monolitico},float=ht]
Algorithm classicalSolver($E, A_E^C$):
	$B = \{s \in S_E \mid \nexists w \ldot \trimlst{s \runw{w}{E} m} \wedge m \in M_E \}$
	$B' = \emptyset$
	while $B' \neq  B:$
		$B' = B$
		$B = B \cup \{s \in S_E \mid $forcedTo($s,e,E$)$\wedge e \in B \}$
	return $\initial \notin B$
\end{lstlisting}

Nuestro problema surge de que para el primer paso, encontrar el conjunto $B$ inicial de estados que no alcanzan un marcado, necesitaríamos conocer los caminos que puede tomar cualquier estado, lo cual implica componer toda la planta.

Sin embargo, utilizamos la idea del punto fijo que detecta errores en la función \texttt{findNewGoalsIn} ya que en ese momento no lo podemos evitar, y simplemente asumimos que lo no explorado no puede llegar a un estado marcado. Esto se discutirá en mayor profundidad en el capítulo~\ref{chpt:dcs}.
