El problema de síntesis de controlador ya tiene una solución clásica, por lo que la dificultad del trabajo no consistió en desarrollar un algoritmo que detectara estados ganadores y perdedores de un LTS totalmente explorado. 

El conflicto reside en que al componer distintos DES, la cantidad de estados de la composición es exponencial respecto de los estados en los componentes. Esto es de suma relevancia ya que la solución clásica, que compone toda la planta para luego explorarla, tiene un límite de escalabilidad en el cual la composición de la planta llega al límite de tiempo o memoria, y nunca se llega a la exploración.

Para combatir esto, la exploración on-the-fly, presentada en \textcolor{red}{REF DANY}, clasifica estados como ganadores o perdedores durante la composición. Se espera que con esto sea posible, en primer lugar, cortar la exploración de una rama de la planta que ya se sabe que es perdedora o ganadora, reduciendo así la memoria y tiempo necesarios. Pero más aún, si el estado inicial fuera marcado como ganador o perdedor antes de la composición completa de la planta, ni siquiera sería necesario completar el proceso de composición.

\begin{lstlisting}[language={pseudocode},label={lst:basic_on-the-fly},caption={Algoritmo on-the-fly básico},float=ht]
Algorithm basicOTF-Exploration($E, A_E^C$):
    $\initial$ = $\<\init{e}^0,\ldots,\init{e}^n\>$
    $\structure \subseteq E$
    until seguroGanaOPierde($\initial$):
        expandirES()
        computarGanadoresYPerdedores($\structure$)
    return armarControlador($\initial$)
        
\end{lstlisting}

En el listing \ref{lst:basic_on-the-fly} mostramos la estructura básica de este método. Trabajando sobre la parte de la planta compuesta hasta el momento (estructura explorada, Explored Structure, $\structure$), se expande $\structure$ consiguiendo nueva información hasta que sea seguro si el estado inicial es ganador o perdedor en $E$. Al llegar a esta conclusión se retorna el controlador para $\initial$ en $E$ o se notifica que no es controlable.

Puede haber muchas variantes de cada una de estas partes, cómo se expande, cómo se computan nuevos ganadores y perdedores, etc. En particular, nuestro enfoque se muestra en el listing \ref{lst:on-the-fly} y consiste en ir agregando una transición a la vez a la parte conocida de la planta, y en cada paso ver si esta nueva transición permite concluir que un estado es ganador o perdedor. Si algún nuevo estado se clasifica como ganador  o perdedor, se propaga esta información a sus antecesores, posiblemente marcándolos a su vez como ganadores o perdedores, respectivamente.

A medida que exploramos mantenemos dos conjuntos de estados de $\structure$ ($\Goals$ y $\Errors$) para los cuáles ya se tiene una conclusión, es decir, se sabe que son ganadores (o perdedores) en $E$.

Como se demostrará en el capítulo \ref{chpt:dcs}, al expandir $\structure$ con una transición a la vez $(e,\l,e')$, a menos que $e'$ ya fuera un ganador/perdedor entonces sólo puede haber nueva información si existe un loop entre $e$ y $e'$. Ésto permite optimizar la detección de nuevos ganadores/perdedores en lugar de ejecutar un algoritmo clásico sobre todo $\structure$.

En el peor caso, no se pudo concluir nada antes de componer la planta en su totalidad, se perdió tiempo en los puntos fijos, intentando clasificar estados, y se realiza una última vez el algoritmo clásico con la planta totalmente explorada. Esto garantiza la completitud del algoritmo, como se detalla en mayor profundidad en el capítulo~\ref{chpt:dcs}.

\begin{lstlisting}[language={pseudocode},label={lst:on-the-fly},caption={Nuestro enfoque on-the-fly},float=ht]
Algorithm genericOTF-Exploration($E, A_E^C$):
   $\initial$ = $\<\init{e}^0,\ldots,\init{e}^n\>$
   $\Goals = \Errors = \emptyset$
   $\structure = initial$ //la parte conocida de la planta
   while $initial \notin \Goals \cup \Errors$:
     $(e,\l,e') = $proxTransicion($\structure, heuristica$)
     expandirES($\structure,(e,\l,e')$)
     if $e' \in \Errors$:
       propagarError($e'$)
     else if $e' \in \Goals$:
       propagarGoal($e'$)
     else if isLoop($e,e'$):
       if nuevoLoopGanador($e,e'$):
         propagarGoal($e'$)
       else if nuevoLoopPerdedor($e,e'$):
         propagarError($e'$)
         
   if $initial \in \Goals$:
     return armarControlador($\Goals$)
   else:
     return "UNREALIZABLE"  
\end{lstlisting}

Para incrementar las ramas podadas se utiliza una heurística de exploración Best First Search \cite{tesisDani} que busca ganar controlablemente o perder no controlablemente, para garantizar con la menor exploración posible que el estado actual es ganador o perdedor.

Una heurística no presenta garantía de resultados perfectos, más bien da una recomendación. Son ampliamante utilizadas al optimizar, pero es importante que la correctitud de los algoritmos no dependan de estas recomendaciones, ya que por su misma naturaleza no tienen garantías fuertes.

A continuación presentamos los puntos más delicados de la exploración on-the-fly, y presentamos las ideas que moldearon nuestro algoritmo para resolver estos puntos conflictivos.

En todos los gráficos siguientes vamos a utilizar las letras $c$, $u$, para denotar si una transición es controlable o no, respectivamente; en caso de no especificar letra para una transición es porque su controlabilidad no afecta el resultado del ejemplo.


\section{Agnosticismo a la heurística}

Una distinción clave del algoritmo \textit{on-the-fly} es que está dividido en dos partes. Por un lado se tiene el algoritmo de exploración responsable de que al final se llegue al resultado correcto, por el otro tenemos una heurística que le brinda la próxima transición a explorar. Ese algoritmo de exploración no puede depender de la heurística, ya que la misma no garantiza siempre elegir el mejor camino posible, sino solo la mejor aproximación que encuentre. Uno de los focos de nuestro trabajo fue en esa corrección independiente del orden de exploración de las transiciones.

El proyecto \texttt{MTSA} inicialmente contaba con dos heurísticas \texttt{Best First Search} para exploración, \textit{Monotonic Abstraction} y \textit{Ready Abstraction}. 

El inconveniente con la versión anterior del algoritmo de exploración es que había sido desarrollado en conjunto con las heurísticas. Si bien esto ayudaba a la eficiencia del mismo, generaba una dependencia del orden de observación de las transiciones, dando resultados erróneos al cambiar las recomendaciones. El nuevo enfoque no depende de la forma de explorar, por ende, da una mayor libertad de investigar a futuro nuevos criterios de evaluación para mejorar la eficiencia de la técnica sin comprometer corrección ni completitud. Esto fue útil durante el desarrollo del trabajo, ya que facilitó la inclusión de nuestras nuevas heurísticas (\texttt{Dummy} y \texttt{Breadth First Search}, ver sección \ref{chpt:heurist-nuevas}) para la experimentación.





\section{Invariante}\label{sct:invariante}

Intentando resolver el problema del marcado explícito de errores, buscamos una separación fuerte entre los estados $error \in L_E$ y los estados para los cuales no tenemos suficiente información para clasificar en este momento $\NONE$. Para esto, los lemas detallados en la sección de demostración del algoritmo aseguran que un estado marcado como ganador o perdedor lo es en la planta compuesta totalmente explorada, y nunca puede cambiar su estado.

Como se verá en la propiedad~\ref{def:invariant} del siguiente capítulo, un estado $s$ solo puede seguir sin clasificar (siendo $\NONE$) si, con los explorados hasta el momento, y siendo totalmente optimistas sobre las transiciones desconocidas no podemos asegurar que $s$ está condenado a ser perdedor (y tampoco podemos concluir que es ganador siendo pesimistas).

Al momento de haber explorado todos los descendientes de $s$, incluso si no se exploró toda la planta total, es claro que no importa si se es optimista ($\top$) o pesimista ($\bot$) para saber si $s$ es un estado ganador o perdedor, por lo que nos forzamos por nuestro invariante a clasificarlo. Con esto evitamos las ramas totalmente exploradas pero sin clasificar que traían complicaciones en la figura~\ref{fig:falenciasErrores} y solo permitimos que un estado sea $\NONE$ si tiene un camino a una transición no explorada.\\



